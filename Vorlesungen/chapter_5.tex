%! TEX root=../algebra.tex
\graphicspath{{Images/}}

\chapter{Modultheorie}
(siehe Seite $288$, aber \enquote{kommutativ})

\section{Definition \& Beispiel}
\enquote{Moduln verhalten sich zu Ringen wie Vektorräume zu Körpern.}

\begin{definition}
	Sei $R$ ein Ring. Ein \emph{$R$-Modul $M$} ist eine abelsche Gruppe gemeinsam mit einer Skalarmultiplikation $R \times M \to M, (a,m) \mapsto a \cdot m$
	mit folgenden Eigenschaften:
	\begin{itemize}
		\item $a \cdot (m_1 + m_2) = a m_1 + a m_2$ für $a \in R, m_1,m_2 \in M$.
		\item $(a+b) \cdot m = a m + b m$ für $a,b \in R, m \in M$.
		\item $a \cdot (b \cdot m) = (a b ) \cdot m$ für $a,b \in R, m \in M$.
		\item $1 \cdot m = m$ für $m \in M$.
	\end{itemize}
\end{definition}

\begin{definition}
	Seien $R$ ein Ring und $M,N$ $R$-Moduln. Wir sagen $\phi: M \to N$ ist \emph{$R$-linear} (ein \emph{Modulmorphismus über $R$})
	falls $\phi$ ein Gruppenmorphismus ist und $\phi(a m) = a \phi(m)$ für alle $a \in R$ und $m \in M$.
\end{definition}

\begin{definition}
	Sei $R$ ein Ring und $M$ ein $R$-Modul. Ein \emph{Untermodul} ist eine Untergruppe $N < M$ mit $a \cdot n \in N$ für alle $a \in R$ und $n \in N$.
\end{definition}

\begin{lemma}
	Sei $R$ ein Ring, $M$ ein $R$-Modul und $N < M$ ein Untermodul. Dann induziert die $R$-Modulstruktur auf $M$ eine $R$-Modulstruktur auf $\sfrac{M}{N}$ so dass
	die kanonische Projektion $\begin{cases}
		\pi: M \to \sfrac{M}{N}\\ m \mapsto m+N
	\end{cases}$ $R$-linear ist.
\end{lemma}

\begin{proof}
	Übung
\end{proof}

\begin{eg}
	\begin{enumerate}
		\setcounter{enumi}{-1}
		\item Falls $R = K$ ein Körper ist, so reden wir genau über \emph{Vektorräume über $K$}.
		\item $M = R$ ist ein $R$-Modul und Ideale $I < R$ sind genau die Untermodulnvon $R$.
		\item Angenommen $M,N$ sind $R$-Moduln. Dann ist auch
			\[
				\hom_{R}(M,N) = \{f: M \to N \mid f \text{ ist $R$-linear}\} 
			\]
			ein $R$-Modul: $(f_1+f_2)(m) = f_1(m) + f_2(m)$ und $(a\cdot f)(m) = a \cdot f(m)$.
		\item Sei $R = \Z$. Dann ist jede abelsche Gruppe $M$ auch ein $\Z$-Modul.
			Falls wir $\Z$-Modul klassifizieren können, so erhalten wir eine Klassifikation von abelschen Gruppen.
			Dies ist eines der \emph{Hauptziele des Kapitels}.
	\end{enumerate}
\end{eg}

\begin{proposition}[Erster Isomorphiesatz]
	Seien $R$ ein Ring, $M,N$ $R$-Moduln, $\phi: M \to N$ $R$-linear.
	Dann sind $\ker(\phi) < M, \Im(\phi) < N$ Untermoduln und $\phi$ induziert einen $R$-linearen Isomorphismus
	\[
		\overline{\phi}: \sfrac{M}{\ker(f)} \to \Im(f)
	.\] 
\end{proposition}

\begin{lemma}
	Seien $R$ ein Ring und $M_1,\ldots,M_{n}$ $R$-Moduln.
	Dann ist auch $M_1 \times \ldots \times M_{n}$ ein $R$-Modul mit koordinatenweiser Skalarmultiplikation
	\[
		a \cdot (m_1,\ldots,m_{n}) = (a m_1,\ldots,a m_{n}) \qq{für} a \in R, (m_1,\ldots,m_{n}) \in M_1 \times \ldots \times M_{n}
	.\] 
\end{lemma}

\begin{proof}
	Übung
\end{proof}

\begin{lemma}
	Seien $R,S$ zwei Ringe, $M$ ein $R$-Modul und $N$ ein $S$-Modul.
	Dann ist $M \times N$ ein $R \times S$-Modul mit koordinatenweiser Skalarmultiplikation
	\[
		(a,b) \cdot (m,n) = (a m, b n) \qq{für} (a,b) \in R \times S, (m,n) \in M \times N
	.\] 
\end{lemma}

\begin{proof}
	Übung
\end{proof}

\textbf{Übung:}
Charakterisiere die Untermoduln von $M \times N$ (über $R \times S$ ).

Welche Ringe könnten interessant sein?
\[
	\text{Körper} \to \text{Vektorräume} \qquad \Z \to \text{Abelsche Gruppen} \qquad K[X] \to ?
\] 
\begin{theorem}
	Sei $K$ ein Körper und $M$ ein Vektorraum über $K$.
	Die Definition einer Modulstruktur auf $M$ über $K[X]$ (die mit der Vektorraumstruktur von $M$ über $K$ kompatibel ist)
	ist gleichzusetzen mit der Auswahl einer $K$-linearen Abbildung $\varphi: M \to M$.
	Formaler formuliert sind die folgenden beiden Abbildungen invers zueinander:
	\begin{center}
		\begin{tabu} to \linewidth {X[2] X[0.2] X[2]}
			Eine Skalarmultiplikation auf $M$ über $K[X]$ dessen Einschränkung auf $K \times M$ die Skalarmultiplikation von $M$ über $K$ ist. & & 
			Eine $K$-lineare Abbildung $ \varphi: M \to M$ \\
			\qquad \qquad \qquad \qquad \qquad  $\cdot $ & $\longmapsto $ & $\varphi(m) = X \cdot m$ für $m \in M$ \\
			$f \cdot m = (f(\varphi))(m) = (\sum_{k} a_{k} \varphi^{k}) (m)$ für $f = \sum_{k} a_{k} X^{k} \in K[X]$ & $\longmapsfrom$ & $\varphi$
		\end{tabu}
	\end{center}
\end{theorem}

\begin{proof}
	Angenommen $\cdot : K[X] \times M \to M$ definiert eine Modulstruktur auf $M$ über $K[X]$. Dann ist $\varphi(m) = X \cdot m$ für $m \in M$ eine 
	$K$-lineare Abbildung auf $M$.
	\[
	\begin{cases}
		\varphi(m_1 + m_2) = X \cdot (m_1 + m_2) = X m\cdot _1 + X \cdot m_2 = \varphi(m_1) + \varphi(m_2)\\
		\varphi(a \cdot m) = X \cdot ( a \cdot m) = (a X) \cdot m = a \cdot \varphi(m)
	\end{cases}
	.\] 
	Angenommen $\varphi: M \to M$ ist $K$-linear. Dann definiert $f = \sum_{k} a_{k} X^{k} \in K[X] \mapsto f(\varphi) = \sum_{k} a_{k} \varphi^{k} \in \hom_{K}(M,M)$
	einen Ringhomomorphismus. Wir verwenden dies um $f \cdot m = (f(\varphi))(m)$ für $f \in K[X]$ und $m \in M$ zu definieren.
	Dies erfüllt die Axiome eines $K[X]$-Moduls:\\
	Z.B. gilt für $f_1,f_2 \in K[X], m \in M$
	\[
		f_1 \cdot (f_2 \cdot m) = f_1(\varphi) (f_2(\varphi) m) = \underbrace{(f_1(\varphi) \cdot f_2(\varphi))}_{= (f_1 \cdot f_2)(\varphi)} = (f_1 \cdot f_2) \cdot m
	.\] 
	wobei $(f_1 \cdot f_2)(\varphi)$ die Auswertung ein Ringhomomorphismus ist.
	Diese beiden Abbildungen sind invers zueinander.
	\begin{align*}
		&\cdot \mapsto  \begin{cases}
			\varphi(m) = X \cdot m\\
			\varphi^2(m) = X^2 \cdot m\\
			\qquad \vdots
		\end{cases} \mapsto f(X) \cdot m = \sum_{k} a_{k} ( \underbrace{X^{k} \cdot m}_{= \varphi^{k}(m)}) = f \cdot m\\
		&\varphi \mapsto \begin{cases}
			\cdot \text{ definiert durch}\\
			f \cdot m = f(\varphi) m
		\end{cases} \mapsto X \cdot m = \varphi(m)
	.\end{align*}
	Wobei $*$ die neue Skalarmultiplikation und $\cdot $ die alte Skalarmultiplikation ist.
	Und außderdem $X\cdot m$ die neue lineare Abbildung und $\varphi(m)$ die alte lineare Abbildung ist.
\end{proof}

Wir wollen endlich erzeugte Moduln über Hauptidealringen klassifizieren!\\
$\overset{\Z}{\longrightarrow}$ Klassifikation von endlich erzeugten abelschen Grupppen.\\
$\overset{K[X]}{\longrightarrow}$ Satz über Jordan Normalform.

\section{Freie Moduln}

\begin{definition}
	Sei $I$ eine Menge und $R$ ein Ring. Wir bezeichnen
	\[
		R^{(I)} = \{x : I \to R \mid x_{i} = 0 \text{ für alle bis auf endlich viele } i \in I\} 
	\] 
	als den \emph{freien $R$-Modul} (über der Indexmenge $I$ ). Wir nennen
	\[
		e_{i} = \mathbbm{1}_{\{i\}} \qq{für} i \in I
	\]
	die \emph{Standardbasis} von $R^{(I)}$.
	Ein \emph{freier Modul} $M$ ist ein Modul isomorph zu $R^{(I)}$ für eine Menge $I$.
	Die Kardinalität von $I$ wird als der \emph{Rang} von $M \cong R^{(I)}$ bezeichnet.
\end{definition}

\begin{lemma}
	Sei $R \neq \{0\} $ ein Ring. Dann ist der Rang eines Moduls wohldefiniert.
\end{lemma}

\begin{proof}
	Sei $J_{\max} \subseteq R$ ein Maximalideal (welches auf grund eines Satzes aus dem Kapitel über Ringe existiert).
	Sei $M$ ein freier $R$-Modul. Dann ist
	\[
	J_{\max} \cdot M = \{\sum_{k} a_{k} m_{k} \mid a_{k} \in J_{\max}, m_{k} \in M\} 
	\] 
	ist ein Untermodul. Sei $I$ eine Menge mit  $M \cong R^{(I)}$. Dann gilt  
	\[
		J_{\max} \cdot M \qq{wird auf} \{\sum_{i \in I} a_{i} e_{i} \mid a_{i} \in J_{\max}, a_{i} = 0 \text{ für alle bis auf endlich viele } i \in I\}
	\]
	abgebildet. Daraus folgt, dass
	\[
		\sfrac{M}{J_{\max} \cdot M} \cong \left( \sfrac{R}{J_{\max}} \right)^{(I)}
	\] 
	ein Vektorraum über $\sfrac{R}{J_{\max}}$ der Dimension $\abs{I}$ ist. 
	Das Lemma folgt nun aus der Linearen Algebra.
\end{proof}

\begin{claim}
	Freie Moduln verhalten sich am ehesten wie Vektorräume \ldots
\end{claim}

\begin{proposition}
	Seien $m,n \geq 1$ natürliche Zahlen und $R$ ein Ring. Dann gilt
	\[
		\hom(R^{n},R^{m}) \cong \mat_{mn}(R)
	\] 
	wie in der Linearen Algebra.
\end{proposition}

\begin{proof}
	wie in der Linearen Algebra.
\end{proof}

\begin{definition}
	Sei $M$ ein $ R$-Modul über einem Ring $R$. Wir sagen $x_1,\ldots,x_{n} \in M$ sind \emph{frei} oder \emph{linear unabhängig} (l.u.) falls die Abbildung
	$a \in R^{n} \mapsto \sum_{i = 1}^{n} a_{i} x_{i}$ injektiv ist.

	Falls $x_1,\ldots,x_{n} \in M$ l.u. sind, so ist das Bild der Abbildung ein freier Untermodul von $M$.
\end{definition}

\section{Torsionsmoduln}
\begin{definition}
	Sei $R$ ein Ring und $M$ ein $R$-Modul. Wir sagen $m \in M$ ist ein \emph{Torsionselement}, falls es ein $a \in R \setminus \{0\} $ gibt
	mit $a \cdot m = 0$. Wir sagen $M$ ist ein \emph{Torsionsmodul} falls jedes $m \in M$ ein Torsionselement ist.
	Wir sagen $M$ ist \emph{torsionsfrei} falls $m = 0$ das einzige Torsionselement von $M$ ist.
\end{definition}

\begin{eg}
	\begin{itemize}
		\item Sei $R = \Z$ und $M = G$ eine additiv geschriebene endliche abelsche Gruppe. Dann ist $M$ ein Torsionsmodul.
			Jedes $g \in M$ hat endliche Ordnung $n < \infty$ womit $n \cdot g = 0$ ist.
		\item Sei $R = \Z$. Dann ist $M = \sfrac{\Q}{\Z}$ ein Torsionsmodul.
		\item Sei $V$ ein endlich-dimensionaler Vektorraum über einem Körper $K$, und $A : V \to V$ $K$-linear.
			Wir verwenden $A$ um $V$ zu einem $K[X]$-Modul zu machen. Dann ist $V$ ein Torsionsmodul über $K[X]$.
			Sei $v \in V \setminus \{0\} $. Dann ist die Abbildung $f \in K[X] \mapsto f \cdot v \in V$
			nicht injektiv (wegen $\dim(V) < \infty = \dim(K[X])$ ).
			Also gibt es ein $f \in K[X] \setminus \{0\} $ mit $f \cdot v = 0$.
		\item Falls $R$ ein Intergritätsbereich ist in $M$ ein freier $R$-Modul ist, so ist $M$ torsionsfrei.
	\end{itemize}
\end{eg}

\section{Struktur von endlich erzeugten Moduln über Hauptidealringe}

\begin{definition}
	Sei $R$ ein Ring und $M $ ein $R$-Modul.
	Für eine Teilmenge $X \subseteq M$ wird
	\[
	\left< X \right>_{R} = \{\sum_{x \in E} a_{x} x \mid a_{x} \in R \text{ für $x \in E$ und $E \subseteq X$ endlich}\} 
	\]
	als die \emph{$R$-lineare Hülle von $X$} oder als der \emph{von $X$ erzeugte Untermodul} bezeichnet.
	Falls es eine Teilmenge $X \subseteq M$ mit $\abs{X} < \infty$ und $\left< X \right>_{R} = M$ gibt, so heißt $M$ \emph{endlich erzeugt}.
\end{definition}

\begin{eg}
	Für $R = K[X_1,X_2,\ldots]$ ist der Untermodul $I = \left< X_1,X_2,\ldots \right>$ nicht endlich erzeugt.
\end{eg}

Wir wollen ab nun nur Hauptidealreinge betrachten - dort wäre jeder Untermodul von $R$ wieder frei mit Rang $0$ oder $1$.

\begin{theorem}[Klassifikationssatz (1. Teil)]
	Sei $R$ ein Hauptidealring und $M$ ein endlich erzeugter Modul über $R$.
	Dann ist $M$ isomorph zu einem direkten Produkt $R^{n} \times T$ wobei
	\[
		T = M_{\on{tors}} = \{m \in M \mid m \text{ ist ein Torsionselement von } M\}
	\]
	und $n$ ist der Rang von $\sfrac{M}{M_{\on{tors}}}$.
	Insbesondere ist $M$ ein freier Modul genau dann wenn $M_{\on{tors}} = \{0\}$.
\end{theorem}

\begin{proposition}
	Sei $R$ ein Hauptidealring und $n \geq 1$. Dann ist jeder Untermodul $M \subseteq R^{n}$ ein freier $R$-Modul mit Rang $\leq n$.
\end{proposition}

\begin{proof}
	Sei $e_{i}$ für $i = 1,\ldots,n$ die Standardbasis von $R^{n}$.
	Wir definieren die Untermoduln
	\[
		M_{i} = M \cap \left< e_1,e_2,\ldots,e_{i} \right> \qq{für} i = 1,\ldots,n
	.\] 
	Dann gilt $M_{n} = M$.
	\begin{claim}
		$M_{i}$ ist ein freier Modul mit Rang $\leq i$.
		Wir beweisen die Behauptung mittels Induktion nach $i$.
		
		\textbf{Induktionsanfang:} Für $i = 1$ ist $M_1 = M \cap \left< e_1 \right> \cong J \subseteq R$.
		Dies zeigt, dass $M_1$ isomorph zu einem Ideal $J$ in $R$.
		Da $R$ ein Hauptidealring ist, folgt entweder $J = (0)$ und $M_{i} = \{0\} $ hat Rang $0$ oder
		$J = (d_1)$ für $d_1 \in R\setminus \{0\}$ und $M_1 \cong (d_1) \cong \underbrace{R}_{\in a \mapsto a d_1}$ hat Rang $1$.

		\textbf{Induktionsschritt:} Angenommen $M_{i-1}$ ist frei mit Rang $\leq i -1$.
		Wir betrachten die Abbildung $\phi: M_{i} \to R, (x_1,\ldots,x_{i},0,\ldots,0) \mapsto x_{i}$.
		Das Bild $\Im(\phi)$ ist ein Untermodul von $R$ also entweder $\Im(\phi) = \{0\}$, und $M_{i} = M_{i-1}$ ist frei mit Rang $\leq i-1 < i$.
		Oder $\Im(\phi) = (d_{i})$ und es gibt ein $m_{i} \in M_{i}$ mit $\phi(m_{i}) = d_{i}$.
		In diesem Fall definieren wir
		\[
			\psi: M_{i-1} \times R \to M_{i} \qquad (m',a) \mapsto m' + a m_{i} \in M_{i}
		.\] 
		Wir zeigen, dass $\psi$ ein Isomorphismus ist.
		Dies impliziert dann, dass $M_{i}$ frei ist und der Rang von $M_{i}$ eins höher ist als der Rang von $M_{i-1}$.

		\textbf{Injektivität von $\psi$:} $\psi(m',a) = 0 = m' + a m \overset{\phi}{\underset{\phi(m') = 0}{\mapsto }} a d_{i}$ und $m' = 0$.

		\textbf{Surjektivität von $\psi$:} Sei $m \in M_{i}$ beliebig, dann ist $\phi(m) \in \Im(\phi) = (d_{i})$ und es existiert
		ein $a \in R$ mit $\phi(m) = a d_{i}$. Damit ist aber $m' = m - a m_{i} \in M_{i-1}$
		und $m = m' + a m_{i} \in \Im(\phi)$.
	\end{claim}
	Dies schließt den Induktionsschritt und damit den Beweis.
\end{proof}

\begin{proof}[Beweis des ersten Teils vom Klassifikationssatz] \leavevmode
	\begin{itemize}
		\item  $M_{\on{tors}}$ ist ein Untermodul, z.B. $a_1 m_1  = 0 = a_2 m_2$, $m_1, m_2 \in M_{\on{tors}}$ für $a_1,a_2 \in R \setminus \{0\} $ 
			Impliziert $a_1 a_2 (m_1 + m_2) = a_2 \underbrace{a_1 m_1}_{= 0} + a_1 \underbrace{a_2 m_2}_{= 0} = 0$.
		\item Da $R$ ein Integritätsbereich ist, hat ein freier Modul ($\cong R^{l}$ )  keine nichttrivialen Torsionselemente.
			Also gilt $M$ frei $\implies M_{\on{tors}} = \{0\}$.
		\item Wir zeigen nun die Umkehrung dieser Aussage, also $M_{\on{tors}} = \{0\} \implies M$ ist frei.\\
			Seien $x_1,\ldots,x_{n} \in M$ eine Erzeugendenmenge von $M$.
			Wir wählen aus dieser Liste eine maximale l.u. Teilmenge $y_1\ldots,y_{k}$ aus.
			Dies impliziert $N = \left< y_1,\ldots,y_{k} \right> \cong R^{k}$.
			\begin{claim}
				Für alle $x_{i}$ in der Erzeugendenmenge gibt es ein $a_{i} \in R \setminus \{0\} $ mit $a_{i} x_{i} \in N$.
				Falls $x_{i} = y_{i}$ ein Erzeuer von $N$ ist, so setzen wir $a_{i} = 1$. 
			\end{claim}
			\begin{proof}[Beweis von $M$ ist frei mittels der Behauptung]
				Für $a  = a_1 \cdot a_2 \cdot \ldots \cdot a_{n}$ gilt auf Grund der Behauptung $a M \subseteq N \cong R^{k}$.
				Also ist $a M$ isomorph zu einem Untermodul von $R^{k}$ und wegen der Proposition selbst frei.
				Des Weiteren ist $a \cdot : M \to aM$ ein Isomorphismus und daher ist auch $M$ frei,
				weil $\ker(a\cdot ) = \{ m \in M \mid am = 0\} \subseteq M_{\on{tors}} = \{0\} $
			\end{proof}

			\begin{proof}[Beweis der Behauptung]
				Sei $x_{i}$ ein Erzeuger von $M$ ungleich $y_1,\ldots,y_{k}$.
				Wir definieren $\varphi: R \times N \to M, (a,m) \mapsto a x_{i} + m$.
				Dann kann $\varphi$ nicht injektiv sein. Denn wenn $\varphi$ injektiv wäre, so wäre $\Im(\varphi)$ frei mit Rang $k+1$ 
				und $y_1,\ldots,y_{k}$ wäre nicht maximal gewesen. Es gibt also $(a,m) \neq 0$ mit $a x_{i} + m = 0$.
				Falls $a = 0$, so wäre auch $m = 0$.
				Also gilt $a+0$ und $a x_{i} \in N$ und die Behauptung gilt.
			\end{proof}
			Dies beweist die Äquivalenz: $M$ ist frei $\Leftrightarrow M_{\on{tors}} = \{0\} $.
	\end{itemize}
	Sei nun $M$ ein beliebiger endlich erzeugter $R$-Modul. Dann ist  $M' = \sfrac{M}{M_{\on{tors}}}$ ebenso endlich erzeugt.
	Des Weiteren ist $M'$ torsionsfrei (also frei wegen obiger Aussage).
	Sei $m + M_{\on{tors}} \in M'$ ein Torsionselement und $a \in R \setminus \{0\} $ mit $a (m + M_{\on{tors}}) = 0 + M_{\on{tors}}$.
	Dies impliziert also $a m \in M_{\on{tors}}$. Also existiert ein $b \in R \setminus \{0\} $ mit $b a m = 0$.
	Folgt $\underbrace{(ab)}_{\in R \setminus \{0\}} \cdot m = 0 \implies m \in M_{\on{tors}}$ und $m + M_{\on{tors}} = 0 + M_{\on{tors}}$.

	Wir erhalten also für einen beliebigen endlich erzeugten Modul $M$, dass $\sfrac{M}{M_{\on{tors}}} \cong R^{n}$ ein freier Modul ist.

	Angenommen $x_1 + M_{\on{tors}}, \ldots, x_{n} + M_{\on{tors}} \in \sfrac{m}{M_{\on{tors}}}$ sind freie Erzeuger von $\sfrac{M}{M_{\on{tors}}}$.
	Dann sind auch $x_1,\ldots,x_{n}$ in $M$ frei (%TODO missing)
	): Falls $\sum_{i=1}^{n} a_{i} x_{i} = 0$ in $M$ ist, so ist
	\[
		\sum_{i=1}^{n} a_{i} (x_{i} + M_{\on{tors}}) = 0 \implies a_1 = a_2 = \ldots = 0
	.\] 
	Wir definieren $\psi: (a,m') \in R^{n} x M_{\on{tors}} \mapsto  \sum_{i=1}^{n} a_{i} x_{i} + m' \in M$ und behaupten,
	dass $\psi$ ein Isomorphismus zwischen $R^{n} \times M_{\on{tors}}$ und $M$ darstellt.

	\textbf{Injektiv:} Angenommen $\psi(a,m') = \sum_{i=1}^{n} a_{i} x_{i} + m' = 0 \implies \sum_{i=1}^{n} a_{i} (x_{i} + M_{\on{tors}}) = 0 \implies a = 0$ \& $m' = 0$ 
	Also ist $\psi$ injektiv.

	\textbf{Surjektiv:} Sei $m \in M$ beliebig. Dann gibt es ein $a \in R^{n}$ mit $m + M_{\on{tors}} = \sum_{i=1}^{n} a_{i} (x_{i} + M_{\on{tors}})$.
	Also ist $m - \sum_{i = 1}^{n} a_{i} x_{i} = m' \in M_{\on{tors}}$ und $\psi(a,m') = m$.
	Damit ist $\psi$ auch surjektiv.
\end{proof}

\begin{theorem}[Klassifikationssatz (2. Teil)]
	Sei $R$ ein Hauptidealring und $M_{\on{tors}}$ ein endlich erzeigter Torsionsmodul. 
	Dann existieren $d_1 \mid d_2 \mid \ldots \mid d_{n}$ in $R \setminus \{0\} $ so dass
	\[
		M_{\on{tors}} = \sfrac{R}{(d_1)} \times \ldots \times \sfrac{R}{(d_{n})}
	.\] 
	Alternativ gilt
	\[
		M_{\on{tors}} \cong \prod_{j=1}^{k} M_{\on{tors}}^{(p_{j})}
	\]
	wobei $p_1,\ldots,p_{k} \in R$ inäquivalente Primzahlen in $R$ sind und %TODO was ist inäquivalent ist das richtig
	\[
		M_{\on{tors}}^{(p_{j})} = \{m \in M_{\on{tors}} \mid \text{ es existiert ein $l \in \N$ mit } p_{j}^{l} m = 0\} \cong 
		\sfrac{R}{(p_{j}^{n_{j,1}})} \times \ldots \times \sfrac{R}{(p_{j}^{n_{j,n}})}
	.\] 
\end{theorem}

\begin{theorem}[Smith Normalform]
	Sei $R$ ein Hauptidealring, $k,l \geq 1$ natürliche Zahlen und $A \in \mat_{kl}(R)$.
	Dann existieren $g \in \GL_{k}(R)$ und $h \in \GL_{l}(R)$ so dass
	\[
	g A h^{-1} = \begin{pmatrix} 
		d_1 \\
		& \ddots \\
		& & d_{n} \\
		& & & 0 \\
		& & & & \ddots
	\end{pmatrix} 
	\]
	für $d_1 \mid d_2 \mid \ldots \mid d_{n}$ in $R \setminus \{0\} $.
\end{theorem}

\begin{itemize}
	\item Wir beweisen diesen Satz nur für Euklidische Ringe.
	\item Im Gauss'schen Eliminationsalgo entsprechen Zeilenoperationen einer Linksmultiplikation und Spaltenoperationen einer Rechtsmulitplikation.
	\item Wir kombinieren Gauss mit Division mit Rest.
	\item Falls $R = K$ ein Körper ist, so können wir $d_1 = d_2 = \ldots = d_{n} = 1$ annehmen und $n = $ Rang von  $A$.
\end{itemize}

\begin{proof}[Beweis für Euklidische Ringe]
	Wir beweisen den Satz mittels doppelter Induktion und zuerst nach $\max(k,l)$.
	\begin{itemize}
		\item Falls $\max(k,l) = 1$ ist, so ist $k=l=1$ und $A = (0)$ oder $A = (d_1)$ für $d_1 \in R \setminus \{0\} $.
		\item Wir können annehmen, dass $\max(k,l) > 1$ ist und der Satz für \enquote{kleinere} Matrizen bereits gilt.

			Falls $A = 0$ ist, so gibt es nichts zu beweisen.

			Also können wir annehmen, dass $A \neq 0$. Wir definieren die \enquote{Norm von $A$} durch 
			\[
				N = \min_{A_{ii} \neq 0} \phi(A_{ii}) \in \N,
			\] 
			wobei $\phi: R \setminus \{0\} $ die Euklidische Normfunktion von $R$ bezeichnet.

			Durch Vertauschung von Zeilen und Spalten können wir annehmen, dass 
			\[
				d_1 = A_{11} \neq 0 \qq{und} \phi(d_1) = N
			.\] 
			Wir verwenden Division durch $d_1$ mit Rest und erhalten
			\[
				A_{1 j} = a_{j} d_1 + r_{j} \qq{für} i = 2,\ldots,l \qq{mit} r_1 = 0 \qq{oder} \phi(r_{j}) < \phi(d_1)
			.\] 
			Wir ziehen das $a_{j}$-fache der 1. Spalte von der $j$-ten Spalte für $j=2,\ldots,l$ ab und erhalten die Matrix
			\[
			A' = \begin{pmatrix} 
				d_1 &r_2 &r_3 &\ldots &r_{l}\\
				A_{21}\\
				\vdots & &*\\
				A_{k_1}
			\end{pmatrix} 
			.\]
			Falls $r_{j} \neq 0$ für ein $j \in \{2, \ldots, l\} $, so ist $N' = \min_{A'_{ij}\neq 0} \phi(A'_{ij} < N$ und wir
			können per Induktion annehmen, dass $A'$ (und damit auch $A$ ) eine Smith Normalform hat.
			Also können wir annehmen, dass $r_2 = r_3 = \ldots = r_{l} = 0$ ist.
			
			Ananlog können wir dieses Argument nun auch für die erste Spalte wiederholen.
			Damit erhalten wir den verbleibenden Fall
			\[
			A'' = \begin{pmatrix} 
				d_1 &0 &\ldots &0\\
				0\\
				\vdots & &B\\
				0
			\end{pmatrix} 
			.\]
			Falls $\min(k,l) = 1$, dann ist $A''$ bereits die Smith Normalform von $A$.
			Falls $B = 0$, dann ist $A''$ bereits die Smith Normalform von  $A$.

			Ansonsten hat $B$ Dimension $k-1$ \& $l-1$. Also hat nach Induktionsannahme $B$
			eine Smith Normalform. Also können wir nach weiteren Zeilen- und Spaltenoperationen eine Matrix der Form
			\[
			A''' = \begin{pmatrix} 
				d_1 \\
				& \ddots \\
				& & d_{n} \\
				& & & 0 \\
				& & & & \ddots
			\end{pmatrix} 
			\] 
			erreichen.

			Falls $d_1 \mid d_2$ (und ebenso $d_2 \mid d_3 \mid \ldots \mid d_{n}$ nach Induktionsannahme), dann ist $A'''$ die Smith Normalform von $A$.

			Falls $d_1 \nmid d_2$, so addieren wir die zweite Zeile zur ersten, verwenden Divisioon mit Rest und derhalten eine Matrix mit kleinerem Minimum.
			\[
			A''' \longmapsto \begin{pmatrix} 
				d_1 &d_2\\
				& \ddots \\
				& & d_{n} \\
				& & & 0 \\
				& & & & \ddots
			\end{pmatrix} \mapsto \underbrace{\begin{pmatrix} 
				d_1 &r \\
				& \ddots \\
				& & d_{n} \\
				& & & 0 \\
				& & & & \ddots
		\end{pmatrix}}_{\text{Matrix mit kleinerer Norm}} \underset{i.A.}{\longmapsto} A''''
			\] 
			wobei $A''''$ in Smith Normalform und $r$ der Rest bei Division durch $d_1$ ist.
	\end{itemize}
	Dies schließt die Induktion(en) und den Beweis.
\end{proof}

\begin{proof}[Beweis beider Teile des Klassifikationssatzes]
	Sei $M$ ein endlich erzeugter $R$-Modul und $R$ ein Euklidischer Ring.
	Angenommen $x_1,\ldots,x_{k} \in M$ erzeugen $M$. Dann ist
	\[
	\phi: a \in R^{k} \mapsto \sum_{i=1}^{k} a_{i} x_{i} \in M
	\] 
	surjektiv. Sei $N = \ker(\phi) \subseteq R^{k}$- ein Untermodul. Nach einer früheren Proposition wissen wir, dass $N$ selbst
	auch ein freier Modul ist -  sei $N = \left< r_1,\ldots,r_{l} \right>$. Damit ist $M \cong \sfrac{R^{k}}{N}$ (induziert von $\phi$ ).

	Wir definieren die Matrix
	\[
		A = (r_1,\ldots,r_{l}) \in \mat_{kl}(R)
	\] 
	und wenden den Satz über die Smith Normalform an: Es existieren Matrizen $g \in \GL_{k}(R)$ und $h \in \GL_{l}(R)$ so dass
	\[
	B = g A h^{-1} = \begin{pmatrix} 
		d_1 \\
		& \ddots \\
		& & d_{n} \\
		& & & 0 \\
		& & & & \ddots
	\end{pmatrix} \qquad \qquad d_1 \mid d_2 \mid \ldots \mid d_{n} \qq{in} R \setminus \{0\}  
	.\] 
	Da wir $A$ mit einer $R$-linearen Abbildungen $R^{l} \to R^{k}$ identifizieren können erhalten wir
	\begin{align*}
		&N = \Im(A) = A(R^{l})\\
		&\Im(B) = B(R^{l}) = g a \underbrace{h ^{-1}(R^{l})}_{R^{l}} = g \Im(A) = g N
	.\end{align*}
	Des Weiteren erhalten wir
	\begin{align*}
		R^{k} &\overset{\sim^{g}}{\longrightarrow} R^{k}\\
		N = \Im(A) &\overset{\sim^{g}}{\longrightarrow} g N = \Im(B)
	.\end{align*}
	Verwende man den Isomorphiesatz indem man die Abbildung
	\[
		\begin{tikzcd}
			R^k \arrow[r, "g"'] \arrow[rr, "\psi", bend left] & R^k \arrow[r] & \sfrac{R^k}{gN}
		\end{tikzcd}
	.\] 
	betrachtet folgt $\sfrac{R^{k}}{\ker(\psi)} \cong \sfrac{R^{k}}{gN}$ und $\ker(\psi) = N$ da $g$ bijektiv ist.
	\[
		M \cong \sfrac{R^{k}}{N} \overset{\sim}{\longrightarrow} \sfrac{R^{k}}{gN} = \sfrac{R^{k}}{\Im(B)} \cong \sfrac{R}{(d_1)} \times  \ldots \times \sfrac{R}{(d_{n})} \times  R^{k-n}
	\] 
	und
	\[
		\Im(B) = (d_1) \times (d_2) \times \ldots \times (d_{n}) \times  \{0\}^{k-n}
	.\]
	wobei die erste Kongruenz von $\phi$ und die derste Abbildung von  $g$ induziert ist.
	Außerdem gilt wegen dem Isomorphismus
	\begin{align*}
		&\sfrac{R \times R}{(d_1) \times (d_2)} \cong \sfrac{R}{(d_1)} \times \sfrac{R}{(d_2)}\\
		&R \times R \overset{\psi}{\longrightarrow} \sfrac{R}{(d_1)} \times \sfrac{R}{(d_2)}\\
		&\ker(\psi) = \{(a_1,a_2) \mid a_1 \in (d_1), a_2 \in (d_2)\} = (d_1) \times (d_2)
	.\end{align*} 
\end{proof}

\section{Endlich erzeugte abelsche Gruppen}
\begin{theorem}
	Sei $G$ eine endlich erzeugte (additiv geschriebene) abelsche Gruppe.
	Dann gilt
	\[
		G \cong \sfrac{\Z}{(d_1)} \times \ldots \times \sfrac{\Z}{(d_{n})} \times \Z^{k}
	\] 
	wobei $1 \leq s_1 \mid d_2 \mid \ldots \mid d_{n} \neq 0$ und $k \geq 0$.
	
	Alternativ gilt
	\[
		G \cong \prod_{p > 0 \text{ prim}} G_{p} \times \Z^{k} \qq{und} G_{p} \cong \sfrac{\Z}{(p^{k_{p,1}})} \times \ldots \times \sfrac{\Z}{(p^{k_{p,n}})}
	.\] 
	wobei $G_{p}$ die Sylow $p$-Untergruppe ist.
\end{theorem}

\begin{proof}
	Folgt aus dem Klassifikationssatz
\end{proof}

\section{Jordan-Normalform}
\begin{theorem}
	Sei $V$ ein endlich dimensionaler Vektorraum über $\C$ und $\varphi: V \to  V$ linear.
	Dann existiert eine Basis von $V$, so dass $\varphi$ eine Matrixdarstellung der folgenden Form besitzt:
	\[
	\begin{pmatrix} 
		J_1\\
		&J_2\\
		& & \ddots
	\end{pmatrix} \qq{und jeder Block $J_{k}$ hat die Form} \begin{pmatrix} 
		\lambda &1\\
		&\lambda &1\\
		& & \ddots &\ddots\\
		& & & \ddots &1\\
		& & & &\lambda
	\end{pmatrix} 
	.\]
	Dies ist die Jordan-Normalform von $\varphi$.
\end{theorem}

\begin{proof}
	Da $V$ ein endlich-dimensional ist und $\C[X]$ unendlich-dimensional ist, muss $V$ ein Torsionsmodul über $\C[X]$ sein 
	(wobei wir $\varphi$ verwenden um die Modulstruktur zu definieren).
	Ebenso ist $V$ als $\C[X]$-Modul endlich erzeugt weil $V$ endlich-dimensional ist.

	Also können wir den Klassifikationssatz für Module anwenden und erhalten
	\[
		V \cong \prod_{\substack{\text{endlich}\\\text{viele } (\lambda,k)}} \sfrac{\C[X]}{((X-\lambda)^{k})} 
	\] 
	mit $V$ aufgefasst als $\C[X]$-Modul.

	Wir beschreiben nun Multiplikation mit $X$ ($\cong$ Anwendung von $\varphi$ auf Teilräumen von $V$ ) auf
	$\frac{\C[X]}{((X-\lambda)^{k})} = M$. $M$ hat über $\C$ die Basis
	\[
		1, (X-\lambda), (X-\lambda)^2,\ldots, (X-\lambda)^{k-1}
	.\] 
	und $X \cdot $ hat die folgende Matrixdarstellung bzgl. dieser (geordneten) Basis
	\[
		\begin{pmatrix} 
			\lambda\\
			1 & \lambda\\
			  & 1 &\ddots\\
			  & &\ddots &\ddots\\
			  & & & 1 &\lambda
		\end{pmatrix} 
	\]
	denn
	\begin{align*}
		&X \cdot 1 = X = (X - \lambda) + \lambda \cdot 1 \implies \text{ bestimmt die 1. Spalte}\\
		&X \cdot (X - \lambda)^{j} = (X-\lambda)^{j+1} + \lambda (X - \lambda)^{j} \implies \text{ bestimmt die $(j+1)$. Spalte}\\
		&\vdots\\
		&X \cdot (X-\lambda)^{k-1} = (X - \lambda + \lambda) (X-\lambda)^{k-1} = \cancel{(X-\lambda)^{k}} + \lambda (X-\lambda)^{k-1} = 0 \text{ in } M
	.\end{align*}
	Nach Umordnung der Basisvektoren ($(X-\lambda)^{k-1}$ zuerst, $1$ zuletzt) ergibt sich ein Jordanblock wie im Satz.
\end{proof}




















