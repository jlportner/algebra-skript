%! TEX root=../algebra.tex
\graphicspath{{Images/}}

\chapter{Galois Theorie}

\section{Einleitung}
Das motivierende Problem der Galois Theorie ist folgendes:
Finde eine \enquote{Formel} für die Lösungen der Gleichung
$x^{n} + a_{n-1} x^{n-1} + \ldots + a_0 = 0$ in Funktion von den Koeffizienten $a_0,\ldots.,a_{n-1}$.

Methoden für den linearen und quadratischen Fall waren schon babylonischen Mathematikern bekannt.  $\sim 1700$ B.C.\\
Euklid ($\sim 300$ B.C.) hat die Lösung von Quadratischen Gleichungen auf geometrische Probleme zurückgeführt.\\
al-Khwarizmi ($780 - 850$): Systematische Behandlung von linearen und quadratischen Gleichungen.\\
16. Jh: Gleichung 3. Grades: Seipione del Ferro 1515. 4.Grades: Ludovico Ferrarr.

Cardano \enquote{Ars Magna} 1545: Cardano's Formeln für  3. Grad.
Sei $x^{3} + a x^{2} + bx + c = 0$. Durch die Substitution $z = x - \frac{a}{3}$ erhält man eine Gleichung der Form:
$z^3 + p z + q = 0$.

Idee: $z = y + u$ wobei man später  $u$ geeignet wählen kann. Durch Substitution in $z^3 + pz + q = 0$ erhalten wir:
\[
	y^3 + \underbrace{2y^2 u + 3yu^2}_{3yu (y+u)} + u^3 + p(y+u) + q = 0
\] 
und erhalten $y^3 + (y+u)(3yu + p) + u^3 + q = 0$.
Setze $3yu +p = 0$ also $u = -\frac{p}{3y}$.
\[
	y^{4} - \frac{p^3}{27 y^3} + q = 0 \implies y^{6} + p y^3 - (\frac{p}{3})^3 = 0 \quad (\text{Resolvente})
.\] 
Diese Gleichung ist quadratisch in $y^3$:
\[
	y^3 = \frac{-q \pm \sqrt{q^2 + 4 (\frac{p}{3})^3} }{2}
.\] 
und bekommt für $z$ die Formel:
\[
	z = \sqrt[3]{-\frac{p}{2} + \sqrt{\left(\frac{q}{2}\right)^2 + \left(\frac{p}{3}\right)^3}} + \sqrt[3]{-\frac{p}{2} - \sqrt{\left(\frac{q}{2}\right)^2 + \left(\frac{p}{3}\right)^3}} 
.\] 

Wesentlicher Schritt: Lagrange (1736-1813): Falls $z_1, z_2, z_3$ Lösungen von $z^3 + pz + q = 0$ sind.
Sind $w = e^{\frac{2}{3}\pi i }$ primitive 3. Wurzeln von $1$. Dann sind die $6$ Lösungen der Resolvente
$y^6 + q y^3 - \left( \frac{p}{3} \right)^3 = 0$ sind gegeben durch
\[
	y_{\sigma} := \frac{1}{3}\left( z_{\sigma(1)} + w z_{\sigma(2)} + w^2 z_{\sigma(3)} \right) 
\] 
wobei $\sigma$ die Menge der Permutationen über $3$ Elemente durch läuft.

Fundamentale Einsicht: $\left( z_{\sigma(1)} + w z_{\sigma(2)} + w^2 z_{\sigma(3)} \right)^3$ nimmt nur $2$ Werte an.

Paolo Raffini: Zeige dass die allgemeine Gleichung 5. Grades keine \enquote{Lösung} besitzt.
Rationale Funktionen $f(z_1,\ldots,z_{5})$ wobei $z_1,\ldots, z_5$ Wurzeln der Gleichung $z_5 + \ldots + a_0 = 0$ sind.
Hat realisiert, dass die Menge der $\sigma \in S_{5}$ für welche $f(z_1,\ldots,z_5) = f(z_{\sigma(1)},\ldots,z_{\sigma(5)})$ 
ist eine \emph{Untergruppe} von $S_{5}$.

Untergruppen von $S_{5}$ klassifiziert. Niels Abels (1812-1829)

\begin{theorem}[Abels-Raffini]
	Die allgemeine Gleichung $5.$ Grades $x^{5} + a x^{4}  + b x^3 + c x^2 + d x + e = 0$ ist mittels Radikalen nicht auflösbar.
\end{theorem}
Eine Lösung mittels Radikalen ist eine \emph{Formel} die endlich viele arithmetische Operationen und Wurzelziehen der Koeffizienten zulässt.

Galois Theorie und Thm. Die alternierende Gruppe $A_{5}$ ist nicht abelsch und einfach.

Wir werden jedem Polynom $f(x) = x^{n} + a_{n-1} x^{n-1} + \ldots + a_0 \in K[x]$, $K$ Körper
ordnen wir eine Gruppe $\gal(f) < S_{n}$.

\begin{theorem}
	Falls $K$ gute Eigenschaften besitzt (z.B. $\charak = 0$)
	$f(x) = 0$ ist genau dann Mittels Radikalen Lösbar falls $\gal(f)$ \emph{auflösbar}.
\end{theorem}

\section{Galois Gruppe einer Körpererweiterung: grundlegende Eigenschaften und Beispiele}
Sei $E$ ein Körper. Die Menge $\aut(E) = \{\sigma: E \to E \mid \sigma \text{ ist eine Körperisomorphismus}\} $ 
ist für die Operation der Verkettung von Abbildungen eine Gruppe.

Sei $K \subseteq E$ eine Unterkörper; $E$ ist eine Körpererweiterung von $K$.
\[
	\gal(E / K) = \{\sigma \in \aut(E) \mid \sigma(x) = x \forall x \in K\} 
\]
ist eine Untergruppe von $\aut(E)$.
\begin{definition}
	$\gal(E / K)$ ist eine Galoisgruppe der Erweiterung $E / K$.
\end{definition}
Aus der Algebra I wissen wir, dass $E$ ein $K$-Vektorraum ist.

\textbf{Übung:} Jedes $\sigma \in \gal(E / K)$ ist ein Isomorphismus des $K$-Vektorraums $E$.

\textbf{Übung:} Sei $K = \R$ und $E = \C$ dann ist $\gal(\C / \R) = \{\id_{\C}, \sigma\} $ wobei $\sigma(x+iy) = x-iy, x,y \in \R$.
Wie \emph{groß} ist $\aut(\C)$.

Sei $f \in K[x]$ ein Polynom und $E / K $ eine Körpererweiterung so dass in $E[x] f$ Produkt von linearen Faktoren ist.
Sei $R(f) \subseteq E$ die Menge der Nullstellen von $f$.

\begin{lemma}
	Jedes $\sigma \in \gal(E / K)$ induziert eine Permutation der Menge $R(f)$ der Nullstellen von $f$.
\end{lemma}

\begin{proof}
	Sei $\alpha \in R(f)$ d.h. $f(\alpha) = 0$ und $\sigma \in \gal(E / K)$.
	Sei $f(X) = a_{n} X^{n} + a_{n-1} X^{n-1} + \ldots + a_0$ wobei $a_{n},\ldots,a_0 \in K$.
	\begin{align*}
		0 = f(x) = \sigma(f(x)) = \sigma(a_{n} \alpha^{n} + \ldots + a_0) = \sigma(a_{n}) \sigma(\alpha)^{n} + \ldots + \sigma(a_0) 
		= a_{n} \sigma(\alpha)^{n} + \ldots + a_0 = f(\sigma(\alpha))
	\end{align*}
	Also folgt $\sigma(\alpha) \in R(f)$.
	$\sigma(R(f)) \subseteq R(f)$. Da $\sigma: E \to E$ injektiv und $\abs{R(f)} < \deg(f) = n$ folgt $\sigma(R(f)) = R(f)$.
\end{proof}

Sei $f \in K[X]$.
\begin{definition}
	Die Galois Gruppe $\gal(f)$ von $f$ ist die Galois Gruppe $\gal(E / K)$ wobei $E / K$ ein Zerfällungskörper
	von $f$ bezeichnet.
\end{definition}
Existenz: Kronecker + Eindeutigkeit bis auf Isomorphismus siehe Algebra I

\textbf{Übung:} Zeige dass falls $E / K$ und $E' / K$ Zerfällungskörper von $f$ bezeichnen, die Gruppen
$\gal(E / K)$ und $\gal(E'/K)$ isomorph sind.

\begin{notation}
	Sei $X$ eine Menge. Wir bezeichnen mit $S_{X}$ die Gruppe aller Bijektionen (Permutationen) von $X \to X$.
	Falls $X = \{1,2,\ldots,n\}$ dann setzen wir $S_{X} = S_{n}$.
\end{notation}

\begin{lemma}
	Sei $E / K$ Zerfällungskörper eines Polynoms $f \in K[X]$ und $R(f) \subseteq E$ die Menge der Nullstellen.
	Dann ist die Restriktionsabbildung 
	\[
		\nstack{\gal(E / K) \to S_{R(f)}}{\sigma \mapsto  \sigma \mid_{R(f)}}
	\] 
	ist eine \emph{injektiver} Gruppenhomomorphismus.
\end{lemma}

\begin{proof}
	Aus Lemma II.4 folgt $\sigma(R(f)) = R(f) \forall \sigma \in \gal(E / K)$.
	Homomorphismus: $(\sigma \circ \eta) \mid_{R(f)} = \sigma \mid_{R(f)} \circ \eta \mid_{R(f)}$\\
	Injektivität: Sei $R(f) = \{\alpha_1,\ldots,\alpha_{n}\} $ dann folgt aus Algebra I, dass $E = k[\alpha_1,\ldots,\alpha_{n}$.
	Wobei $K[\alpha_1,\ldots,\alpha_{n}]$ das Bild der Evaluationsabbildung $\nstack{K[X_1,\ldots,X_{n}] \to E}{P \mapsto P(\alpha_1,\ldots,\alpha_{n})}$.
	Sei $\sigma \in \gal(E / K)$ mit $\sigma \mid_{R(f)} = \id_{R(f)}$. Das heißt $\sigma(\alpha_i) = \alpha_{i} \forall 1 \leq i \leq n$.
	Sei $\xi \in E$ und $P \in K[X_1,\ldots,X_{n}]$ so dass $P(\alpha_1,\ldots,\alpha_{n}] = \xi$.
	Also folgt: da $\sigma(x) = x \forall x \in K$ folgt 
	\[
		\sigma(P) = \sigma(P(\alpha_1,\ldots,\alpha_{n}) = P(\sigma(\alpha),\ldots,\sigma(\alpha_{n})) = P(\alpha_1,\ldots,\alpha_{n}) = P
	\]
	folgt $\sigma = \id_{E}$.
\end{proof}

\begin{proof}[Alternativer Beweis]
	\begin{align*}
		E = K(\alpha_1,\ldots,\alpha_{n}) = \left\{ \frac{P(\alpha_1,\ldots,\alpha_{n})}{Q(\alpha_1,\ldots,\alpha_{n}} \mid P,Q \in K[X_1,\ldots,X_{n} \text{ und } Q(\alpha_1,\ldots,\alpha_{n} \neq 0 \right\} 
	.\end{align*}
	Dann analoger Beweis zu oben.
\end{proof}

Sei $E / K$ eine Körpererweiterung, $\alpha \in E$. Dann ist $K[\alpha] := $ Bild des Evaluationshomomorphismus $\nstack{K[X] \to E]}{P \mapsto P(\alpha)}$ 
Da $E$ Körper ist $K[X]$ ein Integritätsbereich und $K(X)$ der Quotientenkörper von $K[\alpha]$.

Im allgemeinen ist $\abs{R(f)} \leq \deg(f)$. 
\begin{eg}
	$K = \F_{p}(t), f \in K[X], f = X^{p}-t$.
	Dann ist $f$ irreduzibel (Übung) und $\abs{R(f)} = 1$.
	Sei $K \subseteq E$ ein Zerfällungskörper von $f$ und $\alpha \in R(f): \alpha^{p} = t$.
	Da $E$ Charakteristik $p$ ist folgt $(X-\alpha)^{p} = X^{p} - \alpha^{p} = X^{p} - t^{p} = f$.
	Also $R(f) = \{\alpha\}$ und $\abs{\gal(E / K)} = 1$.
\end{eg}

\textbf{Ziel:} $f \in K[X]$ irreduzibles Polynom mit $\abs{R(f)} = \deg(f)$ dann ist  $\abs{\gal(E / K)} = [E:K]$.

\begin{definition}
	Ein Polynom $f \in K[X]$ hat keine mehrfachen Nullstellen falls in einem Zerfällungskörper $\abs{R(f)} = \deg(f)$.
\end{definition}

\begin{lemma}[Übung]
	Sei $f \in K[X]$ und $f' \in K[X]$ die (formelle) Ableitung von $f$.
	$f$ hat keine mehrfachen Nullstellen genau dann wenn $\gcd(f,f') = 1$.
\end{lemma}

\begin{remark}
	Gegeben $f,g \in K[X]$, der euklidische Algorithmus berechnet $\gcd(f,g)$.
\end{remark}

\begin{corollary}
	Sei $f \in K[X]$ irreduzibel und sei eine der folgenden Voraussetzungen erfüllt:
	\begin{enumerate}[(1)]
		\item $\charak(K) = 0$
		\item  Falls $\charak(K) > 0$ dann teilt $\charak(K)$ nicht $d = \deg(f)$.
	\end{enumerate}
	Dann hat $f$ keine mehrfachen Nullstellen.
\end{corollary}

\begin{proof}
	Sei $f(x) = a_{d} x^{d} + a_{d-1} x^{d-1} + \ldots + a_0$, $a_{d} \neq 0, \deg(f) = d$. 
	Dann ist $f'(x) = a_{d} d x^{d-1} + a_{d-1} (d-1) x^{d-2} + \ldots + a_1$.
	Unter der Voraussetzung des Lemmas folgt $d \neq 0$ und somit $a_{d} d \neq 0$, also $\deg(f') = d-1$.
	Falls $p \in K[X]$ mit $p$ dividiert $f$ und $f' \implies \deg(p) \leq d-1$ aber $f$ ist irreduzibel $\deg(f) = d$.
	Folgt $\deg(p') = 0$ d.h.  $p \in K$. Weiters folgt $\gcd(f,f') = 1$.
	Und somit mit dem Lemma folgt $f $ hat keine mehrfachen Nullstellen.
\end{proof}

\begin{definition}
	\begin{enumerate}[(1)]
		\item Ein irreduzibles Polynom ist \emph{separabel} falls es keine mehrfachen Nullstellen besitzt.
		\item Ein Polynom ist \emph{separabel} falls alle seiner irreduziblen Faktoren separabel sind. 
	\end{enumerate}
\end{definition}

\begin{eg}
	$X^{4} + 1 \in Q[X]$ ist irreduzibel; da $\charak(\Q) = 0$ folgt aus dem Lemma, dass $X^{4}+ 1$ separabel ist.
	Also ist auch $(X^{4} + 1)^{15}$ separabel.
\end{eg}

\begin{definition}[Wiederholung]
	Sei $E / K$ eine Körpererweiterung und $\alpha \in E$ : $\nstack{ \varphi_{\alpha}: K[X] \to E}{P \mapsto P(\alpha)}$ ist ein Ringhomomorphismus.
Sei $\ker(\varphi_{\alpha})$ sein Kern, dann ist $\ker(\varphi_{\alpha})$ ist ein Ideal in $K[X]$.
Zwei Möglichkeiten
\begin{enumerate}[(1)]
	\item $\ker(\varphi_{\alpha}) = (0)$ dann heißt $\alpha$ transzendent über $K$.
	\item $\ker(\varphi_{\alpha}) \neq (0)$ dann ist $\alpha$ algebraisch.
		Da $K[X]$ ein Hauptidealring ist gibt es genau ein unitäres Polynom $\irr(\alpha,K)$,
		das Minimalpolynom von $\alpha$ über $K$, das $\ker(\varphi_{\alpha})$ erzeugt: $\ker(\varphi_{\alpha}) = \irr(\alpha,K) \cdot K[X]$.
\end{enumerate}
\end{definition}

Aus der Tatsache, dass $\irr(\alpha,K)$ irreduzibel ist und $K[X]$ ein euklidischer Ring folgt $\sfrac{K[X]}{\ker(\varphi_{\alpha})}$ ist ein Körper und
\begin{lemma}
	$\varphi_{\alpha}$ induziert einen Körperisomorphismus $\overline{\varphi_{\alpha}}: \sfrac{K[X]}{\ker(\varphi_{\alpha})} \overset{\sim}{\to} K(\alpha) (= K[\alpha])$
\end{lemma}

Sei $\varphi: K \to K'$ ein Körperisomorphismus; dieser induziert einen Ring Isomorphismus $\varphi_{*}: K[X] \to K'[X]$ mit
\[
	\varphi_{*}(a_{n}X^{n} + \ldots + a_0) := \varphi_{*}(a_{n}) X^{n} + \ldots + \varphi_{*}(a_{0})
.\] 
Da $\varphi_{*}$ ein Ringisomorphismus ist folgt $p \in K[X]$ ist genau dann irreduzibel, falls $\varphi_{*}(p)$ irreduzibel ist.
Bemerke: $\deg(\varphi_{*}(p)) = \deg(p)$.

\begin{lemma}
	Sei $p \in K[X]$ irreduzibel, $p_{*} = \varphi_{*}(p) \in K'[X]$; seien $E \supseteq K$ und $E' \supseteq K'$ mit $R(p) \subseteq E$ und $R(p_{*}) \subseteq E'$.
	Dann gilt: $\forall \alpha \in R(p) \forall \alpha' \in R(p_{*})$ gibt es einen Isomorphismus $\widehat{\varphi}: K(\alpha) \to K'(\alpha')$ der $\varphi$ erweitert
	und $\widehat{\varphi}(\alpha) = \alpha'$
	\[
		\begin{tikzcd}
			K \arrow[r, "\varphi"] \arrow[d, hook]    & K' \arrow[d, hook] \\
			K(\alpha) \arrow[r, "\widehat{\varphi}"'] & K'(\alpha')       
		\end{tikzcd}
	.\]
\end{lemma}

\begin{proof}
	%TODO Diagramm
	Betrachte das linke Diagramm.
	Da $p$ irreduzibel und $p(\alpha) = 0$ folgt $\ker(\varphi_{\alpha}) = p \cdot K[X]$.
	Gleiches gilt für $p_{*}$ : $\ker(\varphi_{\alpha'}) = p_{*} K'[X]$.
	Da $\varphi_{*}(p) = p_{*}$ folgt $\varphi_{*}(\ker(\varphi_{\alpha})) = \ker(\varphi_{\alpha'})$.
	Daraus folgt, dass $\varphi_{*}$ einen Ringisomorphismus $\overline{\varphi_{*}}: \sfrac{K[X]}{\ker(\varphi_{\alpha})} \overset{\sim}{\to} \sfrac{K'[X]}{\ker(\varphi_{\alpha'})}$ 
	induziert.\\
	Betrachte das rechte Diagramm.
	Die gesuchte Erweiterung $\widehat{\varphi} = \overline{\varphi_{\alpha'}} \circ \overline{\varphi_{*}} \circ \overline{\varphi_{\alpha}}^{-1}$
	\[
	\begin{tikzcd}
		E \arrow[r, "\varphi_*"] \arrow[d, "\varphi_\alpha"] & E_* \arrow[d, "\varphi_{\alpha'}"] &  & \sfrac{K[X]}{\ker(\varphi_{\alpha})}
\arrow[d, "\overline{\varphi_{\alpha}}"] \arrow[r,"\overline{\varphi_{*}}"] & \sfrac{K'[X]}{\ker(\varphi_{\alpha'})} \arrow[d, "\overline{\varphi_{\alpha'}}"] \\
K(\alpha) \subset E                                  & K'(\alpha')                        &  & K(\alpha) \arrow[r, "\widehat{\varphi}"]   & K'(\alpha') 
\end{tikzcd}
	.\] 
\end{proof}

\begin{theorem}
	Sei $\varphi: K \to K'$ ein Isomorphismus, $f \in K[X]$, $f_{*} = \varphi_{*}(f)$.
	Sei $E / K$ ein Zerfällungskörper von $f$ und $E_{*}$ ein Zerfällungskörper von $f_{*}$.
	\begin{enumerate}[(1)]
		\item Annahme $f$ ist separabel. Dann gibt es genau $[E:K]$ Isomorphismen 
			\[
				\begin{tikzcd}
					E \arrow[r, "\Phi"]                    & E_*                \\
					K \arrow[r, "\varphi"] \arrow[u, hook] & K' \arrow[u, hook]
				\end{tikzcd}
			\]
			die $\varphi$ erweitern, d.h. $\Phi \mid_{K} = \varphi$
		\item Sei $E / K$ Zerfällungskörper eines separablen Polynoms dann ist $\abs{\gal(E / K)} = [E:K]$ 
	\end{enumerate}
\end{theorem}

\begin{proof}
	\begin{enumerate}
		\item \begin{enumerate}
			\item Wir hatten mittels $\varphi: K \to K'$ einen Ringhomomorphismus $\varphi_{K}: K[X] \to K'[X]$, nämlich
				$h = a_{n} X^{n} + \ldots + a_0 \in K[X]$.
				Dann ist $\varphi_{*}(h) = \varphi(a_{n}) X^{n} + \ldots + \varphi(a_0)$. Bemerke $\varphi_{*}(h_1 + h_2) = \varphi_{*}(h_1) + \varphi_{*}(h_2)$ und
				$\varphi_{*}(h_1 h_2) = \varphi_{*}(h_1)\varphi_{*}(h_2), \varphi_{*}(\mathbbm{1}) = \mathbbm{1}$ und $\deg(\varphi_{*}(h)) = \deg(h)$.
				Aus diesen Eigenschaften folgt $\varphi_{*}(\gcd(h_1, h_2)) = \gcd(\varphi_{*}(h_1), \varphi_{*}(h_2))$.

				$f(X) = a_{n} X^{n} + \ldots + a_0$, $f'(X) = a_{n} n X^{n-1} + \ldots + a_1$.
				Allgemeine Eigenschaft $\varphi: K \to K'$: $\varphi(m \xi) = m \varphi(\xi) \forall m \in \Z$.
				\begin{align*}
					\varphi_{*}(f) &= \varphi(a_{n}) X^{n} + \ldots + \varphi(a_0)\\
					\varphi_{*}(f) &= \varphi(a_{n} n) X^{n-1} + \ldots + \varphi(a_1)\\
								   &= \varphi(a_{n}) n X^{n-1} + \ldots + \varphi(a_1) = \varphi_{*}(f)'
				.\end{align*}
				Also folgt $\varphi_{*}(\gcd(f,f')) = \gcd(\varphi_{*}(f), \varphi_{*}(f)')$.
				Da $f$ separabel ist folgt $\gcd(f,f') = \mathbbm{1} \implies \gcd(\varphi_{*}(f), \varphi_{*}(f)') = \mathbbm{1}$.
				Und also $f_{*} = \varphi_{*}(f)$ separabel ist.
			\item $[E : K] = 1$ dann $R(f) \subseteq E = K$, $f$ zerfällt also in lineare Faktoren woraus folgt $f_{*} = \varphi_{*}(f)$ in $K'[X]$ auch in
				lineare Faktoren zerfällt $\implies E_{*} = K'$.
			\[
				\begin{tikzcd}
					E \arrow[r, "\Phi"']                               & E_*                               \\
					K \arrow[r, "\varphi"] \arrow[u, equal] & K' \arrow[u,equal]
				\end{tikzcd}
				.\] 	
			\item $[E : K] > 1:$ Dann gibt es $p \in K[X]$ irreduzibel mit $\deg(p) > 1$ und $p$ dividiert $f$. Notation $d := \deg(f) > 1$.
				Sei $f = p g$ wobei $g \in K[X]$. Dann folgt $f_{*} = \varphi_{*}(f) = \underbrace{\varphi_{*}(p)}_{P_{*}}\underbrace{\varphi_{*}(g)}_{g_{*}} = f_{*} g_{*}$.
				Aus $f$ separabel folgt $f_{*}$ separabel.

				Aus $p$ irreduzibel folgt $p_{*}$ irreduzibel: Also ist $p_{*}$ irreduzibel und separabel. 
				Da $\deg(p_{*}) = \deg(p) = d$ hat $f_{*}$ in $E_{*}$ genau $d$ Nullstellen $\alpha_{1}^{*},\ldots, \alpha_{d}^{*}$ alle in $E^{*}$ enthalten.

				Sei $\alpha \in R(p) \subseteq E$. Aus Lemma 2.15 folgt, dass es für jedes $\alpha_{i}^{*}$ einen Isomorphismus
				\[
					\widehat{\varphi_{i}}: K(\alpha) \to K'(\alpha_{i}^{*}) \subseteq E_{*} \qq{mit} \widehat{\varphi_{i}}(\alpha) = \alpha_{i}^{*}
				\] 
				und erweitert $\varphi: K \to K'$. 

				Sei $\alpha_{*} = \alpha_{i}^{*}, \widehat{\varphi} = \widehat{\varphi_{i}}$.
				\[
					\widehat{\varphi}: K(\alpha) \subseteq E \to  K'(\alpha_{*}) \subseteq E_{*}
				.\] 
				\begin{remark}
					$f \in K(\alpha)[X]$ und $E$ ist Zerfällungskörper von $f \in K(\alpha)[X]$.
					$f_{*} \in K'(\alpha)[X]$ und $E_{*}$ ist Zerfällungskörper von $f_{*} \in K'(\alpha)[X]$.
					Es ist  $(\widehat{\varphi})_{*}: K(\alpha)[X] \to K'(\alpha_{*})[X]$ ein Isomorphismus.
					Da $\widehat{\varphi} \mid_{K} = \varphi$ folgt $(\widehat{\varphi})_{*}(f) = f_{*}$.
					$f$ und $f_{*}$ sind wieder separabel.
				\end{remark}

				Da nun $[E:K(\alpha)] = \frac{[E:K]}{[K(\alpha) : K]} = \frac{[E:K]}{d} < [E:K]$.
				Nach Induktionshypothese gibt es $[E:K(\alpha)]$ Erweiterungen von $\widehat{\varphi}: K(\alpha) \to K'(\alpha_{*})$ auf $E \to E_{*}$.
				$\forall 1 \leq i \leq d$ hat $\widehat{\varphi_{i}}: K(\alpha) \to K'(\alpha_{i}^{*})$ ist eine $[E : K(\alpha)]$-Erweiterung auf $E \to E_{*}$.
				Mit dieser Konstruktion erhalten wir also $d [E : K(\alpha)] = [K(\alpha) : K] [E : K(\alpha)] = [E : K]$ Erweiterungen von $\varphi: K \to K'$.
		\end{enumerate}
		\begin{remark}[Einschub für weiter oben]
		$h$ hat keine mehrfachen Nullstellen genau dann wenn $\gcd(h,h') = 1$: Zu zeigen $f$ separabel $\implies f_{*}$ separabel.
		Sei $h$ ein irreduzibler Faktor von $f$ : $f = h \cdot u$, dann ist $f_{*} = \varphi_{*}(h) \varphi_{*}(u)$ : $\varphi_{*}(h)$ irreduzibel.
		$h$ hat keine mehrfachen Nullstellen $\implies \gcd(h,h') = 1 \implies \gcd(\varphi_{*}(h),\varphi_{*}(h)') = 1 \implies \varphi_{*}(h)$ keine mehrfachen Nullstellen.
		\end{remark}
	\item Sei $E / K$ Zerfällungskörper eines separablen Polynoms: $\abs{\gal(E / K)} = [E:K]$.
		$\gal(E/K) = \{\varphi: E \to E \text{ Isomorphismen die $\id:K \to K$}\}$ erweitern.
	\end{enumerate}
\end{proof}

\begin{corollary}
	Sei $E / K$ ein Zerfällungskörper eines separablen Polynom $f \in K[X]$ von $\deg(f) = n$. 
	Falls $f$ irreduzibel folgt: $n$ dividiert $\abs{\gal(E/K)}$.
\end{corollary}

\begin{proof}
	Sei $\alpha \in R(f) \subseteq E$. Dann ist $K(\alpha) \subseteq E$ und da $f$ irreduzibel $[K(\alpha) : K] = n = \deg(f)$.
	Aus Satz 2.17: $\abs{\gal(E/ K)} = [E:K] = [E: K(\alpha)] [K(\alpha) : K] = [E : K(\alpha)] \cdot n$.
\end{proof}

\begin{theorem}
	Sei $p$ eine Primzahl, $n \in \N, n \geq 1$. Dann ist $\gal(\F_{p^{n}} / \F_{p}) \cong \sfrac{\Z}{n \Z}$ 
	ein erzeugendes Element ist gegeben durch $Fr:\nstack{\F_{p^{n}} \to \F_{p^{n}}}{x \mapsto x^{p}}$
\end{theorem}

\begin{proof}[Beweisidee]
	\begin{enumerate}
		\item $\F_{p^{n}}: \abs{\F_{p^{n}}^{\times}} = p^{n}-1$ also ist $\F_{p^{n}}^{\times}$ genau die Menge der Nullstellen des Polynoms
			\[
				x^{p^{n}-1} -1 \in \F_{p}[X]
			.\]
			Dieses Polynom hat keine mehrfachen Nullstellen $\implies$ separabel und mit Satz 2.17 folgt:
			\[
				\abs{\gal(\F_{p^{n}} / \F_{p})} = [F_{p^{n}} : F_{p}] = n
			.\]
			Nun $Fr \in \gal(\F_{p^{n}} / \F_{p})$. Für $k \geq 1, k \in \N$. $Fr^{k}(\xi) = (\xi)^{p^{k}}$.
			Sei $m$ die Ordnung von $Fr$ in $\gal(\F_{p^{n}} / \F_{p}): Fr^{m} = \id_{\F_{p^{n}}}: \forall \xi \in F_{p^{n}} \xi^{p^{m}} = \xi$.
			Also ist $\F_{p^{n}}$ in der Menge der Nullstellen eines Polynoms von Grad $p^{m}$ enthalten $\implies p^{n} \leq p^{m}$.
			Folgt $n \geq m \implies n = m$. Folgt dass $\gal(\F_{p^{n}} / \F_{p}) = \{\id, Fr, \ldots, Fr^{n}\} $
	\end{enumerate}
\end{proof}

\begin{theorem}
	Sei $p$ eine Primzahl und $f \in \Q[X]$ mit $\deg(f) = p$ und Zerfällungskörper $E$. Annahme:
	\begin{enumerate}
		\item $f$ ist irreduzibel
		\item $f$ hat genau $p-2$ reelle Nullstellen.
	\end{enumerate}
	Dann ist $\gal(E / \Q) \cong S_{p}$.
\end{theorem}

\begin{corollary}
	$p$ dividiert die Ordnung von $\gal(E / \Q)$.
\end{corollary}

\begin{lemma}[Cauchy]
	Sei $G$ eine endliche Gruppe und $p$ eine Primzahl die die Ordnung von $G$ dividiert.
	Dann enthält $G$ eine Element der Ordnung $p$.
\end{lemma}

\begin{proof}
	Sei $\Gamma_{p} = \{(g_1,\ldots, g_{p}) \in G^{p}: g_1 \cdot \ldots \cdot g_{p} = e\} \subseteq G^{p}$.
	Die Symmetrie Gruppe $S_{p}$ wirkt auf $G^{p}: \eta \in S_{p}$. $\eta_{*}(g_1,\ldots,g_{p}) = (g_{\eta(1)},\ldots,g_{\eta(p)})$.
	Sei $\sigma = (1,2,\ldots,p)$ der $p$-Zykel.
	Sei $(g_1,\ldots,g_{p}) \in \Gamma_{p}: g_1 \cdot \ldots \cdot g_{p} = e$.
	\[
		\underbrace{g_1^{-1} (g_1 \cdot \ldots\cdot g_{p}) g_{1}}_{g_2 \cdot \ldots\cdot g_{p} g_1 = e} = g_1^{-1} e g_1 = e \qq{und} g_{\sigma(1)} \cdot \ldots\cdot g_{\sigma(p)} = e
	.\] 
	Folglich lässt die durch $\sigma$ erzeugte zyklische Untergruppe von $S_{p}$ 
	\[
	C_{p} = \{\id, \sigma, ,\ldots, \sigma^{p-1}\} \quad \Gamma_{p} \text{ invariant}
	.\]
	Also ist $\Gamma_{p} = $ disjunkte Vereinigung der $C_{p}$-Bahnen in $\Gamma_{p}$.
	Da $p$ Primzahl hat so eine Bahn entweder Kardinalität $1 $ oder $p$.
	Sei $I = $ Menge der Bahnen mit Card $1$, $J$ = Menge der Bahnen mit Card $p$.
	Dann ist  $\Gamma_{p} = \bigsqcup_{i \in I} O_{i} \sqcup \bigsqcup_{j \in J} O_{j}$ und $O_1 = \{(e,\ldots,e)\}$.
	Es ist dann $\abs{G}^{p-1} = \abs{\Gamma_{p}} = \abs{I} \cdot 1 + \abs{J} \cdot p$.
	Nun $p \mid \abs{G}^{p-1}$. Folgt $p \mid \abs{I} \implies \abs{I} \geq_2$.
\end{proof}

\begin{proof}[Beweis vom Theorem]
	$f \in \Q[X]$ und $\Q \subseteq \C$. Da $\C$ algebraisch abgeschlossen ist können wir einen Zerfällungskörper $E$ von $f$ wählen mi $\Q \subseteq E \subseteq \C$.
	Sei $R(f) = \{\alpha_1,\ldots,\alpha_{p}\}$ die Nullstellen von $f$ so geordnet, dass $\{\alpha_3,\ldots,\alpha_{p}\} \subseteq \R$.
	Da die Koeffizienten von $f$ reell sind folgt:$\alpha_2 = \epsilon(\alpha_1), \alpha_1 = \epsilon(\alpha_2)$.
	Hier ist $ \epsilon: \C \to \C$ die komplexe Konjugation: 
	Mittels Lemma 2.7 können wir $\gal(E / \Q) \hookrightarrow S_{R(f)} = S_{p}$ mit einer Untergruppe von $S_{p}$ identifizieren.

	Aus $\epsilon(\alpha_1) = \alpha_2, \epsilon(\alpha_2) = \alpha_1, me(\alpha_{i}) = \alpha_{i} 3 \leq i \leq p $ folgt
	Da $E = \Q(\alpha_1,\ldots,\alpha_{p})$, dass $\epsilon(E) = E$ und $\epsilon \mid_{E} \in \gal(E / \Q)$ entspricht der Transposition $(12)$.
	Aus Korollar 2.18 folgt $p$ dividiert $\abs{\gal(E / \Q)}$ aus Cauchy folgt dass $\gal(E / \Q)$ ein Element der Ordnung $p$ enthält, aber ein Element
	der Ordnung  $p$ in $S_{p}$ ist ein $p$-Zykel. Und da $p$ eine Primzahl ist, folgt dass eine Transposition und ein $p$-Zykel $S_{p}$ erzeugen.
\end{proof}

\begin{corollary}
	Die Galois Gruppe von $X^{5} - 4x + 2 \in \Q[X]$ ist $\cong S_{5}$.
\end{corollary}

\begin{proof}
	\begin{enumerate}[(1)]
		\item Eisensteinkriterium mit $q = 2 \implies X^{5} - 4x +2$ irreduzibel ist.
		\item $f$ hat genau $3 = 5-2$ reelle Nullstellen. Betrachte $f : \R \to \R, f(x) = x^{5} - 4x + 2$.
			$f'(x) = 5x^{4} - 4$ hat zwei Nullstellen: $- \sqrt[4]{\frac{4}{5}}, \sqrt[4]{\frac{4}{5}} $.
			Falls $\abs{x} >  \sqrt[4]{\frac{4}{5}}$ dann ist $f'(x) > 0$.
			Falls $\abs{x} < -\sqrt[4]{\frac{4}{5}}$ dann ist $f'(x) < 0$.
			Außerdem ist $f(\sqrt[4]{\frac{4}{5})}) < 0$ und  $f(-\sqrt[4]{\frac{4}{5}}) < 0 $.
			%TODO Image plot der Fkt mit Extrema und nullstellen hervorgehoben.
			%\begin{tikzpicture}
			%\draw[scale=0.5, domain=-3:3, smooth, blue] plot function {x**2}2};
			%\end{tikzpicture}
	\end{enumerate}
	Die Aussage folgt nun aus vorherigem Theorem
\end{proof}

\subsection{Zusammenhang zwischen Irreduzibilität und Transitivität der Galois Gruppe}
\begin{corollary}
	Sei $f \in K[X]$ und $E$ ein Zerfällungskörper von $f$.
	Dann gilt: $f$ irreduzibel $\Leftrightarrow$ $\gal(E / K)$ wirkt transitiv auf $R(f)$.
\end{corollary}

Sei $G \times X \to X$ eine Gruppenwirkung. Die Wirkung ist \emph{transitiv} falls $\forall x,y \in X \exists g \in G : g(x) = y$.

\begin{claim}
	$\implies$ : Gilt auch ohne Voraussetzung an die Nullstellen von $f$.
\end{claim}

\begin{proof}
	$\implies$ : Sei $f$ irreduzibel. Seien $\alpha,\beta \in R(f)$.
	Aus Lemma 2.15 folgt: Es gibt $\varphi: K(\alpha) \to K(\beta)$ ein Isomorphismus mit $\varphi \mid_{K} = \id_{K}$ und $\varphi(\alpha) = \beta$.
	Nun ist $\varphi_{*}(f) = f$. Dann folgt aus Proposition 2.16, dass sich $\varphi$ zu einem Isomorphismus $\Phi: E \to E$ erweitern lässt ($\Phi \mid_{K(\alpha)} = \varphi$ ).
	Also ist $\Phi \in \gal(E/K)$ mit $\Phi(\alpha) = \beta$.

	$\impliedby$: Sei $f = p\cdot q$ mit $p,q \in K[X]$. Dann folgt $R(p) \subseteq R(f), R(q) \subseteq R(f)$.
	Da $p,q$ Koeffizienten in $K$ hat folgt, dass $\gal(E /K)$ die Teilmengen $R(f)$ und $R(q)$ invariant lässt.
	Da $f$ ohne mehrfachen Nullstellen ist, folgt $R(p) \cap R(q) = \emptyset$.
	Aus der Transitivität von $\gal(E / K)$ auf $R(f)$ folgt $R(p) = \emptyset$ oder $R(q) = \emptyset$.
	$\implies p$ ist eine Konstant und somit ist $f$ irreduzibel.
\end{proof}

\begin{eg}
	$f$ irreduzibel $\implies \gal(E / K)$ transitiv auf $R(p)$. Aber  $g = f^2$ dann ist $E$ auch Zerfällungskörper von $g$ und $\gal(E / K)$ wirkt transitiv auf $R(f) = R(g)$.
\end{eg}

\begin{definition}
	Eine Erweiterung $E / K$ heißt normal falls sie Zerfällungskörper eines Polynoms $f \in K[X]$ ist.
\end{definition}

\begin{claim}
	Seien $K \subseteq B \subseteq E$ Körpererweiterungen. Falls $E / K$ normal ist so folgt, dass $E / B$ auch  normal ist.
\end{claim}

\begin{theorem}
	Seien $K \subseteq B \subseteq E$ (endliche) Erweiterungen mit der der Eigenschaft, dass sowohl $E / K$ wie $B / K$ normale Erweiterungen sind.
	Dann folgt $\forall \sigma \in \gal(E / K)$ ist $\sigma(B) = B$.

	Und der Homomorphismus $\nstack{\gal(E / K) \to \gal(B / K)}{\sigma \mapsto \sigma \mid_{B}}$ ist surjektiv mit Kern $\gal(E / B)$.
\end{theorem}

\begin{proof}
	Da $B / K$ normal, sei $f \in K[X]$ mit $B$ als Zerfällungskörper.
	Aus Lemma 2.4 folgt
	\[
		\sigma(R(f)) = R(f) \forall  \sigma \in \gal(E / K)
	.\] 
	Aus $B = K(R(f))$ folgt $\sigma(B) = B$. Also ist $\sigma \mapsto \sigma \mid_{B}$ ein Wohldefinierter Homomorphismus von $\gal(E / K) \to \gal(B / K)$.
	Der Kern ist offensichtlich $\gal(E / B)$.

	Surjektivität: Da $E / K$ normal, sei $g \in K[X]$ mit Zerfällungskörper $E$.
	Sei $\sigma \in \gal(B / K): $ 
	\[
		\begin{tikzcd}
			\sigma: B \arrow[r]                          & B                 \\
        	K \arrow[r, "\id_K"] \arrow[u, hook] & K \arrow[u, hook]
		\end{tikzcd}
	\]
	Daraus folgt $\sigma_{*}(g) = g$. Aus Proposition 2.16 folgt, dass $\sigma$ sich auf $\Sigma: E \to E$ erweitert.
	Dann folgt $\Sigma \mid_{B} = \sigma$ und $\Sigma \in \gal(E / K)$.
\end{proof}

\begin{theorem}
	Eine endliche Erweiterung $E / K$ ist genau dann normal falls jedes irreduzible Polynom in $K[X]$,
	dass eine Nullstelle in $E$ besitzt, in linear Faktoren in $E$ zerfällt.
\end{theorem}

\begin{proof}
	Siehe Serie 4
\end{proof}

































