%! TEX root=../algebra.tex
\graphicspath{{Images/}}

\chapter{Galois Korrespondenz}

\begin{definition}
	Sei $E$ ein Körper und $H \subseteq \aut(E)$ dann ist $E^{H} := \{x \in E \mid \sigma(x) = x \forall \sigma \in H\} $ ist ein Unterkörper von $E$.
	Dann ist $E^{H}$ der \emph{Fixkörper} von $H$.
\end{definition}

\begin{remark}
	Die Korrespondenz $H \mapsto E^{H}$ hat folgende Monotonie Eigenschaft: $H_1 \subseteq H_2 \implies E^{H_2} \subseteq E^{H_1}$.
\end{remark}

\begin{eg}
	$E / k$ Erweiterung $k \subseteq E^{\gal(E / k)}$; das Beispeil $k = \F_{p}(t)$ und $f(x) = X^{p}-t$, $E$ Zerfällungskörper von $f$, $\gal(E / k) = (e)$.
	In diesem Fall $k \subsetneq E^{\gal(E / k)} = E$.
\end{eg}

\textbf{Ziel:} Bestimmung des Grades $[E : E^{H}]$ wobei $H < \aut(E)$ eine \emph{endliche Untergruppe} bezeichnet.

\begin{definition}
	Sei $G$ eine Gruppe, $E$ ein Körper.
	Ein \emph{Charakter von $G$ in $E$} ist ein Gruppenhomomorphismus $G \to E^{\times}$.
	Wobei $E^{\times}$ die Multiplikative Gruppe $E \setminus \{0\}$ ist.

	Die Menge der Charaktere von $G$ in $E$ wird mit $\hom(G,E^{\times})$ bezeichnet.
	Man kann $H(G,E^{\times})$ als Teilmenge des Vektorraums $F(G,E)$ aller $E$-wertigen Funktionen auf $G$.
\end{definition}

\begin{proposition}[Dedekind]
	$\hom(G,E^{\times}) \subseteq F(G,E)$ ist linear unabhängig.
\end{proposition}

\begin{proof}
	Sei $n \geq 1$ minimal so dass es $n$ Charaktere $\sigma_1,\ldots,\sigma_{n}$ gibt die in $F(G,E)$ linear abhängig sind.
	Also gibt es $c_1 \neq  0,\ldots, c_{n} \neq 0$ in $E$ mit $c_1 \sigma_1(x) + \ldots + c_{n} \sigma_{n}(x) = 0 \forall x \in G$.

	Da $\sigma_1 \neq \sigma_{n}$, sei $y \in G$ mit $\sigma_1(y) \neq \sigma_{n}(y)$.
	Wir ersetzen $x$ durch $xy$ und erhalten
	\[
		c_1 \sigma_1(x) \sigma_1(y) + \ldots + c_{n} \sigma_{n}(x) \sigma_{n}(y) = 0 \implies c_1 \sigma_1(x) \frac{\sigma_1(y)}{\sigma_{n}(y)} + 
		\ldots + c_{n-1} \sigma(x) \frac{\sigma_{n-1}(y)}{\sigma_{n}(y)} + c_{n} \sigma_{n}(x) = 0 \forall x \in G
	.\] 
	verwenden wir nun die erste Gleichung erhalten wir
	\[
		c_1 \sigma_1(x) \left( \frac{\sigma_1(y)}{\sigma_{n}(y)} - 1 \right) + \ldots + c_{n-1} \sigma_{n-1}(x) \left( \frac{\sigma_{n-1}(y)}{\sigma_{n}(y)} - 1 \right) 
	.\]
	Nicht triviale lineare Relation; $\frac{\sigma_1(y)}{\sigma_{n}(y)}- 1 \neq 0$ \contra
\end{proof}

Benutze diesen Satz um eine untere Schranke von $[E: E^{H}]$ zu bestimmen falls $H \subseteq \aut(E)$ eine \emph{endliche Teilmenge} besitzt.

\begin{lemma}[Sublemma]
	Sei $E$ ein Körper, $S$ eine Menge und $\{\sigma_1,\ldots,\sigma_{n}\} \subseteq G(S,E)$ linear unabhängig.
	Dann gibt es $s_1,\ldots,s_{n} \in S$ mit 
	\[
		\begin{pmatrix} \sigma_1(s_1) \\ \vdots \\ \sigma_{n}(s_1) \end{pmatrix} ,\ldots, \begin{pmatrix} \sigma_1(s_{n}) \\ \vdots \\ \sigma_{n}(s_{n}) \end{pmatrix} 
	\] 
	in $E^{n}$ linear unabhängig sind.
\end{lemma}

\begin{lemma}
	Sei $H = \{\sigma_1,\ldots,\sigma_{n}\} \subseteq \aut(E)$, Teilmenge mit $n$ Elementen.
	Dann gilt $[E : E^{H}] \geq n$.
\end{lemma}

\begin{claim}
	Falls $\left< H \right>	$ die von $H$ erzeugte Untergruppe von $\aut(E)$ bezeichnet so ist $E^{H} = E^{\left< H \right>}$.
\end{claim}

\begin{proof}
	Wir betrachten $\sigma_1 \mid_{E^{\times}},\ldots, \sigma_{n} \mid_{E^{\times}}$ als Elemente von $\hom(E^{\times}, E^{\times})$.
	Also sind sie in $F(E^{\times }, E)$ linear unabhängig (Dedekind) und es gibt (Sublemma) $y_1,\ldots,y_{n} $ in $E^{x} $ mit
	\[
		\begin{pmatrix} \sigma_1(y_1) \\ \vdots \\ \sigma_{n}(y_1) \end{pmatrix} ,\ldots, \begin{pmatrix} \sigma_1(y_{n}) \\ \vdots \\ \sigma_{n}(y_{n}) \end{pmatrix} 
	\] 
	linear unabhängig in $E^{n}$.
	\begin{claim}
		$y_1,\ldots,y_{n}$ sind linear unabhängig über  $E^{H}$.
	\end{claim}
	Sei $c_1,\ldots, c_{n} $ in $E^{H}$ mit $c_1 y_1 + \ldots + c_{n} y_{n} = 0$. Für alle $1 \leq j \leq n$ folgt:
	\[
		\underbrace{\sigma_{j}(c_1 y_1 + \ldots + c_{n} y_{n})}_{c_1 \sigma_{j}(y_1) + \ldots + c_{n} \sigma_{j}(y_{n}) = 0} = 0 
	\] 
	d.h.
	\[
	\begin{pmatrix} 
		\sigma_{1}(y_1) &\ldots &\sigma_{1}(y_{n})\\
		\vdots &&\vdots\\
		\sigma_{n}(y_1) &\ldots &\sigma_{n}(y_{n})
	\end{pmatrix}
	\begin{pmatrix} 
		c_1\\ \vdots\\ c_{n}
	\end{pmatrix} = \begin{pmatrix} 
		0\\ \vdots\\ 0
	\end{pmatrix} 
	.\] 
	Diese Matrix hat Rang $n \implies c_1 = \ldots = c_{n} = 0$.
\end{proof}

\begin{proposition}
	Sei $G < \aut(E)$ eine endliche Untergruppe. Dann gilt $[E : E^{G}] = \abs{G}$.
\end{proposition}

\begin{proof}
	Sei $G = \{\sigma_1,\ldots,\sigma_{n}\}$ also $n = \abs{G}$. Aus Lemma folgt $[E : E^{G}] \geq n$.

	\textbf{Annahme:} $[E: E^{G}] > n$. Seien $b_1,\ldots, b_{m} \in E$ linear unabhängig über $E^{G}$ mit $m > n$.

	Wir definieren $T: E^{m} \to  E^{n}$
	\[
	\begin{pmatrix} 
		x_1 \\ \vdots \\ x_{m}
	\end{pmatrix} \mapsto \begin{pmatrix} 
	\sigma_1(b_1) &\ldots &\sigma_{1}(b_{m})\\
	\sigma_2(b_1) &\ldots &\sigma_{2}(b_{m})\\
	\vdots & &\vdots\\
	\sigma_n(b_1) &\ldots &\sigma_{n}(b_{m})
	\end{pmatrix} \begin{pmatrix} 
		x_1 \\ \vdots\\ x_{m}
	\end{pmatrix} 
	\] 
	lineare Abbildung. Dann gilt:
	\begin{enumerate}
		\item $\ker(T) \neq 0$ (klar da $m > n$ )
		\item  \[
		\begin{pmatrix} 
			x_1\\ \vdots\\ x_{m}
		\end{pmatrix} \in \ker(T) \implies \begin{pmatrix} 
		\sigma(x_1)\\ \vdots\\ \sigma(x_{m})
	\end{pmatrix} \in \ker(T) \forall \sigma \in G
		.\] 
	\end{enumerate}
	Wir zeigen $(2)$:  $(x_1,\ldots,x_{m})^{T} \in \ker(T) \Leftrightarrow \sum_{j=1}^{m} \sigma_{i}(b_{j})x_{j} = 0 \forall 1 \leq i \leq n :\implies \sum_{j=1}^{m} \sigma \sigma_{i}
	(b_{j}) \sigma(x_{j}) = 0$. Gegeben $\sigma \in G$. Sei $s \in S_{n}, \sigma \sigma_{i} = \sigma_{s(i)}$ :
	\[
		\sum_{j=1}^{m} \sigma_{s(i)}(b_{j}) \sigma(x_{j}) = 0 \forall 1 \leq i \leq n \implies \begin{pmatrix} 
			\sigma(x_1)\\ \vdots\\ \sigma(x_{n})
		\end{pmatrix} \in \ker(T)
	.\] 
	Sei $r := \min \{k \in \N : k \geq 1, v \in \ker(T) \setminus \{0\} \text{ besitzt $k$ Koordinaten} \neq 0\} $.
	Sei $(x_1,\ldots,x_{m})^{T} \in \ker(T)$ mit $r$ Koordinaten $\neq 0$. O.B.d.A sei $x_1 \neq 0$.
	Dann folgt
	\[
		\frac{1}{\sigma(x_1)} \begin{pmatrix} 
			\sigma(x_1) \\ \vdots \\ \sigma(x_{m})
		\end{pmatrix} - \frac{1}{x_1} \begin{pmatrix} 
			x_1 \\ \vdots \\ x_{m}
		\end{pmatrix} \in \ker(T) = \begin{pmatrix} 
		0\\ \frac{\sigma(x_2)}{\sigma(x_1)} - \frac{x_2}{x_1}\\ \vdots \\ \frac{\sigma(x_{m})}{\sigma(x_1)} - \frac{x_{m}}{x_1}
		\end{pmatrix} = \begin{pmatrix} 
			0 \\ 0 \\ \vdots \\ 0
		\end{pmatrix}  
	\]
	aus Definition von $r$. Es ist $\sigma(\frac{x_{i}}{x_{i}} = \frac{x_{i}}{x_{i}} 1 \leq i \leq m$ d.h.
	\[
	\begin{pmatrix} 
		1\\ \frac{x_2}{x_1}\\ \vdots \\ \frac{x_{m}}{x_1}
	\end{pmatrix} = \frac{1}{x_1}\begin{pmatrix} 
		x_1\\ x_2\\ \vdots \\ x_{m}
	\end{pmatrix} \in \ker(T) \cap (E^{G})^{m}
	.\]
	Das heißt aber $\sum_{j=1}^{m} \sigma_{i}(b_{j}) \frac{x_{j}}{x_{1}} = 0 \forall \sigma_1,\ldots,\sigma_{n}$. Aber, da $\id \in G < \aut(E)$ ist
	$\sum_{j=1}^{m} b_{j} \frac{x_{j}}{x_{i}} = 0$ d.h. $b_1 + b_2 \frac{x_2}{x_1} + \ldots + b_{m} \frac{x_{m}}{x_{1}} = 0$ 
	ist eine nicht triviale lineare Relation über $E^{G}$. \contra 
\end{proof}

\begin{corollary}
	Seien $G,H$ endliche Untergruppen von $\aut(E)$. Dann gilt $E^{G} \subseteq E^{H} \Leftrightarrow H < G$.
\end{corollary}

\begin{proof}
	$\impliedby:$ trivial.\\
	$\implies:$ Wir nehmen an $E^{G} \subseteq E^{H}$. Falls $H \neq G$ gibt es $\sigma \in H$ mit $\sigma \not\in G$.
	Da $\sigma \in H$ fixiert $\sigma$ jedes Element von $E^{H}$ und deswegen auch von $E^{G}$ also $ E^{G} = E^{G \cup \{\sigma\} }$.

	$\abs{G} = [E : E^{G}] = [E: E^{G \cup \{\sigma\}}] \geq \abs{G} + 1$. \contra
\end{proof}

\begin{corollary}
	Seien $G,H$ endliche Untergruppen von $\aut(E)$. Dann ist $E^{G} = E^{H} \Leftrightarrow H = G$.
\end{corollary}

\begin{definition}[Wiederholung]
	\begin{itemize}[-]
		\item Ein irreduzibles Polynom ist separabel, falls es keine mehrfachen Nullstellen besitzt.
		\item Ein Polynom ist separabel falls jeder seiner irreduziblen Faktoren separabel ist.
	\end{itemize}
\end{definition}

Zwei wichtige Resultate:
\begin{itemize}[-]
	\item Falls $E / k$ Zerfällungskörper eines separablen Polynoms $f \in k[x]$ ist, dann ist $[E:k] = \abs{\gal(E / k)}$.
	\item Ist $G \subseteq \aut(E)$ eine endliche Untergruppe, wobei $E$ beliebiger Körper, dann ist $[E : E^{G}] = \abs{G}$.
\end{itemize}

\begin{theorem}
	Sei $E / k$ eine endliche Erweiterung mit Galois Gruppe $G = \gal(E / k)$. Folgende Eigenschaften sind äquivalent:
	\begin{enumerate}[(1)]
		\item $E$ ist Zerfällungskörper eines separablen Polynoms in $k[x]$.
		\item $k = E^{G}$.
		\item Jedes irreduzible Polynom in $k[x]$ mit einer Nullstelle in $E$ ist separabel und zerfällt in $E$.
	\end{enumerate}
\end{theorem}

\begin{proof}
	$(1) \implies (2)$: Da $E$ Zerfällungskörper eines separablen Polynoms in $k[x]$ ist, folgt aus Theorem 17.2, $[E:k] = \abs{G}$.
	Da $G < \aut(E)$ eine endliche Untergruppe von $\aut(E)$ ist, folgt aus Proposition 4.9, $[E:E^{G}] = \abs{G}$.
	Da $E \supseteq E^{G} \supseteq k$, folgt
	\[
		[E^{G} : k] = \frac{[E : k]}{[E : E^{G}]} = \frac{\abs{G}}{\abs{G}} = 1 \implies E^{G} = k
	.\] 
	$(2) \implies (3)$ : Sei $p \in k[x]$ irreduzibel und $\alpha \in E$ mit $p(\alpha) = 0$. Wir können annehmen, dass $p$ unitär ist. Definiere
	\[
		q(X) := \prod_{\sigma \in \sfrac{G}{\operatorname{Stab}(\alpha)}} (X - \sigma(\alpha)) \qq{wobei} \operatorname{Stab}(\alpha) = \{\eta \in G \mid \eta(\alpha) = \alpha\}  
	.\]
	$q$ hat folgende Eigenschaften:
	\begin{enumerate}[(1)]
		\item $q$ unitär
		\item $R(q) \subseteq R(p)$,da $p \in k[x], \alpha \in E, \sigma \in \gal(E / k)$. Dann folgt $0 = p(\alpha) = \sigma(p(\alpha)) = p(\sigma(a))$.
		\item $q$ hat keine mehrfachen Nullstellen
		\item $q \in E^{G}[x]$ (Serie 8, A1)
	\end{enumerate}
	Aus der Voraussetzung $E^{G} = k$ folgt $q \in k[x]$. Aus (2) folgt, dass $q$ das Polynom $p$ dividiert; daraus folgt $q = p$.

	$(3) \implies (1)$ : Sei $k \subseteq E' \subseteq E$ maximal mit der Eigenschaft, dass $E'$ Zerfällungskörper eines separablen Polynoms $g$ in $k[x]$ ist.
	Z.Z. $E' = E$. Falls $E \subsetneq E$, sei $\alpha \in E \setminus E'$. Dann ist das Minimalpolynom $\irr(\alpha,k) \in k[x]$ ein irreduzibles Polynom
	mit einer Nullstellen in $E$, nämlich $\alpha$. Folglich ist $\irr(\alpha,k)$ separabel und zerfällt in $E$.
	Sei $f = g \cdot \irr(\alpha,k)$. Dann ist $f \in k[x]$ separabel und zerfällt in $E$. 
	Falls $E''$ Zerfällungskörper von $f$ ist, dann folgt $E' \subsetneq E'' \subseteq E$ \contra.
\end{proof}

\begin{definition}
	Eine endliche Erweiterung $E / k$ ist eine \emph{Galoiserweiterung von $k$}, falls $E$ die äquivalenten Eigenschaften von vorherigem Theorem 4.11 besitzt.
\end{definition}

$k \subseteq B \subseteq E$. Falls $E / k$ Galois ist, dass muss $B / k$ nicht unbedingt Galois sein, weil eine Galois Erweiterung insbesondere normal ist.
Andererseits sei $f \in k[x]$ separabel mit Zerfällungskörper $E$, dann ist $f \in B[x]$ immer noch separabel und folglich ist $E / B$ Galois.

\begin{corollary}
	Falls $k \subseteq B \subseteq E$ wobei $E / k $ Galois dann ist $E / B$ Galois.
\end{corollary}

\begin{proposition}
	Sei $k \subseteq B \subseteq E$ mit $E / k$ Galois. Dann ist $B / k$ Galois genau dann, wenn $\sigma(B) = B \forall \sigma \in \gal(E / k)$.
\end{proposition}

\begin{proof}
	$\implies:$ Falls $B / k$ Galois, ist insbesondere $B$ eine normale Erweiterung von $k$ d.h. $B$ ist Zerfällungskörper eines Polynoms in $k[x]$.
	Dann folgt aus Theorem 2.26, dass $\sigma(B) = B \forall \sigma \in G = \gal(E / k)$.\\
	$\impliedby:$ $E / k$ ist Galois und $\sigma(B) = B \forall \sigma \in \gal(E / k)$. Wir betrachten $\nstack{\gal(E / k) \to \gal(B / K)}{\sigma \mapsto \sigma \mid_{B}}$,
	das ist ein Homomorphismus. Sei $H < \gal(B / k)$ sein Bild.

	Wir haben $k \subseteq B^{\gal(B / k)} \subseteq B^{H} \subseteq E^{\gal(E / k)} = k$. Woraus folgt $k = B^{\gal(B / k)}$ also ist $B / k$ Galois.
\end{proof}

\begin{definition}
	Sei $G$ eine Gruppe, dann bezeichnet $\operatorname{Sub}(G)$ die Menge der Untergruppen von $G$, geordnet via Inklusion.
	Sei $E / k$ Körpererweiterung. Dann bezeichnet $\operatorname{Int}(E / k)$ die Menge der Zwischenkörper von $E / k$ d.h. Körpererweiterungen $B / k$ mit $B \subseteq E$.
	Auch $\operatorname{Int}(E / k)$ ist geordnet via Inklusion.
\end{definition}

\begin{theorem}[Galois Korrespondenz]
	Sei $E / k$ eine (endliche) Galois Erweiterung.
	\begin{enumerate}[(1)]
	\item Die Abbildung $\gamma: \nstack{\operatorname{Sub}(\gal(E / k)) \to \operatorname{Int}(E / k)}{H \mapsto E^{H}}$ ist eine inklusionsumkehrende Bijektion
		mit Inverser $\delta: \nstack{\operatorname{Int}(E / k) \to \operatorname{Sub}(\gal(E / k))}{B \mapsto \gal(E / B)}$.
	\item $B \in \operatorname{Int}(E / k)$ ist genau denn eine Galoiserweiterung von $k$ falls $\gal(E / B)$ eine \emph{normale} Untergruppe von $\gal(E / k)$ ist.
		In diesem Fall ist $\sfrac{\gal(E / k)}{\gal(E / B)} \cong \gal(B / k)$.
	\end{enumerate}
\end{theorem}

\begin{proof}
	\begin{enumerate}[(1)]
		\item $\gamma$ ist injektiv: Seien $H_1, H_2$ Untergruppen von $\gal(E / k)$ insbesondere sind $H_1, H_2$ endliche Untergruppen von $\aut(E)$.
			Dann folgt aus der Annahme $E^{H_1} = E^{H_2}$  und Korollar 4.10, dass  $H_1 = H_2$.
		\item $\gamma \circ \delta = \id_{\operatorname{Int}(E / k)}$ : Sei $k < B < E$,
			\[
				\gamma \delta(B) =  \gamma(\gal(E / B)) = E^{\gal(E / b)} \implies E^{\gal( E / B)} = B
			.\]
			In Korollar 4.13 hatten wir gezeigt, dass $E / k$ Galois $\implies E / B$ Galois. 
			Daraus folgt $\gamma \circ \delta = \id$ und schlussendlich ist $\gamma$ eine Bijektion
	\end{enumerate}
	Annahme: $B / k$ ist eine Galoiserweiterung. Insbesondere ist $k \subseteq B \subseteq E$ und $B$ ist Zerfällungskörper eines Polynoms in $k[x]$.
	Dann folgt schon aus Theorem 2.26, dass $\sigma(B) = B \forall \sigma \in \gal(E / k)$, $\gal(E / B)$ ist der Kern des Restriktionshomomorphismus.
	\[
		\nstack{\gal(E / K) \to \gal( B / k)}{\sigma \mapsto \sigma \mid_{B}}
	.\] 
	insbesondere $\gal(E / B) \lTri \gal(E / k)$, zudem induziert dieser Homomorphismus einen Isomorphismus  $\sfrac{\gal(E / k)}{\gal(E / B)} \cong \gal(B / k)$.

	Annahme: $E / k$ Galois, $k \subseteq B \subseteq E$ und $\gal(E / B) \lTri \gal(E / K)$. Zu zeigen: $B / k$ ist Galois. 
	Dazu benötigen wir die Charakterisierung von Proposition 4.14: es genügt zu zeigen, dass $\sigma(B) = B \forall \sigma \in \gal(E / k)$.\\
	Da $E / B$ Galois ist, folgt $B = E^{\gal(E / B)}$.
	Sei $\sigma \in \gal(E / K)$, $\xi \in B$ z.z. $\sigma(\xi) \in B \Leftrightarrow $ z.z. $\forall h \in \gal(E / B)$ gilt $h(\sigma(\xi)) = \sigma(\xi)$.
	\[
		h \sigma(\xi) = \sigma (\sigma^{-1} h \sigma)(\xi) = \sigma(\sigma^{-1} h \sigma(\xi)) = \sigma(\xi)
	.\] 
\end{proof}

\begin{eg}
	$f(x) = X^2-2 \in \Q[X]$ ist irreduzibel und sei $E \subseteq \C$ Zerfällungskörper von $f$.
	Dann ist $\gal(E / \Q) \cong S^{3}$.
	Sei $\beta = \sqrt[3]{2} \in \R$, $\omega = e^{\frac{2 \pi i }{3}}$.
	Nullstellen von $f$ sind dann: $\alpha_1 = \beta, \alpha_2 = \beta \omega, \alpha_3 = \beta \omega^2$.
	
	Sei $\sigma_{ij} \in S_{3}$ die Transposition die $\alpha_{i}$ mi $\alpha_{j}$ vertauscht.
	Sei $\tau \in S_3$ die zyklische Permutation, $\alpha_1 \to \alpha_2 \to \alpha_3 \to \alpha_1$.
	\[
	\begin{tikzcd}
                                               & S_3 \arrow[ld, no head] \arrow[d, no head] \arrow[rd, no head] \arrow[rrd, no head] &                                                &                                          &  &                                                   & \Q \arrow[d, no head] \arrow[rd, no head] \arrow[rrd, no head] \arrow[ld, no head] &                                                   &                                                  \\
\langle \sigma_{12}\rangle \arrow[rd, no head] & \langle \sigma_{13}\rangle \arrow[d, no head]                                       & \langle \sigma_{23}\rangle \arrow[ld, no head] & \langle \tau\rangle \arrow[lld, no head] &  & \Q(\alpha_3) \arrow[rd, "2" description, no head] & \Q(\alpha_2) \arrow[d, "2" description, no head]                                   & \Q(\alpha_1) \arrow[ld, "2" description, no head] & \Q(\omega) \arrow[lld, "3" description, no head] \\
                                               & \langle e\rangle                                                                    &                                                &                                          &  &                                                   & E                                                                                  &                                                   &                                                 
\end{tikzcd}
	.\] 
\end{eg}

Einfache Folgerungen der Galois Korrespondenz
\begin{corollary}
	Eine endliche Galois Erweiterung hat nur endlich viele Zwischenkörper.
\end{corollary}

\begin{definition}
	Eine Erweiterung $E / k$ ist \emph{einfach} falls es $u \in E$ gibt mit $E = k(u)$.
\end{definition}

\begin{proposition}
	Eine endliche Erweiterung $E / k$ ist genau dann einfach, falls es nur endlich viele Zwischenkörper gibt.
\end{proposition}

\begin{proof}
	$(\impliedby):$ Annahme: Es gibt nur endlich viele Zwischenkörper.
	\begin{enumerate}[{Fall} 1:]
		\item $k$ ist uneneldich; $E$ ist ein $k$-Vektorraum und da $k$ unendlich ist kann $E$ nicht Vereinigung seiner (endlich vielen)
			echten Zwischenkörper sein (Serie 8 A4). Also gibt es $x \in E$, dass in keinem echten Zwischenkörper enthalten ist $\implies E = k(x)$.
		\item $k$ endlich, $k = \F_{q}$, $E = \F_{q^{n}}$ wobei $n = [E : k]$.
			Sei $u$ ein Erzeuger der zyklischen Gruppe $\F_{q^{n}}^{\times}$ dann folgt $E = k(u)$.
	\end{enumerate}

	$(\implies):$ Sei $E = k(x)$. Sei $k \subseteq F \subseteq E$ ein Zwischenkörper.
	Sei $f_{F}(T) := \irr(x,F)(T) \in F[T] = T^{n} + a_{n-1} T^{n-1} + \ldots + a_0$.
	Sei $F_{0} = K(a_{n-1},\ldots,a_0) \subseteq F$.
	Es ist $f_{F}$ irreduzibel in $F[T]$ und folglich auch in $F_0[T]$. Weiters ist
	\begin{align*}
		[E:F] = [F(x):F] = n \qquad [E:F_0] = [F_0(x):F_0] = n
	\end{align*}
	und folglich $F = F_0$.
	\begin{remark}
		$\irr(x,F)$ dividiert in $E[T]$ das Minimalpolynom $\irr(x,k)$. Also ist die Anzahl der Zwischenkörper kleiner gleich der Anzahl der
		Polynome in $E[T]$ die $\irr(x,k)$ dividieren.
	\end{remark}
\end{proof}

\begin{corollary}
	Eine (endliche) Galois Erweiterung $E / k$ ist immer einfach.
\end{corollary}

\begin{eg}
	$E = \F_{p}(X,X), k = \F_{p}(X^{p},Y^{p})$.
	Dann ist $[E : k] = p^2$ und für $f \in E$ ist $f^{p} \in k$. Folgt es gibt kein $x \in E$ mit $E = k(x)$.
\end{eg}

\begin{theorem}
	Sei $E / k$ eine endliche Galois Erweiterung mit $\charak = 0$. Falls $\gal(E / k)$ auflösbar ist, so ist $E$ in einer radikalen Erweiterung von $k$ enthalten.
\end{theorem}

$G$ endlich auflösbar mit $\abs{G} \geq 2$ $\implies [G,G]  \subsetneq G$.
$\sfrac{G}{[G,G]}$ ist eine endliche abelsche Gruppe $\neq (e)$. Also ein Produkt von $\sfrac{\Z}{p^{n} \Z}$ wobei $p$ Primzahl und $n \geq 1$.

Insbesondere: $\sfrac{\Z}{p^{n}\Z} \supseteq \sfrac{\Z}{p^{n-1} \Z}$ mit Index $p$.
Also enthält $\sfrac{G}{[G,G]}$ eine Untergruppe $M < \sfrac{G}{[G,G]}$ mit Index $p$.
Sei $p : G \to \sfrac{G}{[G,G]}$ und $N := p^{-1}(M)$. Dann ist $N \lTri G$ und hat Index $p$.

$N \lTri G = \gal(E / k)$ und $E^{N} \supseteq k$.
$E^{N}$ ist eine Galois Erweiterung von $k$ von Grad $p$, $p$ eine Primzahl.

\begin{lemma}
	$E / k$ endliche Galois Erweiterung mit  $p := [E : k]$ Primzahl.
	Falls $k$ eine $p$-te Wurzel von $1$ enthält mit $w \neq 1$ dann gibt es $\xi \in E$ mit $\xi^{p} \in k$ und $E = k(\xi)$.
\end{lemma}

\begin{proof}
	$\gal(E / k)$ ist zyklisch der Ordnung $p$, sei $\sigma \in \gal(E / k)$ ein Erzeuger.
	Wir betrachten $\sigma: E \to E$ als $k$-lineare Abbildung.
	Wir bemerken $\sigma^{p} = \id_{E}$. Behauptung $X^{p}-1$ ist das Minimalpolynom von $\sigma$.
	
	Falls es ein Polynom $P \in k[x]$ gibt  mit $\deg(P) \leq p-1$ und $P(\sigma) = 0$.
	dann folgt $ \sum_{i=0}^{p-1} a_{i} \sigma^{i} = 0$, $a_{i} \in k$. Aber $1,\sigma,\ldots,\sigma^{p-1}$ sind Paarweise verschiedene Charaktere
	in $\hom(E^{\times}, E^{\times})$ und daher (nach Prop 4.5) linear unabhängig in $F(E^{\times},E)$. Also $P \cong 0$.
	Folglich ist $X^{p}-1$ das charakteristische Polynom von $\sigma$.
	Folglich ist $\omega \in k$ ein Eigenwert von $\sigma$. Sei $\xi \in E \setminus \{0\}$ ein Eigenvektor mit $\sigma(\xi) = \omega \xi$ :
	Da $\omega \neq 1$ folgt $\xi \neq k$ und somit $E = k(\xi)$.

	Zudem: $\sigma(\xi^{p}) = \sigma(\xi)^{p} = \omega^{p} \xi^{p} = \xi^{p}$. Da $\sigma$ die Galoisgruppe $\gal(E / k)$ erzeugt folgt $\xi^{p} \in E^{\gal(E / k)} = k$.
\end{proof}

\begin{proof}[Beweis vom Satz]
	Sei $E / k$ Galois mit $\gal(E / k)$ auflösbar. 
	Induktiv über $[E:k]$. Falls $[E:k] = 1$ ok.
	Annahme: $[E:k] \geq 2$, $\abs{\gal(E / k)} = [E:k] \geq 2$.
	Es gibt eine Primzahl $p$ und $N \lTri \gal(E / k)$ von Index $p$.

	$k \subseteq E^{N}$. Also ist $E^{N} / k$ eine Galois Erweiterung von Grad $p$.
	Sei $k^{*} / k$ ein Zerfällungskörper von $X^{p}-1 \in k[x]$ und $w \in k^{*}$ eine erzeugende $\{\eta \in (k^{*})^{\times} \mid \eta^{p} = 1\}$. Also $w \neq 1$.

	\begin{enumerate}[{Fall} 1]
		\item $w \in k$. Dann (Lemma 4.25) $E^{N} / k$ reine Erweiterung. Und $[E:E^{N}] < [E:k]$.
			Außerdem $\gal(E / E^{N}) = N$ und $N < \gal(E / k)$, daher auflösbar.
			Also folgt aus Induktionshypothese, dass es einen Turm von Erweiterungen gibt,
			\[
			K_1 = E^{N} \subseteq K_2 \subseteq \ldots \subseteq K_{t}
			\] 
			wobei $K_{i} / K_{i-1}$ rein ist, $2 \leq i \leq t$.
			Also ist $k \subseteq K_1 = E^{N} \subseteq K_2 \subseteq \ldots \subseteq K_{t}$ und $K_{t}$ eine radikale Erweiterung von $k$ mit $K_{t} \supseteq E$.

			%TODO Fall 2
	\end{enumerate}
\end{proof}



































