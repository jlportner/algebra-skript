%! TEX root=../algebra.tex
\graphicspath{{Images/}}

\chapter{Galois Korrespondenz}

\begin{definition}
	Sei $E$ ein Körper und $H \subseteq \aut(E)$ dann ist $E^{H} := \{x \in E \mid \sigma(x) = x \forall \sigma \in H\} $ ist ein Unterkörper von $E$.
	Dann ist $E^{H}$ der \emph{Fixkörper} von $H$.
\end{definition}

\begin{remark}
	Die Korrespondenz $H \mapsto E^{H}$ hat folgende Monotonie Eigenschaft: $H_1 \subseteq H_2 \implies E^{H_2} \subseteq E^{H_1}$.
\end{remark}

\begin{eg}
	$E / k$ Erweiterung $k \subseteq E^{\gal(E / k)}$; das Beispeil $k = \F_{p}(t)$ und $f(x) = X^{p}-t$, $E$ Zerfällungskörper von $f$, $\gal(E / k) = (e)$.
	In diesem Fall $k \subsetneq E^{\gal(E / k)} = E$.
\end{eg}

\textbf{Ziel:} Bestimmung des Grades $[E : E^{H}]$ wobei $H < \aut(E)$ eine \emph{endliche Untergruppe} bezeichnet.

\begin{definition}
	Sei $G$ eine Gruppe, $E$ ein Körper.
	Ein \emph{Charakter von $G$ in $E$} ist ein Gruppenhomomorphismus $G \to E^{\times}$.
	Wobei $E^{\times}$ die Multiplikative Gruppe $E \setminus \{0\}$ ist.

	Die Menge der Charaktere von $G$ in $E$ wird mit $\hom(G,E^{\times})$ bezeichnet.
	Man kann $H(G,E^{\times})$ als Teilmenge des Vektorraums $F(G,E)$ aller $E$-wertigen Funktionen auf $G$.
\end{definition}

\begin{proposition}[Dedekind]
	$\hom(G,E^{\times}) \subseteq F(G,E)$ ist linear unabhängig.
\end{proposition}

\begin{proof}
	Sei $n \geq 1$ minimal so dass es $n$ Charaktere $\sigma_1,\ldots,\sigma_{n}$ gibt die in $F(G,E)$ linear abhängig sind.
	Also gibt es $c_1 \neq  0,\ldots, c_{n} \neq 0$ in $E$ mit $c_1 \sigma_1(x) + \ldots + c_{n} \sigma_{n}(x) = 0 \forall x \in G$.

	Da $\sigma_1 \neq \sigma_{n}$, sei $y \in G$ mit $\sigma_1(y) \neq \sigma_{n}(y)$.
	Wir ersetzen $x$ durch $xy$ und erhalten
	\[
		c_1 \sigma_1(x) \sigma_1(y) + \ldots + c_{n} \sigma_{n}(x) \sigma_{n}(y) = 0 \implies c_1 \sigma_1(x) \frac{\sigma_1(y)}{\sigma_{n}(y)} + 
		\ldots + c_{n-1} \sigma(x) \frac{\sigma_{n-1}(y)}{\sigma_{n}(y)} + c_{n} \sigma_{n}(x) = 0 \forall x \in G
	.\] 
	verwenden wir nun die erste Gleichung erhalten wir
	\[
		c_1 \sigma_1(x) \left( \frac{\sigma_1(y)}{\sigma_{n}(y)} - 1 \right) + \ldots + c_{n-1} \sigma_{n-1}(x) \left( \frac{\sigma_{n-1}(y)}{\sigma_{n}(y)} - 1 \right) 
	.\]
	Nicht triviale lineare Relation; $\frac{\sigma_1(y)}{\sigma_{n}(y)}- 1 \neq 0$ \contra
\end{proof}

Benutze diesen Satz um eine untere Schranke von $[E: E^{H}]$ zu bestimmen falls $H \subseteq \aut(E)$ eine \emph{endliche Teilmenge} besitzt.

\begin{lemma}[Sublemma]
	Sei $E$ ein Körper, $S$ eine Menge und $\{\sigma_1,\ldots,\sigma_{n}\} \subseteq G(S,E)$ linear unabhängig.
	Dann gibt es $s_1,\ldots,s_{n} \in S$ mit 
	\[
		\begin{pmatrix} \sigma_1(s_1) \\ \vdots \\ \sigma_{n}(s_1) \end{pmatrix} ,\ldots, \begin{pmatrix} \sigma_1(s_{n}) \\ \vdots \\ \sigma_{n}(s_{n}) \end{pmatrix} 
	\] 
	in $E^{n}$ linear unabhängig sind.
\end{lemma}

\begin{lemma}
	Sei $H = \{\sigma_1,\ldots,\sigma_{n}\} \subseteq \aut(E)$, Teilmenge mit $n$ Elementen.
	Dann gilt $[E : E^{H}] \geq n$.
\end{lemma}

\begin{claim}
	Falls $\left< H \right>	$ die von $H$ erzeugte Untergruppe von $\aut(E)$ bezeichnet so ist $E^{H} = E^{\left< H \right>}$.
\end{claim}

\begin{proof}
	Wir betrachten $\sigma_1 \mid_{E^{\times}},\ldots, \sigma_{n} \mid_{E^{\times}}$ als Elemente von $\hom(E^{\times}, E^{\times})$.
	Also sind sie in $F(E^{\times }, E)$ linear unabhängig (Dedekind) und es gibt (Sublemma) $y_1,\ldots,y_{n} $ in $E^{x} $ mit
	\[
		\begin{pmatrix} \sigma_1(y_1) \\ \vdots \\ \sigma_{n}(y_1) \end{pmatrix} ,\ldots, \begin{pmatrix} \sigma_1(y_{n}) \\ \vdots \\ \sigma_{n}(y_{n}) \end{pmatrix} 
	\] 
	linear unabhängig in $E^{n}$.
	\begin{claim}
		$y_1,\ldots,y_{n}$ sind linear unabhängig über  $E^{H}$.
	\end{claim}
	Sei $c_1,\ldots, c_{n} $ in $E^{H}$ mit $c_1 y_1 + \ldots + c_{n} y_{n} = 0$. Für alle $1 \leq j \leq n$ folgt:
	\[
		\underbrace{\sigma_{j}(c_1 y_1 + \ldots + c_{n} y_{n})}_{c_1 \sigma_{j}(y_1) + \ldots + c_{n} \sigma_{j}(y_{n}) = 0} = 0 
	\] 
	d.h.
	\[
	\begin{pmatrix} 
		\sigma_{1}(y_1) &\ldots &\sigma_{1}(y_{n})\\
		\vdots &&\vdots\\
		\sigma_{n}(y_1) &\ldots &\sigma_{n}(y_{n})
	\end{pmatrix}
	\begin{pmatrix} 
		c_1\\ \vdots\\ c_{n}
	\end{pmatrix} = \begin{pmatrix} 
		0\\ \vdots\\ 0
	\end{pmatrix} 
	.\] 
	Diese Matrix hat Rang $n \implies c_1 = \ldots = c_{n} = 0$.
\end{proof}

\begin{proposition}
	Sei $G < \aut(E)$ eine endliche Untergruppe. Dann gilt $[E : E^{G}] = \abs{G}$.
\end{proposition}

\begin{proof}
	Sei $G = \{\sigma_1,\ldots,\sigma_{n}\}$ also $n = \abs{G}$. Aus Lemma folgt $[E : E^{G}] \geq n$.

	\textbf{Annahme:} $[E: E^{G}] > n$. Seien $b_1,\ldots, b_{m} \in E$ linear unabhängig über $E^{G}$ mit $m > n$.

	Wir definieren $T: E^{m} \to  E^{n}$
	\[
	\begin{pmatrix} 
		x_1 \\ \vdots \\ x_{m}
	\end{pmatrix} \mapsto \begin{pmatrix} 
	\sigma_1(b_1) &\ldots &\sigma_{1}(b_{m})\\
	\sigma_2(b_1) &\ldots &\sigma_{2}(b_{m})\\
	\vdots & &\vdots\\
	\sigma_n(b_1) &\ldots &\sigma_{n}(b_{m})
	\end{pmatrix} \begin{pmatrix} 
		x_1 \\ \vdots\\ x_{m}
	\end{pmatrix} 
	\] 
	lineare Abbildung. Dann gilt:
	\begin{enumerate}
		\item $\ker(T) \neq 0$ (klar da $m > n$ )
		\item  \[
		\begin{pmatrix} 
			x_1\\ \vdots\\ x_{m}
		\end{pmatrix} \in \ker(T) \implies \begin{pmatrix} 
		\sigma(x_1)\\ \vdots\\ \sigma(x_{m})
	\end{pmatrix} \in \ker(T) \forall \sigma \in G
		.\] 
	\end{enumerate}
	Wir zeigen $(2)$:  $(x_1,\ldots,x_{m})^{T} \in \ker(T) \Leftrightarrow \sum_{j=1}^{m} \sigma_{i}(b_{j})x_{j} = 0 \forall 1 \leq i \leq n :\implies \sum_{j=1}^{m} \sigma \sigma_{i}
	(b_{j}) \sigma(x_{j}) = 0$. Gegeben $\sigma \in G$. Sei $s \in S_{n}, \sigma \sigma_{i} = \sigma_{s(i)}$ :
	\[
		\sum_{j=1}^{m} \sigma_{s(i)}(b_{j}) \sigma(x_{j}) = 0 \forall 1 \leq i \leq n \implies \begin{pmatrix} 
			\sigma(x_1)\\ \vdots\\ \sigma(x_{n})
		\end{pmatrix} \in \ker(T)
	.\] 
	Sei $r := \min \{k \in \N : k \geq 1, v \in \ker(T) \setminus \{0\} \text{ besitzt $k$ Koordinaten} \neq 0\} $.
	Sei $(x_1,\ldots,x_{m})^{T} \in \ker(T)$ mit $r$ Koordinaten $\neq 0$. O.B.d.A sei $x_1 \neq 0$.
	Dann folgt
	\[
		\frac{1}{\sigma(x_1)} \begin{pmatrix} 
			\sigma(x_1) \\ \vdots \\ \sigma(x_{m})
		\end{pmatrix} - \frac{1}{x_1} \begin{pmatrix} 
			x_1 \\ \vdots \\ x_{m}
		\end{pmatrix} \in \ker(T) = \begin{pmatrix} 
		0\\ \frac{\sigma(x_2)}{\sigma(x_1)} - \frac{x_2}{x_1}\\ \vdots \\ \frac{\sigma(x_{m})}{\sigma(x_1)} - \frac{x_{m}}{x_1}
		\end{pmatrix} = \begin{pmatrix} 
			0 \\ 0 \\ \vdots \\ 0
		\end{pmatrix}  
	\]
	aus Definition von $r$. Es ist $\sigma(\frac{x_{i}}{x_{i}} = \frac{x_{i}}{x_{i}} 1 \leq i \leq m$ d.h.
	\[
	\begin{pmatrix} 
		1\\ \frac{x_2}{x_1}\\ \vdots \\ \frac{x_{m}}{x_1}
	\end{pmatrix} = \frac{1}{x_1}\begin{pmatrix} 
		x_1\\ x_2\\ \vdots \\ x_{m}
	\end{pmatrix} \in \ker(T) \cap (E^{G})^{m}
	.\]
	Das heißt aber $\sum_{j=1}^{m} \sigma_{i}(b_{j}) \frac{x_{j}}{x_{1}} = 0 \forall \sigma_1,\ldots,\sigma_{n}$. Aber, da $\id \in G < \aut(E)$ ist
	$\sum_{j=1}^{m} b_{j} \frac{x_{j}}{x_{i}} = 0$ d.h. $b_1 + b_2 \frac{x_2}{x_1} + \ldots + b_{m} \frac{x_{m}}{x_{1}} = 0$ 
	ist eine nicht triviale lineare Relation über $E^{G}$. \contra 
\end{proof}

\begin{corollary}
	Seien $G,H$ endliche Untergruppen von $\aut(E)$. Dann gilt $E^{G} \subseteq E^{H} \Leftrightarrow H < G$.
\end{corollary}

\begin{proof}
	$\impliedby:$ trivial.\\
	$\implies:$ Wir nehmen an $E^{G} \subseteq E^{H}$. Falls $H \neq G$ gibt es $\sigma \in H$ mit $\sigma \not\in G$.
	Da $\sigma \in H$ fixiert $\sigma$ jedes Element von $E^{H}$ und deswegen auch von $E^{G}$ also $ E^{G} = E^{G \cup \{\sigma\} }$.

	$\abs{G} = [E : E^{G}] = [E: E^{G \cup \{\sigma\}}] \geq \abs{G} + 1$. \contra
\end{proof}

\begin{corollary}
	Seien $G,H$ endliche Untergruppen von $\aut(E)$. Dann ist $E^{G} = E^{H} \Leftrightarrow H = G$.
\end{corollary}



























