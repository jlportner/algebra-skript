%! TEX root=../algebra.tex
\graphicspath{{Images/}}

\chapter{Lösung durch Radikale und auflösbare Gruppen}

Sei $K = k(u)$ eine Körpererweiterung von $k$, $u \neq 0$.
Dann ist $\{n \in \Z \mid u^{n} \in k\}$ ist eine Untergruppe von $\Z$ und deshalb von der Form $m \Z$ wobei $m \in \N$
eindeutig bestimmt.

\begin{definition}
	$k(u) / k$ ist eine reine Erweiterung vom Typ $m$ falls $m \Z = \{n \in \Z \mid u^{n} \in k\} \neq 0$
\end{definition}

\begin{definition}
	Eine Körpererweiterung $K / k$ heißt radikal falls es einen Turm von Zwischenkörpern gibt
	\[
	k = K_0 \subseteq K_1 \subseteq \ldots \subseteq K_{t} = K
	\] 
	so dass $K_{i+1} / K_{i} \forall 0 \leq i \leq t-1$ reine Erweiterungen sind.
\end{definition}

\begin{definition}
	Ein Polynom $f \in k[x]$ ist mittels \emph{redikalen Lösbar} falls ein Zerfällungskörper von $f$ in einer radikalen Erweiterung
	von $k$ enthalten ist.
\end{definition}

$k(u) / k$ : $u^{m} \in k$ $u$ ist $m$-te Wurzel von einem Element in $k$.
Sei $E$ der Zerfällungskörper von $f$.
%TODO missing 8:28 26.03.2021

\begin{eg}
	Sei $f(x) = x^2 + b x + c \in k[x]$. Sei $E$ Zerfällungskörper von $f$ und $R(f) = \{\alpha_1,\alpha_2\}$. Sei $\alpha \in R(f)$.
	\begin{align*}
		\alpha^2 + b \alpha + c = 0 \implies (\alpha + \frac{b}{2})^2 + c - \frac{b^2}{4} = 0
	.\end{align*}
	Falls $u := \alpha_1 +\frac{b}{2} \in E \supseteq k(u)$. Dann ist $u^2 = \frac{b^2}{4}-c \in k$.
	Da $E = k(\alpha_1,\alpha_2)$ : $\alpha_1 = u-\frac{b}{2}$, $\alpha_2 + \alpha_1 = -b \implies \alpha^2 = -b -\alpha_1 = -u -\frac{b}{2}$ 
	ist  $\{\alpha_1,\alpha_2\} \in k(u) \implies E = k(u)$.
\end{eg}

Sei $k(u) / k$ eine reine Erweiterung von Typ $m \geq 1$.
Sei $m = p_1 \cdot \ldots \cdot p_{r}$ eine Zerlegung in Primzahlen.
\[
	k(u) \supseteq k(u^{p_1}) \supseteq k(u^{p_1 p_2} \supseteq \ldots \supseteq k(u^{m}) = k
\] 
wobei die erste Erweiterung von Typ $p_1$, die zweite von Typ $p_2$ etc. ist.
Dies Führt zum Studium von $x^{p} - c \in k[x]$.

\begin{lemma}
	Sei $p$ ein Primzahl. Sei $f(x) = x^{p}-c \in k[x]$.
	\begin{enumerate}[(1)]
		\item Folgende Dichotomie:
			\begin{enumerate}[({1}.1)]
				\item $(f)$ ist irreduzibel
				\item $c$ ist eine $p$-te Potenz eines Elements in $k$
			\end{enumerate}
		\item Sei $E / k$ der Zerfällungskörper von $f$. Wir nehmen an, $k$ enthält alle $p$-ten Wurzeln von $1$.
			Sei $u \in E, u \in R(f)$. Dann ist $E = k(u)$.
			\begin{enumerate}[(2.1)]
				\item $f$ irreduzibel:
					\begin{itemize}
						\item Falls $\charak(k) \neq p$ ist $\gal(E / k) \cong \sfrac{\Z}{p \Z}$ 
						\item Falls $\charak(k) = p$ ist $\gal(E / k) \cong e$.
					\end{itemize}
				\item $f$ reduzibel so ist $E = k$ und $\gal(E / k) \cong (e)$.
			\end{enumerate}
	\end{enumerate}
\end{lemma}

\begin{proof}
	\begin{enumerate}[(1)]
		\item Wir nehmen an $f$ ist reduzibel: $f = g \cdot h$ mit $g(x) = x^{d} + b_{d-1} x^{d-1} + \ldots + b_0, 1 \leq d < p$.
			Sei $E$ Zerfällungskörper von $f(x) = x^{p}-c$ und $u \in R(f)$.

			Bemerkung: $w \in R(f)$ dann ist $u^{p} = w^{p} = c \implies (\frac{u}{w})^{p} = 1$.

			Daraus folgt $R(f) = \{u \cdot \xi \mid \xi^{p} = 1, \xi \in E\}$.
			Da $R(g) \subseteq R(f)$ und $b_0$ das Produkt aller Nullstellen von $g$ ist, ist $b_0 = u^{d} \cdot \eta, \eta^{p} =1$.
			Folgt $b_0^{p} = u^{d p} = (u^{p})^{d} = c^{d}$.
			Da $p$ eine Primzahl ist und $1 \leq d < p$ sind $p$ und $d$ Teilerfremd $\implies \exists r,s \in \Z$ mit $r p + s d = 1$.
			\[
				c = c^{rp + sd} =  (c^{r})^{p} \cdot (c^{d})^{s} = (c^{r})^{p} \cdot (b_0^{p})^{s} = (c^{r} b_0^{s})^{p}.
			.\] 
		\item Es ist $k \supseteq \{\xi \in E \mid \xi^{p} = 1\} \implies E = k(u)$ weil $R(f) = \{u \xi \mid \xi^{p} = 1\}$.
			\begin{enumerate}[(1)]
				\item $f$ irreduzibel: Sei $\charak(k) \neq p$ dann ist $f' = p x^{p-1} \neq 0 \implies f, f'$ Teilerfremd.
					Also ist $f$ separabel.
					\[
						\abs{\gal(E / k)} = [E : k] = [k(u) : k] = p
					\] 
					und also $\gal(E / k) \cong \frac{\Z}{p \Z}$.
			\end{enumerate}
			Der Rest ist Übung.
	\end{enumerate}
\end{proof}


Sei $f \in k[x]$. $k \subseteq E \subseteq K$ mit $E$ Zerfällungskörper, $K$ Radikale Erweiterung. 
In Verbindung bringen mit Galois Gruppe.
Wir wollen zeigen, dass jede radikale Erweiterung $K / k$ in einer normalen radikalen Erweiterung $F$ enthalten ist.
\[
k \subseteq E \subseteq K \subseteq F
\] 
normal und Radikal. Aus Satz 2.26 folgt $\nstack{\gal(F / k) \to \gal( E / k)}{\sigma \mapsto \sigma \mid_{E}}$ surjektiv.
Falls wir zeigen, dass $\gal(F / k)$ von $\frac{f}{k}$ normal radikal auflösbar ist.
Dann folgt, dass $\gal(E / k)$ auflösbar ist.
In Algebra I hatten wir den Satz
\begin{theorem}
	Jede Untergruppe und jeder Quotient einer auflösbaren Gruppe ist auflösbar.
\end{theorem}

Kontext folgender zwei Lemmata: Sei $B = k(u_1,\ldots,u_{t})$ eine endliche Erweiterung von $k$.
Insbesondere sind $u_1,\ldots,u_{t}$ algebraisch über $k$.
Sei $p_{i} = \irr(u_{i},k) \in k[x]$ das Minimalpolynom von $u_{i}$ über $k$.
Sei $f = p_1 \ldots p_{t} \in k[x]$. Sei $E$ Zerfällungskörper von $f$ und $G = \gal(E / k) = \{\sigma_1,\ldots,\sigma_{l}\}	$.

\begin{lemma}
	$E = k(\sigma(u_1),\ldots,\sigma(u_{t}), \sigma \in G) = k \begin{pmatrix} 
		\sigma_1(u_1), &\ldots, &\sigma_{l}(u_1)\\
		\vdots & &\vdots\\
		\sigma_1(u_{t}), &\ldots, &\sigma_{l}(u_{t})
	\end{pmatrix} $
\end{lemma}

\begin{proof}
	Da $E$ Zerfällungskörper von $f$ und $f = p_1 \cdot \ldots\cdot p_{t}$ folgt $R(p_{i}) \subseteq E, 1 \leq i \leq t$.
	Sei $1 \leq i \leq t$, $u, u'$ in $R(p_{i})$. Nach Lemma 2.15 gibt es einen Isomorphismus $\varphi: k(u) \subseteq E \to k(u') \subseteq E$ 
	der $\id_{k}: k \to k$ erweitert, $\varphi(u) = u'$.
	Da $f \in k[x] \subseteq k(u)[x]$ und $\varphi_{*}(f) = f$, da $E$ Zerfällungskörper von $f \in k(u)[x]$ ist
	folgt aus Proposition 2.16, dass $\varphi$ sich zu einem Isomorphismus $\Phi: E \to E$ erweitert.
	Insbesondere ist $\Phi \in \gal(E / k)$ und $\Phi(u) = u'$.

	Insbesondere: angewendet auf $u_{i} \in R(p_{i})$ folgt $\forall u' \in R(p_{i}) \exists \sigma \in \gal(E / k)$ mit $\sigma(u_{i}) = u'_{i}$.
	\[
		R(f) = \bigcup_{i=1}^{t} R(p_{i}) \subseteq \{\sigma(u_{i}) \mid 1\leq i \leq t, \sigma \in \gal(E / k)\} 
	.\] 
	Folglich $R(f) \subseteq k(\sigma(u_1),\ldots,\sigma(u_{t}) \mid \sigma \in G) \subseteq E$.
	Da $E$ Zerfällungskörper von $f$ ist folgt $k(\sigma(u_1),\ldots,\sigma(u_{t}) \mid \sigma \in G) = E$.
\end{proof}

\begin{lemma}
	Im Kontext von Lemma 3.6 nehmen wir an, dass: $u_1^{m_1} \in k, u_2^{m_2} \in k(u_1),\ldots, u_{t}^{m_{t}} \in k(u_1,\ldots,u_{t-1})$.
	Dann ist $E / k$ eine radikale Erweiterung.
\end{lemma}

\begin{proof}
	Sei $B_1 = k(\sigma_1(u_1),\ldots,\sigma_{l}(u_{1}))$ und induktiv
	\[
		B_{j} = B_{j-1}(\sigma_1(u_{j}),\ldots,\sigma_{l}(u_{j})) \quad 2 \leq j \leq t
	.\] 
	Wir erhalten $k \subseteq B_1 \subseteq B_2 \subseteq \ldots \subseteq B_{t} = E$.
	Wir zeigen, dass $B_{j}$ eine radikale Erweiterung von $B_{j-1}$ ist $\forall 2 \leq j \leq t$,
	sowie, dass $B_1$ eine radikale Erweiterung von $k$ ist.

	$B_{1} = k(\sigma_1(u_1),\ldots,\sigma_{l}(u_1))$.
	\[
		k \subseteq k(\sigma_1(u_1)) \subseteq k(\sigma_1(u_1),\sigma_2(u_1)) \subseteq \ldots \subseteq k(\sigma_1(u_1),\ldots,\sigma_{j-1}(u_1))
		\subseteq \underbrace{k(\sigma_1(u_1),\ldots,\sigma_{j}(u_1))}_{k(\sigma_1(u_1),\ldots,\sigma_{j-1}(u_1))(\sigma_{j}(u_1)} \subseteq \ldots = B_1
	.\] 
	Nun ist $\sigma_{j}(u_1)^{m_1} = \sigma_{j}(\underbrace{u_1^{m_1}}_{\in k}) = u_1^{m_1} \in k \subseteq k(\sigma_1(u_1),\ldots,\sigma_{j-1}(u_1))$.
	Folglich ist $\forall 1 \leq j \leq t$, $k(\sigma_{1}(u_1),\ldots,\sigma_{j}(u_1))$ eine reine Erweiterung von $k(\sigma_1(u_1),\ldots,\sigma_{j-1}(u_1))$.
	Um zu zeigen, dass $B_{j}$ eine radikale Erweiterung von $B_{j-1}$ ist bemerken wir zuerst, dass $\sigma(B_{j}) = B_{j} \forall \sigma \in \gal(E / k)$.

	\textbf{Induktiv:} Für $B_1 = k(\sigma_1(u_1),\ldots,\sigma_{l}(u_1))$: Betrachte $\sigma(B_1) = k(\sigma \sigma_1(u_1),\ldots, \sigma \sigma_{l}(u_1))$ 
	da $\{\sigma \sigma_1,\ldots,\sigma \sigma_{l}\}$ eine Permutation von $\{\sigma_1,\ldots,\sigma_{l}\}$ ist folgt $= k(\sigma_1(u_1),\ldots,\sigma_{l}(u_1)) = B_1$.
	Mit dem selben Argument folgt aus $\sigma(B_{j-1}) = B_{j-1}$, dass $\sigma(B_{j}) = B_{j}$.

	\[
		B_{j-1} \subseteq B_{j-1}(\sigma_1(u_{j})) \subseteq B_{j-1}(\sigma_1(u_{j}),\sigma_2(u_{j})) \subseteq \ldots \subseteq B_{j-1}(\sigma_1(u_{j}),\ldots,\sigma_{r-1}(u_{j}))
		\subseteq \underbrace{B_{j-1}(\sigma_1(u_{j}),\ldots,\sigma_{r}(u_{j}))}_{B_{j-1}(\sigma_1(u_{j}),\ldots,\sigma_{r-1}(u_{j}))(\sigma_{r}(u_{j}))} \subseteq \ldots = B_{j}
	.\] 
	Weiters sei $u_{j}^{m_{j}} \in k(u_1,\ldots,u_{j-1}) \subseteq B_{j-1}$ dann ist
	$\sigma_{r}(u_{j})^{m_{j}} = \sigma_{r}(u_{j}^{m_{j}}) \subseteq \sigma_{r}(B_{j-1}) = B_{j-1}$.
\end{proof}

\begin{corollary}
	Sei $K / k$ eine radikale Erweiterung. Dann gibt es $k \subseteq K \subseteq F, F / k$ radikal und normal.
\end{corollary}

\begin{proof}
	Lemma 3.7
\end{proof}

\begin{definition}[Algebra I]
	Eine Gruppe $G$ ist auflösbar falls es eine subnormale Folge
	\[
	\{e\} = G_0 \lTri G_1 \lTri g_2 \lTri \ldots \lTri G_{t} = G
	\]
	gibt mit $\sfrac{G_{i+1}}{G_{i}}$ abelsch $0 \leq i \leq t-1$.
\end{definition}

\begin{eg}[Algebra I]
	$A_{n},S_{n}$ sind auflösbar falls $n \leq 4$.\\
	$A_{n}$ ist einfach (nicht abelsch) $\forall n \geq 5$.
\end{eg}

Es gibt ein Kriterium für Auflösbarkeit, dass iterierte Kommutatorunterguppen benützt.
Für eine Gruppe $G$ bezeichnet $[G,G]$ die von $\{[a,b] \mid a,b \in G\} $ erzeugte Untergruppe.
Hier ist $[a,b] = a b a^{-1} b^{-1}$.
Die Untergruppe $[G,G]$ ist \emph{charakteristisch} d.h. $\forall \alpha \in \aut(G)$ ist $\alpha([G,G]) = [G,G]$.

Wir führen folgende Notation ein $G^{(1)} = [G,G] = $ Kommutatorgruppe, $G^{(j)} = [G^{(j-1)},G^{(j-1)}]$.
\begin{proposition}
	$G$ ist genau dann auflösbar falls es $n$ gibt mit $G^{(n)} = (e)$.
\end{proposition}

\begin{proof}
$\sfrac{G}{[G,G]} := G_{ab} = $ größte abelsche Quotient von $G$ in folgendem Sinn:
\[
\begin{tikzcd}
G \arrow[d, "\pi"'] \arrow[r, "\varphi"]             & A \\
{\sfrac{G}{[G,G]}} \arrow[ru, "\overline{\varphi}"'] &  
\end{tikzcd}
.\] 
$\varphi$ hom. und $A$ abelsch.

Betrachte $G \supseteq G^{(1)} \supseteq G^{(2)} \supseteq \ldots \supseteq G^{(j-1)} \supseteq G^{(j)} \supseteq \ldots \supseteq G^{(n)} = e$.
Es ist $\sfrac{G^{(j-1)}}{G^{(j)}}$ abelsch.
Dies zeigt, falls $G^{(n)} = e \implies G$ ist auflösbar.
\end{proof}

\begin{proposition}
	\begin{enumerate}[(1)]
		\item $H < G$ : $G$ auflösbar $\implies H$  auflösbar.
		\item $N \lTri G$ : $G$ ist gdw. auflösbar falls $N$ und $\sfrac{G}{N}$ auflösbar ist.
	\end{enumerate}
\end{proposition}

\begin{proof}
	Wir beginnen mit einer allgemeinen Bemerkung. 
	Sei $\varphi: G \to L$ ein Homomorphismus.
	Dann folgt $\varphi([a,b]) = [\varphi(a),\varphi(b)] \forall a,b \in G$. Woraus $\varphi([G,G]) \subseteq [L,L]$.
	Falls $\varphi$ \emph{surjektiv} folgt $\varphi([G,G]) = [L,L]$.

	Induktiv: $\varphi(G^{(j)}) \subseteq L^{(j)}$ und falls $\varphi$ surjektiv, $\varphi(G^{(j)}) = L^{(j)}$.
	\begin{enumerate}[(1)]
		\item $H < G \implies [H,H] \subseteq [G,G] \implies H^{(j)} \subseteq G^{(j)}$, falls es $n$ gibt mit $G^{(n)} = (e)$ folgt $H^{(n)} = (e)$.
		\item Annahme: $G$ ist auflösbar, dann ist $N$ wegen $(1)$ auflösbar. Sei $\pi: G \to \sfrac{G}{N}$ der Quotientenhomomorphismus.
			Da $\pi$ surjektiv ist, folgt
			\[
				\pi(G^{(j)}) = \left( \sfrac{G}{N} \right)^{(j)}
			.\] 
			Falls $G^{(n)} = e \implies (e) = \pi(G^{(n)}) = \left( \sfrac{G}{N} \right)^{(n)}$.
			
			Annahme: $N$ und $\sfrac{G}{N}$ auflösbar. Wir benützen $\pi(G^{(j)}) = \left( \sfrac{G}{N} \right)^{(j)}$.
			Sei $n \geq 1$ mit $\left( \sfrac{G}{N} \right)^{(n)} = (e)$. Dann folgt
			\[
				\pi(G^{(n)}) = (e) \implies G^{(n)} \subseteq N
			.\] 
			Da $N$ auflösbar, gibt es $l \geq 1$ mit $N^{(l)} = (e)$. Woraus folgt: $G^{(n+l)} = \left( G^{(n)} \right)^{(l)} \subseteq N^{(l)} = (e)$.
	\end{enumerate}
\end{proof}

\begin{theorem}
	Sei $f \in k[X]$, $E$ ein Zerfällungskörper von $f$. Falls $f$ mittels Radikalen lösbar ist, folgt, dass $\gal(E / k)$ auflösbar ist.
\end{theorem}

\begin{lemma}
	Sei $k = K_0 \subseteq K_1 \subseteq \ldots \subseteq K_{t}$ ein Turm von Erweiterungen wobei
	\begin{enumerate}[(1)]
		\item $K_{t} / k$ normale Erweiterung
		\item $K_{i}$ ist eine reine Erweiterung von Primzahlen $p_{i}$ mit $1 \leq i \leq t$.
		\item $k$ enthält alle $p_{i}$-ten Wurzeln von $1$, $1 \leq i \leq t$.
	\end{enumerate}
	Dann ist $\gal(K_{t} / k)$ auflösbar.
\end{lemma}





















