%! TEX root=../algebra.tex
\graphicspath{{Images/}}

\chapter{Gruppentheorie}
\section{Definition und Beispiele}

\begin{definition}
	Eine Menge $G$ gemeinsam mit einer Abbildung $\cdot : G \times G \to G$ heißt eine Gruppe falls folgende Axiome erfüllt sind:
	\begin{enumerate}[1)]
		\item Assoziativität: $\forall a,b \in G: (a \cdot  b) \cdot  c = a \cdot (b \cdot c)$
		\item Einheit: $\exists e \in G \forall a \in G: e \cdot a = a \cdot e = a$
		\item Inverse: $\forall a \in G \exists x \in G: a \cdot  x = x \cdot a = e$ (wobei $e$ wie in 2) ist)
	\end{enumerate}
\end{definition}

\begin{lemma}
	Sei $G$ eine Gruppe.  Die Einheit $e$ wie in 2) ist eindeutig bestimmt durch $e \cdot  a = a$
	für alle $a \in G$, oder  auch durch $e \cdot e = e$. Für jedes $a \in G$ ist die Inverse $x \in G$ durch
	$a \cdot x = e$ eindeutig bestimmt, wie schreiben $a^{-1} = x$.
	Insbesondere gilt $e^{-1} = e$, $(a^{-1})^{-1} = a$ und $(ab)^{-1} = b^{-1} a^{-1}$ für alle $a,b \in G$.
\end{lemma}

\begin{remark}
	Wir bezeichnen die Einheit auch als das Einselement und schreiben $e = e_{G} = 1 = 1_{G}$.
\end{remark}

\begin{proof}
	Angenommen $f \in G$ erfüllt $f \cdot a = a$ für alle $a \in G$ und $e \in G$ erfüllt Axiom 2).
	Dann gilt $f \stackrel{2)}{=} f \cdot e = e$.

	Angenommen $f \in G$ erfüllt $f \cdot f = f$. Wir multiplizieren mit $f^{-1}$ wie in Axiom 3) und erhalten
	\[
		f = f \cdot (\underbrace{f \cdot f^{-1}}_{e}) = (f \cdot f) \cdot f^{-1} = f \cdot f^{-1}= e
	.\] 
	Angenommen $a \cdot y = e$ und $x$ ist wie in Axiom 3). Dann gilt
	\[
		x \cdot (a \cdot y) = x \cdot e = x \Leftrightarrow (\underbrace{x\cdot a}_{e} ) \cdot y = x \Leftrightarrow x = y
	.\] 
\end{proof}

\begin{definition}
	Sei $G$ eine Gruppe und $a,b \in G$. Falls $ab = ba$ gilt, so sagen wir, dass $a$ und $b$ kommutieren.
	Falls alle Paare in $G$ \emph{kommutieren}, so heißt $G$ \emph{kommutativ} oder auch \emph{abelsch}.
\end{definition}

\begin{remark}
	Für abelsche Gruppen verwenden wir manchmal auch additive Notation $+ : G \times G \to G$.
\end{remark}

\begin{definition}
	Für eine Gruppe $G$ und $a \in G$ definiere wir die Potenzen von $a$ durch
	\[
	a^{k} := \begin{cases}
		\underbrace{a \cdot \ldots \cdot a}_{k-\text{fache}} &\text{für } k > 0\\
		e &\text{für } k=0\\
		\underbrace{a^{-1} \cdot \ldots \cdot ^{-1}}_{\abs{k}-\text{fache}} &\text{für } k < 0
	\end{cases}
	\qq{für alle} k \in Z
	.\] 
\end{definition}

\begin{lemma}[Potenzregel]
	\begin{enumerate}[a)]
		\item $a^{k} a^{l} = a^{k+l}$ für $k \in \Z$.
		\item $(a^{k})^{l} = a^{k l}$ für  $k \in \Z$.
		\item Falls $a,b \in G$ kommutieren so kommutieren auch $a^{k}$ und $b^{l}$ und es gilt $(ab)^{k} = a^{k} b^{k}$.
	\end{enumerate}
\end{lemma}

\begin{proof}
	Für $k,l \geq 0$ mittels Induktion nach $l$ :
	\begin{enumerate}
		\item IA: $a^{k} a^{0} = a^{k+0}$ \\
			IS: $a^{k} a^{l+1} = a^{k} a^{l} a = a^{k+l} a = a^{k+l+1}$ per rekursiver Definition
		\item IA: $(a^{k})^{0} = e = a^{k \cdot 0}$\\
			IS: $(a^{k})^{l+1} = (a^{k})^{l} a^{k} = a^{kl} a^{k} \stackrel{a)}{=} a^{k(l+1)}$ 
		\item IA: $a b^{0} = b^{0} a$ \\
			IS: $a b^{l+1} = a b^{l} b = b^{l} a b = b^{l} b a = b^{l+1} a$ also  $a$ kommutiert mit  $b^{l}$.\\
		IA: $(ab)^{0} = e = a^{0} b^{0}$\\ 
			IS: $(ab)^{k+1} = (ab)^{k} (ab) = a^{k} b^{k} a b = a^{k+1} b^{k+1}$
	\end{enumerate}
	Beweis für negative Potenzen analog.
\end{proof}

\begin{lemma}[Gleichungen und Kürzen]
	Für alle $a,b \in G$ existiert ein eindeutig bestimmtes $x \in G$ mit $ax = b$, nämlich $x = a^{-1} b$.
	Für alle $a,b,c \in G$ gilt $a=b \Leftrightarrow ac = bc \Leftrightarrow ca = cb$.
\end{lemma}

\begin{proof}
	Angenommen $ax=b$, dann gilt  $\underbrace{a^{-1} a}_{e} x = a^{-1} b \implies x = a^{-1} b$.
	Und in der Tat gilt $a (a^{-1} b) = b$. 

	$\implies$ trivial\\
	$\impliedby$: Angenommen $ac = bc$, dann gilt $(ac) c^{-1} = (bc) c^{-1} \implies ae = be \implies a = b$.
\end{proof}

\begin{definition}
	Angenommen $G_1, G_2$ sind Gruppen.
	Ein \emph{Homomorphismus} von $G_1$ nach $G_2$ ist eine Abbildung $\varphi: G_1 \to G_2$ mit
	$\varphi(ab) = \varphi(a) \varphi(b)$ für alle $a,b \in G$.
	Wir definieren den \emph{Kern} $\ker(\varphi) = \varphi^{-1} \{e_{G_2}\} = \{a \in G \mid \varphi(a) = e_{G_2}\} $
	und das \emph{Bild} $\Im(\varphi) = \varphi(G_1) = \{b \in G_2 \mid \exists a \in G \text{ mit } \varphi(a) = b\} $.
	Falls $\varphi$ bijektiv ist, so sprechen wir auch von einem \emph{Isomorphismus} der Gruppen
	und sagen $G_1$ und $G_2$ sind \emph{isomorph }.
\end{definition}

\begin{definition}
	Sei $G$ eine Gruppe. Eine Untergruppe von $G$ ist eine nichtleere Teilmenge $H \subseteq G$ mit $a b^{-1} \in H$ für alle $a,b \in H$. Wir schreiben $H < G$.
\end{definition}

\textbf{Übung.} Sei $G$ eine Gruppe und $H \subseteq G$. Äquivalent sind:
\begin{enumerate}[1)]
	\item $H$ ist eine Untergruppe
	\item $e \in H$, und $a,b \in H \implies ab \in H$ und $a^{-1} \in H$ 
	\item $H$ ist eine Gruppe und $\iota: H \to G$ ist ein Homomorphismus.
\end{enumerate}
Falls $\abs{H} < \infty$, so ist auch folgende Aussage mit obigen Aussagen äquivalent:
\begin{enumerate}[1)]
	\setcounter{enumi}{3}
	\item $H$ ist nichtleer, und $a,b \in H \implies ab \in H$.
\end{enumerate}

\begin{eg}
	Für einen Homomorphismus $\varphi: G_1 \to G_2$ ist $\ker(\varphi)$ eine Untergruppe von $G_1$
	und $\Im(\varphi)$ eine Untergruppe von $G_2$.
\end{eg}

\begin{eg}
	\begin{enumerate}
		\item $\{1\} $ 
		\item Addition in Ringen (und Körper) und Vektorräume.
		\item Die Gruppe $R^{\times}$ der Einheiten in einem Ring.
			Insbesondere $K^{\times} = K \setminus \{0\}$ für einen Körper.
			Also $\R^{\times} = \R \setminus \{0\}$ bzw $\C^{\times} = \C \setminus \{0\}$.
		\item Sei $M$ eine nichtleere Menge. Dann ist $\bij(M) = \{\varphi: M \to M \text{ bijektiv}\} $ eine Gruppe (bzgl. Verknüpfung der Abbildungen).
			Falls $M = \{1,\ldots,n\}$ für ein $n \geq 1$, so nennen wir $\sym_{n} = S_{n} = \bij(\{1,\ldots,b\})$ auch die \emph{symmetrische Gruppe}.
		\item Sei $M$ eine nichtleere Menge mit \enquote{einer Struktur}. Dann ist 
			\[
				\operatorname{Aut}(M) = \{\varphi: M \to M \text{ bijektiv \& \enquote{strukturerhaltend} }\} 
			\] 
			oft eine Gruppe.
			\begin{center}
				\begin{tabu} to \linewidth {X|X[2.5]}
					$M$ \& Struktur auf $M$ 			& $\aut(M)$\\ \hline
					$M$ \& ohne Struktur 				& $\bij(M)$\\
					$V$ Vektorraum über einem Körper 	& $\GL(V)$\\
				$K \supseteq \Q$ ein Körper 			& $\gal(K:\Q) = \{\varphi: K \to K : \Q\text{-linear, bijektiv und }\varphi(ab) = \varphi(a) \varphi(b)\} $ 
														(Galois-Gruppe von $K$)\\
					$G$ eine Gruppe 					& $\aut(G) = \{\varphi: G \to G \underbrace{\text{ Isomorphismus von $G$ nach $G$}}_{\text{Automorphismus von } G}\} $\\
					Affine reele Ebene 					& $\GL_2(\R) \ltimes \R^2$ \\
					Euklidische reelle Ebene  			& $O_2(\R) \ltimes \R^2$\\
					Sphärische Geometrie 				& $O_3(\R)$\\
					Hyperbolische Ebene		 			& $\SO_{2,1}(\R), P \GL_{2}(\R)$ ($P$ für Projektiv, also, dass man das Zentrum rausfaktorisiert)\\
					\qquad \qquad\quad \vdots \\
					Topologischer-Raum $X$ 				& $\text{Homöo}(X) = \{\varphi: X\to X \text{ bijektiv, stetig, } \varphi^{-1} \text{ stetig}\} $\\
					Mannigfaltigkeit $M$				& $\operatorname{Diffeo}^{\infty}(M) = \{\varphi: M \to M$ bijektiv, stetig, glatt und ebenso  $\varphi^{-1}\} $\\
					$M =$ regelmäßiges Polygon in $\R^2$ & Diedergruppe $D_{n} = \{$ lineare Abb. in $\GL_2(\R)$, die $M$ auf sich abbilden $\} $ \\
					$M = $ Platonische Körper im $\R^3$ \\
					$M =$ Zauberwürfel Rubik's Cube		& Bewegungen des Zauberwürfels
				\end{tabu}
			\end{center}
		\item Sei $K$ ein Körper. Dann ist
			\[
				\GL_{n}(K) = \{A \in \mat_{nn}(K) : A \text{ invertierbar}\} 
			\]
			ein Gruppe. Falls $V$ ein $n$-dimensionaler Vektorraum über $K$ ist, so ist $\GL(V)$ isomorph zu $\GL_{n}(K)$ 
			- dies ist mit der Auswahl einer Basis von $V$ gleichzusetzen.
			Des Weiteren ist 
			\[
				\det: \GL_{n}(K) \to K^{\times}
			.\] 
			ein Gruppenhomomorphismus und $\ker(\det) = \SL_{n}(K)$.
		\item .
			\begin{center} 
				\begin{tabu} to \linewidth {XX}
					$(0,\infty) < R^{\times}$ ist eine Untergruppe. &$S^{1} = \{z \in \C \mid \abs{z} = 1\} = \ker(\abs{\cdot}) < \C^{\times}$ ist eine Untergruppe.\\
					$\exp: \R \to \R^{\times }$ ist ein Homomorphismus. &$\exp: \C  \to \C^{\times}$ ist ein Homomorphismus. $\ker(\exp) = 2\pi i \Z$ 
				\end{tabu} 
			\end{center}
		\item $G_1 \times G_2$ ist eine Gruppe (komponenetenweisen Operationen) falls $G_1,G_2$ Gruppen sind.
	\end{enumerate}
\end{eg}

\begin{lemma}
	Sei $G$ eine Gruppe und $a \in G$. Dann definiert $k \in \Z \mapsto a^{k} \in G$ einen Gruppenhomomorphismus.
	Entweder ist $\varphi$ injektiv oder es gibt ein $n_0 > 0$ mit $\ker(\varphi) = (n_0) = \Z n_0$.
\end{lemma}

\begin{definition}
	Falls $\varphi$ wie im Lemma injektiv ist, so sagen wird, dass $a$ \emph{unendliche Ordnung} hat.
	Falls $\ker(\varphi) = (n_0)$ mit $n_0 > 0$ ist, so sagen wir, dass $a$ \emph{Ordnung $n_0$} hat.
\end{definition}

\begin{proof}
	$\varphi: n \mapsto a^{n}$ ist ein Homomorphismus wegen dem zweiten Lemma von heute.
	\begin{remark}
		$I = \ker(\varphi)$ ist ein Ideal in $\Z$ und eine Untergruppe.
		Angenommen  $k \in I$ und $n \in \Z$. Dann gilt $\varphi(n^{k}) = a^{nk} = (a^{k})^{n} = e$ und daher $nk \in I$.
		Entweder $I = (0)$ oder $I = (n_0)$ für $n_0 > 0$:\\
		$I = (0)$ dann ist $\varphi$ injektiv:
		Angenommen $\varphi(m) = \varphi(n) \Leftrightarrow \varphi(m-n) = e \Leftrightarrow m-n \in I = (0) \implies m=n$.
	\end{remark}
\end{proof}

\begin{eg}
	z.B. hat
	$\begin{pmatrix} 
		1 & 1\\ 0 &1
	\end{pmatrix} \in \GL_2(\R)$ unendliche Ordnung und  $\begin{pmatrix} 
		0 &-1\\ 1 & 0
	\end{pmatrix} \in \GL_2(\R)$ hat Ordnung $4$.
\end{eg}

\section{Konjugation}

\begin{lemma}
	Sei $G$ eine Grupee.
	\begin{enumerate}[a)]
		\item Für jedes $g \in G$ ist $\gamma_{g}: G \to G, x \mapsto g x g^{-1}$ ein Automorphismus von $G$, welche ein \emph{innerer Automorphismus} genannt wird.
		\item Die Abbildung $g \in G \mapsto \gamma_{g} \in \aut(G)$ ist ein Homomorphismus.
			Der Kern von $\Phi$ ist das Zentrum $Z_{G} = \{g \in G \mid gx = xg \forall x \in G\} $.
	\end{enumerate}
\end{lemma}

\begin{proof}
	Für $g,x,y \in G$ gilt 
	\[
		\gamma_{g}(x y) = g x y g^{-1} = g x g^{-1} g y g^{-1} = \gamma_{g}(x) \gamma_{g}(y)
	.\]
	Also ist $\gamma_{g}$ ein Homomorphismus $G \to G$.
	Für $g,h,x \in G$ gilt 
	\[
		\gamma_{g}(\gamma_{h}(x) = g (\gamma_{h}(x)) g^{-1} = g (h x h^{-1}) g = (gh) x (gh)^{-1} = \gamma_{gh}(x)
	.\]
	Insbesondere gilt
	\[
		\gamma_{g} \cdot \gamma_{g^{-1}}(x) = \gamma_{g g^{-1}}(x) = \gamma_{e}(x) = \id(x)
	.\] 
	und daher $\gamma_{g} \gamma_{g^{-1}} = \id = \gamma_{g^{-1}} \gamma_{g}$.
	Also ist $\gamma_{g}$ ein Automorphismus und a) ist bewiesen.

	Für b) haben wir bereits gezeigt, dass $\Phi: G \to  \aut(G)$ ein Homomorphismus ist:
	\[
		\Phi(gh) = \gamma_{gh} = \gamma_{g} \cdot \gamma_{h} = \Phi(g) \Phi(h)
	.\] 
	Des Weiteren gilt
	\[
		\ker(\Phi) = \{g \in G \mid \gamma_{g} = \id\}  = \{g \in G \mid \underbrace{g x g^{-1} = x}_{gx = xg} \text{ für alle } x \in G\} 
	.\] 
\end{proof}

\begin{definition}
	Sei $G$ ein Gruppe und $g \in G$. Dann ist die Menge der Fixpunkte $\gamma_{g}$ gleich dem Zentralisator von $g$ :
	\[
		\operatorname{Cent}_{g} = \{x \in G \mid gx = xg\} 
	.\] 
\end{definition}

\begin{definition}
	Sei $G$ eine Gruppe und $x,y \in G$.
	Wir sagen $x,y$ sind \emph{zueinander konjugiert}, falls es ein $g \in G$ mit $g x g^{-1} = y$.
\end{definition}

\begin{lemma}
	\enquote{Konjugiert sein} definiert eine Äquivalenzrelation auf jeder Gruppe.
\end{lemma}

\begin{proof}
	Übung
\end{proof}

\begin{eg}
	\begin{enumerate}[a)]
		\item Sei $G = \GL_{n}(\C)$. Zwei Matrizen $A,B$ sind konjugiert falls es ein $g \in \GL_{n}(\C)$ gibt mit $g A g^{-1} = B$.
			Dies gilt genau dann, wenn $A$ und $B$ dieselbe Jordan-Normalform hat.
		\item Sei $G = \on{U}_{n}(\C) = \{A \in \GL_{n}(\C) \mid A^{*} A = A A^{*} = I\} $.
			Jedes $g \in G$ ist mittels einem Element von $G$ diagonalisierbar.
			$\implies$ Konjugationsklassen für $G$ können wir durch Elemente von $(S^{1})^{n}$ modulo Vertauschung der Koordinaten beschreiben.
	\end{enumerate}
\end{eg}

Manchmal ist $G$ sehr kompliziert und unüberschaubar aber die Konjugationsklassen sind einfacher zu verstehen.

\begin{eg}
	$\sym_{n} = S_{n}$ hat $n! \approx \left( \frac{n}{e} \right)^{n} \sqrt{2\pi n} $ Elemente (Sterling-Formel).
	Die Anzahl der Konjugationsklassen ist hingegen ungefähr $\frac{1}{4 \sqrt{3} n} e^{2\pi \sqrt{\frac{n}{6}}}$ (Hardy-Ramanujan 1918).
\end{eg}

\begin{eg}
	\begin{enumerate}[1)]
		\item Das Zentrum von $S_{n}$ für $n \geq 3$ ist $\{1\}$. (Ü)
		\item Das Zentrum von $\GL_{n}(K)$ ist $\{A \in \GL_{n}(K) \mid  A \text{ ist Diagonal mit Diagonaleintrag } t \in K^{\times}\}$ :
			\begin{align*}
			\begin{pmatrix} 
				t &0\\ 0 &1
			\end{pmatrix} = \begin{pmatrix} 
				0 &1\\ 0&0
			\end{pmatrix} = \begin{pmatrix} 
				0 &t\\ 0 & 0
			\end{pmatrix} \\
			\begin{pmatrix} 
				0 &0\\ 0 & 0
			\end{pmatrix} = \begin{pmatrix} 
				t &0\\ 0 &1
			\end{pmatrix} = \begin{pmatrix} 
				0 &1\\ 0 &0
			\end{pmatrix} 
			\end{align*}
		\item Das Zentrum von $\SL_{n}(K)$ ist $\{A \in \GL_{n}(K) \mid  A \text{ ist Diagonal mit Diagonaleintrag } t \in K^{\times}, t^{n} = 1\}$
	\end{enumerate}
\end{eg}

\section{Untergruppen und Erzeuger}
\textbf{Wiederholung:}
$H\subseteq G$ nichtleer ist eine \emph{Untergruppe} ($H<G$ ) falls für alle $a,b \in H$ gilt $ab^{-1} \in H$.

\begin{eg}
	Sei $G = \Z$. Dann ist jede Untergruppe $H < \Z$ ein Ideal und damit von der Form $H = (n_0)$ für ein $n_0 \in \N$.
	Denn: Für $n \in H$ und $k \in \Z$ gilt 
	\[
		k \cdot n = \begin{cases}
			\underbrace{n + \ldots + n}_{k-\text{mal}} \in H &\text{für } k > 0\\
			0 \in H &\text{für } k = 0\\
			\underbrace{-n - \ldots -n}_{\abs{k}-\text{mal}} \in H &\text{für } k <0
		\end{cases}
	\]
\end{eg}

\begin{eg}
	Sei $n \geq 2$ eine natürliche Zahl.
	Dann definieren wir die \emph{Diedergruppe} $D_{2n}$ mittels $\zeta = e^{\frac{2 \pi i}{n}}$ und $\R$-lineare Transformationen auf $\C$ :
	\[
		D_{2n} = \underbrace{\{z \mapsto \zeta^{k} z \mid k = 0,1,\ldots,n-1\}}_{C_{n} \cong \sfrac{\Z}{(n)}} \cup 
		\{\underbrace{z \mapsto \zeta^{k} \overline{z}}_{\sigma_{k}} \mid k = 0,1,\ldots,n-1\} 
	.\] 
	und es gilt $\sigma_{k}(\sigma_{k}(z)) = \sigma_{k}(\zeta^{k} \overline{z}) = \zeta^{k} \overline{(\zeta^{k} \overline{z})} = z$, also definiert $\sigma_{k}$ eine Spiegelung 
	des regelmäßigen $n$-Ecks. $C_{n}$ definiert die Drehungen.

	\textbf{Untergruppen:} $\{\id\}, D_{2n}, C_{n}, \{\id, \sigma_{k}\}$ für $k = 0,\ldots,n-1$, für $k \mid n$ gibt es auch eine Untergruppe von $C_{n}$ isomorph zu $\sfrac{\Z}{(n)}$ 
	und von $D_{2n}$ isomorph zu $D_{2k}$. $D_{2\cdot 2}$ hat noch eine weitere Untergruppe.
\end{eg}

\begin{lemma}
	Eine Untergruppe von einer Untergruppe ist eine Untergruppe.
\end{lemma}

\begin{lemma}
	Sei $G$ eine Gruppe und $I$ eine Menge und $H_{i} < G$ für jedes $i \in I$.
	Dann ist $\bigcap_{i \in I} H_{i} < G$.
\end{lemma}

\begin{definition}
	Sei $G$ eine Gruppe und $X \subseteq G$ eine Teilmenge. Die Untergruppe, die von $X$ erzeugt wird ist definiert als
	\[
		\left< X \right> = \bigcap_{\substack{H < G\\ X \subseteq H}} H
	.\] 
	Wir nennen $X$ die \emph{Erzeugendenmenge} von $\left< X \right>$. Falls $\left< X \right> = G$ sagen wir, dass $G$ \emph{durch $X$ erzeugt} wird.
	Falls $X = \{g\}$ dann nennen wir $\left< X \right> = \left< g \right>$ die von $g$ erzeugte \emph{zyklische Untergruppe} von $G$. 
\end{definition}

\begin{lemma}
	Sei $G$ eine Gruppe und $X \subseteq G$. Dann ist $\left< X \right> = 
	\{x_{1}^{\epsilon_1}\ldots x_{n}^{\epsilon_{n}} \mid n \in \N, x_1,\ldots,x_{n} \in X, \epsilon_1,\ldots,\epsilon_{n} \in \{\pm 1\} \} $.
\end{lemma}

\begin{proof}
	Sei $H_0$ die Menge rechts im Lemma. Dann gilt $X \subseteq H_0$ und $H_0 < G$.
	Daher tritt $H_0$ als eine der Untergruppen in der Definition von $\left< X \right>$ auf und wir erhalten $\left< X \right> \subseteq H_0$.
	Falls $H < G$ und $X \subseteq H$, dann enthält $H$ auch jeden Ausdruck der Form $x_1^{\epsilon_1} \ldots x_{n}^{\epsilon_{n}}$.
	Daher gilt $H_0 \subseteq H$. Da dies für alle derartigen $H$'s gilt, folgt $H_0 \subseteq \left< X \right>$.
\end{proof}

\begin{lemma}
	Sei $G$ eine Gruppe und $a \in G$. Dann gilt $\left< a \right> \cong \sfrac{\Z}{(n_0)}$ für ein $n_0 \in \N$.
\end{lemma}

\begin{proof}
	Wir definieren $\varphi: n \in \Z \mapsto a^{n} \in G$. Dies ist ein Homomorphismus und $\ker(\varphi) = I = (n_0)$ für ein $n_0 \in \N$.
	Nun definieren wir $\Phi: \sfrac{\Z}{(n_0)} \to  \left< a \right>, k+(n_0) \mapsto a^{k}$.
	Dies ist wohldefiniert und injektiv wegen
	\[
		k+(n_0) = l+(n_0) \Leftrightarrow k-l \in (n_0) = \ker(\varphi) \Leftrightarrow a^{k-l} = e \Leftrightarrow a^{k} = a^{l}
	.\] 
\end{proof}

\begin{eg}
	$S_{n}$ (mit $n!$ Elementen verschiedenster Natur) ist durch zwei Elemente erzeugt:
	\begin{align*}
		\tau_{1,2} = \text{ Vertauschung von $1$ und $2$: } \begin{cases}
			1 \mapsto 2\\ 2 \mapsto 1 \\ 3 \mapsto 3\\ \quad \vdots\\ n \mapsto n
		\end{cases} (\text{Ordnung } 2)\\
		\sigma = \text{zyklische Vertauschung aller Zahlen: } \begin{cases}
			1 \mapsto 2\\ 2 \mapsto 3\\ 3 \mapsto 4 \\ \quad\vdots \\ n\mapsto 1
		\end{cases} (\text{Ordnung } n)
	\end{align*}
	Zum Beispiel $\sigma \tau_{1,2} \sigma^{-1} = \begin{cases}
		1 \mapsto n \mapsto  n \mapsto  1\\
		2  \mapsto 1 \mapsto 2 \mapsto 3\\
		2 \mapsto 2 \mapsto 1 \mapsto 2\\
		\qquad\quad\vdots
	\end{cases} = \tau_{2,3}$. Alle $\tau_{i,i+1} \in \left< \tau_{1,2}, \sigma \right>$.
	Diese Vertauschungen erzeugen ganz $S_{n}$.
	Sei $\rho \in S_{n}$ beliebig. Durch Linksmultiplikation von $\rho$ mit Vertauschungen $\tau_{i,i+1}$ können wir $\rho$ schrittweise vereinfachen
	und erhalten nach endlich vielen Schritten die Identität $\tau_{i_{k},i_{k}+1} \ldots \tau_{i_{n}, i_{n} + 1} \rho = \id$.
\end{eg}

\begin{remark}
	Es gibt keinen \enquote{Basis- oder Diemensionsbegriff}:
	Denn is $S_{6}$ gibt es eine Untergruppe, die von $3$ oder mehr Elementen erzeugt wird, aber nicht von weniger:
	\[
	H = \left< \tau_{1,2}, \tau_{3,4}, \tau_{5,6} \right> \cong \F_{2}^{3}
	.\] 
\end{remark}

\begin{definition}
	Sei $G$ eine Gruppe. Der \emph{Kommutator} von $a,b \in G$ ist 
	\[
		[a,b] = ab a^{-1} b^{-1}
	.\]
	Die \emph{Kommutatorgruppe} ist
	\[
		[G,G] = \left< [a,b]: a,b \in G \right>
	.\] 
\end{definition}

\section{Nebenklassen und Quotienten}

\begin{definition}
	Sei $G$ eine Gruppe und $H < G$.
	Wir definieren zwei Relationen auf $G$ 
	\[
		a \sim_{H} b \Leftrightarrow b^{-1} a \in H \qquad a \prescript{}{H}{\sim} b \Leftrightarrow b a^{-1} \in H
	.\] 
	Wir nennen die Menge $a H = \{a h \mid h \in H\} $ die Linksnebenklasse mit Linksrepräsentanten $a$ und schreiben auch
	\[
	\sfrac{G}{H} = \{a H \mid a \in G\} 
	.\] 
	Außerdem nennen wir die Menge $H a = \{h a \mid h \in H\} $ die Rechtsnebenklasse mit Rechtsrepräsentanten $a$ und schreiben 
	\[
	\sfrac{H}{G} = \{H a \mid a \in G\} 
	.\] 
\end{definition}

\begin{lemma}
	Sei $G$ eine Gruppe und $H < G$. Dann ist $\sim_{H}$ eine Äquivalentrelation und $[ a ]_{\sim_{H}}$ und $G / H$ ist der Quotient von $G$ bzgl. $\sim_{H}$.
	Dies gilt analog für $\prescript{}{H}{\sim}$
\end{lemma}

\begin{proof}
	\begin{itemize}
		\item $a \sim_{H} a$ denn $a^{-1} a = e \in H$.
		\item $a \sim_{H} b \implies b^{-1} a \in H \implies (b^{-1} a)^{-1} = a^{-1} b \in H \implies b \sim_{H} a$.
		\item $a \sim_{H} b $ und $b \sim_{H} c \implies b^{-1} a, c^{-1} b \in H \implies c^{-1} a = (c^{-1} b) (b^{-1} a) \in H \implies a \sim_{H} c$.
	\end{itemize}
	Also ist $\sim_{H}$ eine Äquivalenzrelation. Des Weiteren gilt:
	\[
		[ a ]_{\sim_{H}} = \{b \mid b \sim_{H} a \}  = \{b \mid b^{-1} a \in H\} = a H \quad (b^{-1} a = h \in H \implies a = b h \text{ für } h \in H)
	.\] 
\end{proof}

\begin{eg}	
	$S_{3} > H = \left< \tau_{12} \right> = \{e, \tau_{12}\} $. Sei $\sigma$ die zyklische Vertauschung von  $1,2,3$.
	Dann ist  $\sigma H \neq H \sigma$, da $\{\sigma, \sigma \tau_{12}\} \neq \{\sigma, \tau_{12} \sigma\} $.
	\[
	\sigma \tau_{12}: 1 \to 2 \to 3 \qquad \tau_{12} \sigma : 1 \to 2 \to 1
	.\] 
\end{eg}

\begin{theorem}
	Sei $G$ eine Gruppe und $H < G$.
	\begin{enumerate}[(1)]
		\item $\sfrac{G}{H}$ und $\sfrac{H}{G}$ sind (auf natürliche Weise) gleichmächtig.
		\item {[Lagrange]} Falls $\abs{G} < \infty$, dann gilt $\abs{G} = \abs{\sfrac{G}{H}} \cdot \abs{H}$.
			Insbesondere gilt $\abs{H}$ ist ein Teiler von $\abs{G}$.
	\end{enumerate}
\end{theorem}

\begin{definition}
	Die Kardinalität von $G$ wird auch die \emph{Ordnung} von $G$ genannt.
	Die Kardinalität von $\sfrac{G}{H}$ wird der \emph{Index $[G:H]$} von $H$ in $G$ genannt.
\end{definition}

\begin{proof}
	Wir definieren $\varphi: \sfrac{G}{H} \to \sfrac{H}{G}$ und $\psi: \sfrac{H}{G} \to \sfrac{G}{H}$ durch
	\[
		a H \mapsto (a H)^{-1} = \{g^{-1}: g \in a H\} = H a^{-1} \qquad H a \mapsto (Ha)^{-1} = a^{-1} H
	.\] 
	Wir sehen auch $\psi(\varphi(aH)) = \psi(H a^{-1}) = a H $ und analog $\psi \circ \varphi = \id$.
	Also folgt $\abs{\sfrac{H}{G}} = \abs{\sfrac{G}{H}}$ wie in $(1)$ behauptet.

	Für $(2)$ wählen wir aus jeder Linksnebenklasse $a H$ für $a \in G$ genau einen Linksrepräsentanten
	$x \in a H$ aus. Die Menge der ausgewählten Linksrepräsentanten bezeichnen wir mit $X$. Es gilt  $\abs{\sfrac{G}{H}} = \abs{X}$.
	\begin{claim}
		\[
			\abs{G} \stackrel{!}{=} \abs{X \times  H} = \abs{X} \abs{H} = \abs{\sfrac{G}{H}} \cdot \abs{H}
		\]
		$\psi: X \times  H \to G$ mit $(x,h) \mapsto  x h $
	\end{claim}
	$\psi$ ist \emph{surjektiv}: Sei $g \in G$, dann ist $gH \in \sfrac{G}{H}$ und wir haben in der Konstruktion von $X$ aus $g H$ ein $x \in X \cap g H$ 
	ausgewählt. Insebsondere gibt es ein $h \in H$ (weil $g \sim_{H} x$) mit $g = xh = \psi(x,h)$. Also ist $\psi$ surjektiv.

	$\psi$ ist \emph{injektiv}: Angenommen $\psi(x_1,h_1) = \psi(x_2,h_2)$ also $x_1 h_1 = x_2 h_2$ für $(x_1,h_1),(x_2,h_2) \in X \times G$.
	Insbesondere gilt $x_1 H = x_2 H$ in der Konstruktion von $X$ nur einen Linksrepräsentanten ausgewählt haben,
	gilt also $x_1 = x_2$. Daher gilt $x_1 h_1 = x_1 h_2$ und auch $h_1 = h_2$. 
	Wir haben also $(x_1,h_1) = (x_2,h_2)$ überprüft. Da dies für alle Paare $(x_1,h_1),(x_2,h_2)$ gilt, ist also $\psi$ injektiv.
\end{proof}

\begin{corollary}
	Sei $G$ eine endliche Gruppe und $g \in G$. Dann teilt die Ordnung von $g$ die Ordnung von $G$.
	Des Weiteren gilt $g^{\abs{G}} = e$.
\end{corollary}

\begin{proof}
	Sei $m = \abs{G}$ und $n = \abs{\left< g \right>} =$ Ordnung von $g$.
	Dann gilt $n \mid m$ wegen des Satzes von Lagrange.
	Sei $k = \frac{m}{n}$. Dan gilt
	\[
		g^{\abs{G}} = g^{m} = g^{nk} = (g^{n})^{k} = e^{k} = e
	.\] 
\end{proof}

\begin{corollary}
	In $\F_{p} = \sfrac{\Z}{(p)}$ gilt $a^{p-1} = \begin{cases}
		0 &a = 0\\
		1 &\text{für alle } a \in \F_{p}^{\times}
	\end{cases}$
\end{corollary}

\begin{proof}
	$G = \F_{p}^{\times }$ hat Ordnung $p-1$.
\end{proof}

\begin{corollary}[Erste Klassifikation von Gruppen]
	Sei $G$ eine endliche Gruppe und $\abs{G} = p \in \N$ prim. Dann ist $G$ isomorph zu $\sfrac{\Z}{(p)}$.
\end{corollary}

\begin{proof}
	Sei $g \in G \setminus \{e\} $. Dann ist $n = \operatorname{Ord}(g) = \left< g \right> = 1$ und ein Teiler von $p$.
	Also ist $n = p$ und $\left< g \right> = G$.
\end{proof}

$\implies$ Es gibt bis auf Isomorphie nur eine Gruppe der Ordnung $2,3,5,7,\ldots$.

Im Allgemeinen haben $\sfrac{G}{H}$ und $\sfrac{H}{G}$ keine natürliche Gruppenstruktur.

\begin{theorem}
	Sei $G$ eine Gruppe und $H < G$. Die folgenden Bedingungen sind äquivalent
	\begin{enumerate}[(1)]
		\item Für alle $x \in G$ ist $x H = H x$.
		\item Für alle  $x \in G$ ist $x H x^{-1} = H$.
		\item Es existiert eine gruppe $G_1$ und ein Gruppenhomomorphimus $\varphi: G \to G_1$ mit $H = \ker(\varphi)$.
		\item Für alle $x,y \in G$ gilt $(xH)(yH) = (xy) H$.
		\item $\sfrac{G}{H}$ ist (auf natürliche Weise) eine Gruppe so dass $\varphi: G \to \sfrac{G}{H}, g \mapsto g H$ ein Gruppenhomomorphismus ist.
	\end{enumerate}
\end{theorem}

\begin{proof}
	$(5) \implies (3)$ : $\ker(\varphi) = \{g \mid gH = e H\} = H$ \\
	$(3) \implies (2)$ : Sei $x \in G, h \in H = \ker(\varphi)$.
	Dann gilt $\varphi(x h x^{-1}) = \varphi(x) \underbrace{\varphi(h)}_{= e} \varphi(x)^{-1} = e$ also $x h x^{-1} \in H = \ker(\varphi)$.
	$\implies x H x^{-1} \subseteq H, x^{-1} H x \subseteq H \implies H \subseteq x H x^{-1}$. Somit folgt $x H x^{-1} = H$\\
	$(2) \Leftrightarrow (1)$ : Rechtsmultiplikation mit $x^{-1}$ oder $x$.\\
	$(1) \implies (4)$ : Seien $x,y \in G$. Dann gilt $(xH)(yH) = (Hx)(yH) = (Hxy)H = xy H H = xy H$.\\
	$(4) \implies (5)$ : Nach Annahme in $4$ ist die Abbildung 
	\[
		\sfrac{G}{H} \times \sfrac{G}{H} \to \sfrac{G}{H} \qquad (xh) \times (yH) \mapsto (xH)(yH) = (xy) H
	\]
	wohldefiniert. Nun folgen die Gruppenaxiome in $\sfrac{G}{H}$ direkt aus den Gruppenaxiome in $G$.
	\[
		((xH)(yH)) (zH) = ((xy)H)(zH) = ((xy) z) H = (x(yz)H) = \ldots = (xH)((yH)(zH))
	.\] 
	also ist $a$ in $\sfrac{G}{H}$ assoziativ.
	\begin{align*}
		&(xH)(eH) = (eH)(xH) = xH\\
		&(xH)(x^{-1} H) = (x^{-1} H)(xH) = e H
	.\end{align*}
\end{proof}

\begin{definition}
	Sei $G$ eine Gruppe und $H < G$.
	Wir sagen $H$ ist \emph{normal} in $G$ oder ein \emph{Normalteiler} von $G$ falls $H$ die Bedingungen in obigem Satz erfüllt.
	Wir schreiben in diesem Fall auch $H \lTri G$.
	Falls $H \lTri G$ so nennen wir $\sfrac{G}{H}$ die \emph{Faktorgruppe} von $G$ modulo $H$.
\end{definition}

\begin{definition}
	Sei $G \neq \{e\} $ eine Gruppe. Wir sagen $ G$ ist \emph{einfach} falls $G$ nur $\{e\} $ und $G$ als Normalteiler besitzt.
\end{definition}

\begin{eg}
	Eine abelsche Gruppe ist genau dann einfach wenn $G \cong \sfrac{\Z}{(p)}$ für eine Primzahl $p \in \N$.
\end{eg}

\begin{eg}
	Auf $S_{n}$ gibt es den Homomorphismus $\sgn: S_{n} \to \{\pm 1\} $. Der Kern $A_{n} = \ker(\sgn)$ wird die \emph{alternierende Gruppe} genannt.
	Für $n \geq 5$ ist $A_{n}$ eine nicht abelsche einfache Gruppe.
\end{eg}

\begin{theorem}[Erster Isomorphiesatz]
	Sei $\varphi: G \to H$ eine Homomorphismus zwischen zwei Gruppen $G$ und $H$.
	Dann induziert $\varphi$ einen Isomorphismus $\abs{\varphi}: \sfrac{G}{\ker(\varphi)} \to  \Im(\varphi)$ so dass 
	folgendes Diagram komutiert
	\[
	\begin{tikzcd}
		G \arrow[d, "\pi"'] \arrow[r, "\varphi"]                 & H                                    \\
	\sfrac{G}{\ker(\varphi)} \arrow[r, "\overline{\varphi}"] & \Im(\varphi) < H \arrow[u, "\iota"']
		\end{tikzcd}
	\] 
	mit $\pi$ als der kanonischen Projektion und $\iota$ der Einbettung.
	Also gilt $\varphi = \iota \circ \overline{\varphi} \circ \pi$.
\end{theorem}

\begin{proof}
	Wir zeigen, dass $\overline{\varphi}(x \ker(\varphi)) = \varphi(x)$ auf $\sfrac{G}{\ker(\varphi)}$ wohldefiniert und injektiv ist:

	Seien $x,y \in G$, dann gilt 
	\[
		\varphi(x) = \varphi(y) \Leftrightarrow \varphi(y)^{-1} \varphi(x) = e \Leftrightarrow \varphi(y^{-1} x) = e \Leftrightarrow y^{-1} x \in \ker(\varphi) \Leftrightarrow
		x \ker(\varphi) = y \ker(\varphi)
	\]
	$\implies$ wohldefiniert. $\impliedby$ injektiv.

	Auch gilt $\Im(\overline{\varphi}) = \Im(\varphi)$ womit $\overline{\varphi}: \sfrac{G}{\ker(\varphi)} \to \Im(\varphi)$ ein Isomorphismus ist.
	Für $g \in G$ gilt $\iota(\overline{\varphi}(\pi(g))) = \iota(\overline{\varphi}(g \ker(\varphi))) = \iota(\varphi(g)) = \varphi(g)$.
\end{proof}

\begin{eg}
	Sei $p$ prim. Dann ist $\abs{\GL_{2}(\F_{p})} = (p^2 -1)(p^2 - p)$ und 
	$\det: \GL_{2}(\F_{p}) \to \F_{p}^{\times}$ ist surjektiv z.B. wegen $\det\begin{pmatrix} 
		t & 0\\ 0 & 1
	\end{pmatrix} = t$. 

	Weiters gilt $\SL_2(\F_{p}) = \ker(\det)$. Aus dem Satz von Lagrange und dem ersten Isomorphiesatz folgt
	\[
		\abs{\SL_2(\F_{p})} \cdot (p-1) = \abs{\GL_2(\F_{p})}
	\]
	wobei Index $= \abs{\sfrac{G}{\ker(\det)}} = \abs{\Im(\det)} = p-1$.
	\[
		\implies \abs{\SL_2(\F_{p})} = \frac{(p^2-1)(p^2-p)}{p-1} = p (p^2-1)
	.\] 
\end{eg}

\begin{corollary}[Zweiter Isomorphiesatz]
	Sei $G$ eine Gruppe, $H \lTri G$. und $K < G$.
	Dann gilt $K H = H K < G, H \lTri KH, H \cap K \lTri K $ und 
	\[
	\sfrac{K}{H \cap K} \cong \sfrac{KH}{H}
	.\] 
	mit $x H \cap K \leftrightarrow x H$ für $x \in K$
\end{corollary}

\begin{proof}
	Für $k \in K$ gilt $kH = Hk$. 
	Für die Vereinigung über alle $k \in K$ gilt daher $KH = HK$.
	Des Weiteren gilt für  $k_1, k_2 \in K, h_1,h_2 \in H$ dass
	\begin{align*}
		&(k_1 h_1) (k_2 h_2) \in \underbrace{KH}_{HK} KH = \underbrace{HK}_{KH} H = K\\
		&(k_1 h_1)^{-1} = h_1^{-1} k_1^{-1} \in KH = HK
	\end{align*}
	Folgt $KH < G$.

	Des Weiteren ist $H \lTri KH$ weil $H < KH$ und für $x \in KH \subseteq G$ gilt $xH = Hx$.

	Wir definieren den Gruppenhomomorphismus
	\[
		\begin{tikzcd}
			K \arrow[r, hook] \arrow[rr, "\varphi"', bend right] & KH \arrow[r] & \sfrac{KH}{H}
		\end{tikzcd}
	.\] 
	Es gilt $\ker(\varphi) = \{k \in K \mid \underbrace{k H = e H}_{k \in H}\} = K \cap H \lTri K$
	und $\sfrac{K}{K \cap H} \cong \Im(\varphi) = \{k H \mid k \in K\}  = \sfrac{KH}{K}$
\end{proof}

\textbf{Übung:}
 Das Produkt von zwei Untergruppen ist im Allgemeinen keine Untergruppen.
 Das Produkt von zwei normalen Untergruppen ist eine normale Untergruppe.

 \begin{corollary}[Dritter Isomorphiesatz]
 	Sei $G$ eine Gruppe, $H \lTri G$, $K \lTri G$ und $K < H$.
	Dann ist $\sfrac{H}{K} \lTri \sfrac{G}{K}$ und es gilt
	\[
	\sfrac{\sfrac{G}{K}}{\sfrac{H}{K}} \cong \sfrac{G}{H}
	\] 
	wobei $(xK) \, \sfrac{H}{K} \; \widehat{=} \; x H$ einander im Isomorphismus entsprechen.
 \end{corollary}

\begin{proof}
	Wir definieren $ \varphi: \sfrac{G}{K} \to  \sfrac{G}{H}, gK \mapsto gH$.
	Dies ist wohldefiniert, da $K \subseteq H$ und wir einfach $gK$ rechts mit $H$ multiplizieren:
	$\varphi(gK) = (gK)H = gH$.
	Da die Gruppenstrukturen in $\sfrac{G}{K}$ und $\sfrac{G}{H}$ durch Multiplikation der Repräsentanten definiert ist,
	ist $\varphi$ auch ein Gruppenhomomorphismus
	\[
		\varphi((g_1 K)(g_2 K)) = \varphi((g_1 g_2)K) = (g_1 g_2) H = (g_1 H) (g_2 H) = \varphi(g_1 K) \varphi(g_2 K)
	.\] 
	$\varphi$ ist surjektiv. Daher gilt $\sfrac{\sfrac{G}{K}}{\ker(\varphi)} \cong \sfrac{G}{H}$ und $\ker(\varphi) = \{gK \mid gH = e H\} = \{hK \mid h \in H\} = \sfrac{H}{K}$
\end{proof}

\begin{corollary}
	Sei $G$ eine Gruppe und $H \lTri G$.
	Für eine beliebige weitere Gruppe $K$ gibt es eine natürliche Bijektion zwischen 
	\[
		\hom(\sfrac{G}{H},K) = \{ \varphi: \sfrac{G}{H} \to K \text{ Homomorphismus}\} \qq{und} \{\varphi: \hom(G,K) \mid \varphi \vert_{H} \equiv e_{K}\} 
	.\] 
\end{corollary}

\begin{corollary}
	Sei $G$ eine Gruppe und $H \lTri G$.
	Dann sind die folgenden beiden Abbildungen invers zueinander:
	\[
		(K < G \text{ mit } H < K) \mapsto  \sfrac{K}{H} < \sfrac{G}{H} \qq{und}(\pi^{-1}(\overline{K}) < G \text{ mit } H < \pi^{-1}(\overline{K})) \mapsfrom \overline{K} < \sfrac{G}{H}
	.\] 
\end{corollary}

\begin{eg}
	\begin{itemize}
		\item $C_{n} \lTri D_{2n}$ denn für eine Relation $R \in C_{n}$ und eine Reflexion $T \in D_{2n}$ gilt
			\[
			T R T^{-1} = R^{-1} \in C_{n}
			.\]
			(und jede Untergruppe $H < C_{n}$ ist auch ein Normalteiler von $D_{2n}$ )
		\item Zentrum und Kommutatorgruppe sind immer normal.
		\item Affine Gruppe $G = \{\begin{pmatrix} 
					a &b \\ 0 & 1
		\end{pmatrix} \mid a \in K^{\times}, b \in K\} $ für einen Körper $K$.
		\[
		H_1 = \{\begin{pmatrix} 
				1 &b\\ 0 &1
		\end{pmatrix} : b \in K\} \lTri G \qquad H_2 = \{ \begin{pmatrix} 
				a &0\\ 0 &1
		\end{pmatrix} : a \in K^{\times}\} < G
		\]
		aber nicht normal wenn $\abs{K^{\times}} > 1$.
	\end{itemize}
\end{eg}

\textbf{Übung:}
Sei $G$ eine Gruppe und $H < G$ mit Index $2$.
Dann gilt $H \lTri G$.

\textbf{Übung:} Klassifizieren/Beschreiben Sie alle Gruppen der Ordnung $\leq 7$ / $\leq 8$ / $\leq 10$.

\section{Gruppenwirkungen}

\begin{definition}
	Sei $G$ eine Gruppe und $T$ eine Menge.
	Eine \emph{Gruppenwirkung} (Linkswirkung, Linksaktion) von $G$ auf $T$ ist eine Abbildung $\cdot: G \times T \to T, (g,t) \mapsto g \cdot t$, so dass
	\begin{itemize}
		\item $e\cdot t = t$ für $t \in T$ 
		\item $g_1 \cdot (g_2 \cdot t) = (g_1 g_2) \cdot t$ für $g_1,g_2 \in G$ und $t \in T$.
	\end{itemize}
	Wir sagen in diesem Fall auch kurz, dass $T$ eine \emph{$G$-Menge} ist.
\end{definition}

\begin{remark}
	Obige Definition können wir äquivalent auch in folgender Form formulieren:\\
	Es gibt einen Gruppenhomomorphismus $\alpha: G \to \bij(T), g \in G \mapsto \alpha_{g}$.

	Der Zusammenhang zur obigen Definition ergibt sich durch die Formel $\alpha_{g}(t) = g \cdot t$
\end{remark}

\begin{eg}
	\begin{enumerate}[1)]
		\setcounter{enumi}{-1}
		\item Sei $T$ eine Menge und $G$ eine Gruppe. Dann ist die triviale Gruppenwirkung $g \cdot t = t$ für alle $g \in G, t \in T$ eine Gruppenwirkung.
		\item $G = S_{n}$ wirkt auf $T = \{1,\ldots,n\} $ durch $\sigma \cdot t = \sigma(t)$ für $\sigma \in S_{n}, t \in \{1,\ldots,n\} $ 
		\item $G = \GL(V)$ wirkt auf $V$, ein Vektorraum über  $K$ durch
			$\varphi \cdot v = \varphi(v)$ für $\varphi \in \GL(V)$ und $v \in V$.
		\item Sei $G$ eine Gruppe und $H < G$. Wir definieren $T =  \sfrac{G}{H}$ und $g \cdot (xH) = gxH$ für $g \in G$ und $xH \in \sfrac{G}{H}$.
			Dies definiert eine Gruppenwirkung: 
			 \begin{itemize}
				 \item $e \cdot (xH) = e x H = x H$ für $x H \in \sfrac{G}{H}$ 
				 \item $g_1 \cdot (g_2 \cdot (xH)) = g_1 \cdot (g_2 x H) = (g_1 g_2) x H = (g_1 g_2) \cdot (xH)$ für $g_1 g_2, x \in G$.
			\end{itemize}
			\begin{attention}
				Wir können auch eine Gruppenwirkung auf $\sfrac{H}{G}$ definieren,
				müssen dies aber mittels der Formel $g \cdot (Hx) = H x g^{-1}$ machen.
			\end{attention}
		\item Sei $G$ eine Gruppe und $T = G$. Dann können wir Konjugation als eine Gruppenwirkung betrachten:
			$g \cdot x = g x g^{-1}$ für $g \in G$ und $x \in T = G$.
		\item Sei $G$ eine Gruppe und $T = \mathcal{P}(G) = \{A \subseteq G\} $. Für $g \in G$ und $A \in T$ definieren wir
			$g \cdot A = g A = \{g a | a \in A\} $.
		\item Sei $G$ eine Gruppe und $T = \{H < G\} $. Für $g \in G$ und $H \in T$ definieren wir
			$g \cdot H = g H g^{-1}$.
	\end{enumerate}
\end{eg}

\begin{definition}
	Sei $G$ eine Gruppe und $T$ eine $G$-Menge.
	\begin{itemize}
		\item $S \subseteq T$ heißt \emph{invariant} falls $g \cdot S = S$ für alle $g \in G$.
		\item $t_0 \in T$ heißt \emph{Fixpunkt} falls $g \cdot t_0 = t_0$ für alle $g \in G$.
			Die Menge der Fixpunkte wird mit $\operatorname{Fix_{G}}(T) = \{t_0 \in T \mid t_0 \text{ ist ein Fixpunkt}\} $ bezeichnet.
		\item Für $t_0 \in T$ wird $G \cdot t_0 = \{g \cdot t_0 : g \in G\}$ als die \emph{Bahn ($G$-Bahn)} bezeichnet.
		\item Für $t_0 \in T$ heißt $\operatorname{Stab}_{G}(t_0) = \{g \in G \mid g \cdot t_0 = t_0\}$ der \emph{Stabilisator von $t_0$}.
		\item Falls $g \in G \mapsto \alpha_{g} \in \bij(T)$ wie in obiger Bemerkung injektiv ist, so heißt die Gruppenwirkung \emph{treu}.
		\item Die Gruppenwirkung heißt \emph{transitiv} falls es zu jedem Paar $t_1,t_2 \in T$ ein $g \in G$ mit $g \cdot  t_1 = t_2$ gibt.
			Die Gruppenwirkung heißt \emph{scharf transitiv} falls es zu jedem Paar $t_1,t_2 \in T$ genau ein $g \in G$ mit $g \cdot t_1 = t_2$ gibt.
		\item Die Menge der $G$-Bahnen wird mit $G \setminus T = \{G \cdot t_0 \mid t_0 \in T\} $ bezeichnet.
	\end{itemize}
\end{definition}

%TODO check all sfracs before for right orientation and if that is even important

\begin{lemma}
	Sei $G$ eine Gruppe und $T$ eine $G$-Menge.
	Dann definiert $t_1 \sim_{G} t_2 \Leftrightarrow \exists g \in G$ mit $g \cdot t_1 = t_2$ eine Äquivalenzrelation auf $T$. Die Bahnen sind genau die Äquivalenzklassen
	und $\sfrac{G}{\sim_{G}} = G \setminus T$ ist der Quotientenraum.
\end{lemma}

\begin{proof}
	\begin{itemize}
		\item Reflexivität: $t \sim t$ da $\underbrace{e}_{\in G} \cdot  t = t$.
		\item Symmetrie: Angenommen $t_1 \sim t_2$, dann exisitert ein $g \in G$ mit $g \cdot t_1 = t_2$.
			Wir wenden auf diesen Punkt $g^{-1}$ an und erhalten $g^{-1} \cdot (g \cdot t_1) = g^{-1} \cdot t_2$ und
			$t_1 = e \cdot t_1 = g^{-1} t_2$ und damit $t_2 \sim t_1$.
		\item Transitivität: Angenommen $t_1 \sim t_2, t_2 \sim t_3$ : Dann existieren $g_1, g_2 \in G$ mit $g_1 \cdot t_1 = t_2$, $g_2 \cdot t_2 = t_3$.
			$(\underbrace{g_2 g_1}_{\in G}) \cdot t_1 = g_2 \cdot (\underbrace{g_1 t_1}_{t_2}) = t_3 \implies t_1 \sim t_3$.
	\end{itemize}
	Des Weiteren $[ t ]_{\sim_{G}} = \{t_2 \sim t\} = \{g \cdot t : g \in G\} = G \cdot t$ und $\sfrac{T}{\sim} = \{[ t ]_{\sim} : t \in T\} = G \setminus T$.
\end{proof}

\begin{definition}
	Sei $G$ eine Gruppe und $T_1, T_2$ zwei $G$-Mengen.
	Ein $G$-Morphismus von $T_1$ nach $T_2$ ist eine Abbildung $f: T_1 \to T_2$ mit
	\[
		f(g \underbrace{\cdot }_{\text{in } T_1} t) = g \underbrace{\cdot}_{\text{in } T_2} f(t)
	\]
	für alle $t \in T_1$ und $g \in G$.
	$g$ ist ein \emph{$G$-Isomorphismus} falls $f$ zusätzlich bijektiv ist.
\end{definition}

\begin{theorem}[Satz (über Bahnen und Stabilisator)]
	Sei $G$ eine Gruppe und $T$ eine $G$-Menge.
	Sei $t_0 \in T$, $T_0 = G \cdot t_0$ und $H = \on{Stab}_{G}(t_0)$.
	Dann ist $H < G$, $T_0$ ist invariant und 
	\[
		f: \sfrac{G}{H} \to T_0, gH \mapsto g\cdot t_0
	\]
	ist ein wohldefinierter $G$-Isomorphismus.
	In diesem Satz ist also die Bahn isomorph zu $G$ modulo Stabilisator.
\end{theorem}

\begin{proof}
	Seien $h_1,h_2 \in H = \on{Stab}_{G}(t_0)$. Dann gilt $(h_1 h_2) \cdot t_0 = h_1 \cdot (\underbrace{h_2 \cdot t_0}_{t_0}) = t_0$ und $h_1 h_2 \in H$.
	Außerdem $h_1 \cdot t_0 = t_0 \implies t_0 = h_1^{-1} \cdot t_0$ und $h_1^{-1} \in H$. Folgt $H < G$ (da auch $e \in H$ ).
	
	Angenommen $g \in G$ und $g' \cdot t_0 \in T_0 = G \cdot t_0$.
	Dann ist $g \cdot (g' \cdot t_0) = (g g') \cdot t_0 \in T_0 = G \cdot t_0$.

	Angenommen $g_1, g_2 \in G$. Dann gilt $g_1 \cdot t_0 = g_2 \cdot t_0 \Leftrightarrow (g_2^{-1} g_1) \cdot t_0 = t_0 \Leftrightarrow g_2^{-1} g_1 \in H \Leftrightarrow g_1H = g_2 H$.
	Dies zeigt ($\impliedby$), dass $f$ wohldefiniert ist und ($\implies$) injektiv ist.

	$T_0 = G \cdot t_0 = \{g \cdot t_0\}  = f(G)$ und $f: \sfrac{G}{H} \to T_0$ ist surjektiv.

	Sei nun $g_1,g_2 \in G$. Dann gilt
	\[
		f(g_1 \cdot (g_2 H)) = f((g_1 g_2) H) = (g_1 g_2) \cdot t_0 = g_1 \cdot (g_2 \cdot to) = g_1 \cdot f(g_2 H)
	.\]
	Also ist $f$ ein $G$-Isomorphismus.
\end{proof}

\begin{corollary}
	Sei $G$ eine Gruppe und $T$ eine $G$-Menge. Falls $\abs{G} < \infty$, dann gilt 
	\[
		\abs{G} = \abs{G \cdot t_0} \cdot \abs{\on{Stab}_{G}(t_0)}
	\]
\end{corollary}

\begin{proof}
	Nach dem Satz gilt $G \cdot t_0 \cong \sfrac{G}{\on{Stab}_{G}(t_0)}$, d.h. $\abs{G \cdot t_0} = [G : \on{Stab}_{G}(t_0)]$
	und das Korollar folgt aus dem Satz von Lagrange
\end{proof}

\begin{corollary}
	Sei $G$ eine Gruppe und $T$ eine endliche $G$-Menge. Dann gilt
	\[
		\abs{T} = \abs{\on{Fix}_{G}(T)} + \sum_{\abs{G\cdot t} > 1} [G: \on{Stab}_{G}(t)]
	,\] 
	also die summe über die nicht trivialen Bahnen.
\end{corollary}

\begin{proof}
	Nach einem Lemma vom letzten Mal ist die Menge der Bahnen eine Partition von $T$.
	\[
		T = \bigsqcup_{\text{alle Bahnen}} G \cdot t = \on{Fix}_{G}(T) \sqcup \bigsqcup_{\abs{G \cdot t} > 1} G \cdot t 
	.\] 
	Des Weiteren gilt für eine Bahn $\abs{G \cdot t} = [G : \on{Stab}_{G}(t)]$ womit das Korollar folgt.
\end{proof}

\begin{theorem}[Cayley]
	Sei $G$ eine endliche Gruppe.
	Dann ist $G$ isomorph zu einer Untergruppe einer symmetrischen Gruppe $S_{n}$ für $n \in \N$.
\end{theorem}

\begin{proof}
	Sei $T = G$ und $g_1 \cdot g_2$ für $g_1 \in G$ und $g_2 \in T = G$ durch Gruppenmultiplikation definiert.
	Äquivalent dazu definiert dies einen Homomorphismus $\alpha: G \to \bij(G), g \mapsto (\alpha_{g_{1}}: g_2 \mapsto  g_1g_2)$.
	Es gilt
	 \[
		 \ker(\alpha) = \{g \in G \mid \alpha_{g} = \id \} \subseteq \{g \in G \mid \alpha_{g}(e) = e\} = \{g \in G \mid g e = e\} = \{e\} 
	\] 
	also ist $\alpha$ injektiv. Nach Annahme ist $\abs{G} = n \in \N$  und $\bij(G) \cong S_{n}$.
\end{proof}

\begin{remark}
	Falls $H < G$ mit endlichem Index, so gibt es einen Homomorphismus $\alpha: G \to S_{n}$ mit $n = [G:H]$ und 
	$\ker(\alpha) < H$.
\end{remark}

\section{Nilpotente und auflösbare Gruppen}

\begin{definition}
	Sei $G$ eine Gruppe. Wir sagen $G$ ist \emph{nilpotent mit Nilpotenzgrad $1$} falls $G$ abelsch ist.
	Wir sagen $G$ ist nilpotent mit \emph{Nilpotenzgrad $n+1$} (für $n \in \N_{\geq 1}$) falls
	$\sfrac{G}{Z_{G}}$ nilpotent mit Nilpotenzgrad $n$ ist.

	Wir sagen $G$ ist \emph{nilpotent} falls es ein $n \in \N$ gibt so dass $G$ nilpotent mit Nilpotenzgrad $n$ ist.
\end{definition}

\begin{eg}
	Sei $R$ ein Ring. dann ist die Heisenberggruppe
	\[
	H_{R} = \left\{ \begin{pmatrix} 
			1 & a & b\\ 0 & 1 & c\\ 0 & 0 & 1
	\end{pmatrix} \mid a,b,c \in R \right\} 
	\] 
	nilpotent mit Nilpotenzgrad $2$.
	Hierfür muss man zeugen:
	\[
	Z_{H_{R}} = \left\{ \begin{pmatrix} 
			1 & 0 & b\\ 0 & 1 & 0\\ 0 & 0 &1
	\end{pmatrix} \mid b \in R \right\} 
	\] 
	und $\sfrac{H_{R}}{Z_{H_{R}}} \cong R^2$.
\end{eg}

\begin{definition}
	Sei $G$ eine Gruppe und $p \in \N$ eine Primzahl.
	Wir sagen $G$ ist eine $p$-Gruppe falls $\abs{G} = p^{k}$ für ein $k \in \N$.
\end{definition}

\begin{lemma}[Fixpunkte von $p$-Gruppen]
	Sei $p \in \N$ eine Primzahl und $G$ eine $p$-Gruppe.
	Sei $T$ eine $G$-Menge. Dann gilt $\abs{\on{Fix}_{G}(T)} \equiv \abs{T} \mod p$.
\end{lemma}

\begin{proof}
	Auf Grund des Korollars vom Anfang der Stunde wirssen wir
	\[
		\abs{T} = \abs{\on{Fix}_{G}(T)} + \sum_{\abs{G \cdot t} > 1} [G : \on{Stab}_{G}(t)]
	.\] 
	Da $\abs{G} = p^{k}$ ist, ist $[ G : \on{Stab}_{G}(t) ] = p^{l}$ für $l \geq 1$ wenn $t \not\in \on{Fix}_{G}(T)$.
	Daher gilt $p \mid \sum_{\abs{G \cdot t} > 1} [G : \on{Stab}_{G}(t)]$.
\end{proof}

\begin{theorem}
	Eine $p$-Gruppe ist nilpotent.
\end{theorem}

\begin{proof}
	Angenommen $p$ ist eine Primzahl und $G$ ist eine $p$-Gruppe.
	Wir definieren $T = G$ und machen $G$ zu einer $G$-Menge mittels Konjugation. Dann gilt
	\[
		\on{Fix}_{G}(T) = \{t \in G \mid g t g^{-1} = t \forall g \in G\} = Z_{G}
	.\] 
	Wegen obigem Lemma gilt also
	\[
		\abs{\on{Fix}_{G}(T)} = \abs{Z_{G}} = \equiv \abs{T} = \abs{G} = p^{k} = 0 \mod p
	.\] 
	Da $e \in Z_{G}$ gilt $\abs{T_{G}} \geq 1$, also $\abs{Za} \geq p$.
	Insbesondere ist $Z_{G}$ eine nicht trivialte Untergruppe, und
	$\sfrac{G}{Z_{G}}$ ist eine kleinere $p$-Gruppe.

	Falls $\abs{G} = p$ ist, so ist $G = Z_{G}$ zyklisch, also abelsch, also nilpotent mit Nilpotenzgrad $1$.
	
	Ansonsten:
	Mittels Induktion nach  $G$ dürfen wir bereits annehmen, dass $\sfrac{G}{Z_{G}}$ nilpotent mit Nilpotenzgrad $e$ ist.
	Demnach ist also $G$ nilpotent mit Nilpotentgrad $l+1$.
\end{proof}

\begin{corollary}
	Sei $p \in \N$ eine Primzahl und $G$ eine Gruppe mit $\abs{G} = p^2$.
	Dann ist $G$ abelsch.
\end{corollary}

\begin{proof}
	Aus dem Satz erhalten wir, dass $Z_{G}$ eine nichttriviale Untergruppe ($\abs{Z_{G}} > 1$) ist.
	Falls $Z_{G} = G$ ist, so folgt das Korollar.
	Angenommen dem ist nicht so, dann ist $\abs{Z_{G}} = p$.
	Dann ist aber $\sfrac{G}{Z_{G}}$ eine Gruppe der Ordnung $p$ und damit zyklisch.
	Also existiert ein $g \in G$ so dass $\sfrac{G}{Z_{G}} =  \left< g Z_{G} \right> = \{g^{k} Z_{p} \mid k = 0,\ldots,p-1\} $.
	Insbesondere gilt also
	\[
	G = \{g^{k} z \mid k = 0,\ldots,p-1, z \in Z_{G}\} 
	.\]
	Damit gilt aber für $g^{k_1} z_1, g^{k_2} z_2 \in G$, dass $g^{k_1} z_1 g^{k_2}z_2 = g^{k_1 + k_2} \underbrace{z_1 z_1}_{= z_2 z_1} = g^{k_2} z_2 g^{k_1} z_1$.
	Dies widerspricht der Annahme, dass $Z_{G} \subsetneq G$.
\end{proof}

\begin{definition}
	Sei $G$ eine Gruppe. Eine \emph{Subnormalreihe in $G$} ist eine Folge von Untergruppen so dass
	 \[
		 \{e\} = G_0 \lTri G_1 \lTri G_2 \lTri \ldots \lTri G_{n} = G
	\] 
	jede Untergruppe in der nächsten normal ist.
\end{definition}

\begin{definition}
	Sei $G$ eine Gruppe. Wir sagen $G$ ist \emph{auflösbar} falls es eine Subnormalreihe in $G$ (wie oben) gibt, so dass
	$\sfrac{G_{k+1}}{G_{k}}$ eine abelsche Gruppe (für $k = 0, \ldots, n-1$) ist.
\end{definition}

\begin{eg}
	\begin{enumerate}
		\item Diedergruppe $D_{2\cdot n}$ ist auflösbar.
		\item Affine Gruppe $A_{R} = \left\{\begin{pmatrix} 
					a & b\\ 0 & 1
		\end{pmatrix} \mid a \in R^{\times}, b \in R\right\} $ ist auflösbar und ist nicht nilpotent falls $\abs{R^{\times}} > 1$.
	\end{enumerate}
\end{eg}

\begin{proposition}
	Sei $G$ eine Gruppe. Dann ist $[G,G] = \left< \{ [a,b] \mid a,b \in G \} \right> \lTri G$, und $\sfrac{G}{[G,G]}$ ist abelsch.
	Falls $H$ eine ablesche Gruppe ist und $\varphi: G \to H$ ein Homomorphismus ist, so ist $\varphi([G,G]) = \{e_{H}\}$
	und $\varphi$ induziert einen Gruppenhomomorphismus $\overline{\varphi}: \sfrac{G}{[G,G]} \to H$.
	In diesem Sinne ist $\sfrac{G}{[G,G]}$ die größte abelsche Faktorgruppe von $G$.
\end{proposition}

\begin{proof}
	In der Tat ist $[G,G]$ eine charakteristische Untergruppe (siehe Übung) und damit auch normal.
	Seien $a,b \in G$. Dann gilt $[a [G,G], b [G,G]] = [a,b][G,G] = [G,G]$, aber dies bedeutet genau,
	dass die Elemente $a[G,G]$ und $b[G,G]$ in $\sfrac{G}{[G,G]}$ kommutieren.
	Da $a,b \in G$ beliebig waren, ist also $\sfrac{G}{[G,G]}$ abelsch.

	Sei nun $H$ abelsch und $\varphi: G \to H$ ein Homomorphismus. Für $a,b \in G$ gilt dann
	\[
		\varphi([a,b]) = [\varphi(a), \varphi(b)] = e_{H}
	\]
	Da dies für alle $a,b \in G$ gilt und $[G,G]$ von diesen Elementen erzeugt wird, erhalten wir daraus
	$\varphi([G,G]) = \{e_{H}\}$. Auf Grund eines Korollars zum ersten Isomorphiesatz gibt es damit
	einen Homomorphismus
	\[
		\begin{tikzcd}
			{\sfrac{G}{[G,G]}} \arrow[r, "\overline{\varphi}"] & H \\
			G \arrow[u, "\pi"] \arrow[ru, "\varphi"']          &  
		\end{tikzcd}
	\] 
	mit $\varphi = \overline{\varphi} \circ \pi$.
\end{proof}

\begin{proposition}
	Sei $G$ eine Gruppe. Dann ist $G$ auflösbar genau dann wenn die folgende induktiv definierten
	höheren Kommutatorgruppen nach endlich vielen Schritten die triviale Untergruppe $\{e\}$ erreicht:
	\begin{align*}
		&G^{(0)} = G \\
		&G^{(1)} = [G^{(0)}, G^{(0)}] \;(\text{Kommutatorgruppe})\\
		&G^{(2)} = [G^{(1)}, G^{(1)}] \;(\text{2. Kommutatorgruppe})\\
		&\;\;\vdots \\
		&G^{(n+1)} = [G^{(n)}, G^{(n)}]
	\end{align*}
\end{proposition}

\begin{proof}
	Angenommen $G^{(n+1)} = \{e\}$ für ein $n \in \N$. Dann ist
	\[
	\{e\} = G^{(n+1)} \lTri G^{(n)} \lTri G^{(n-1)} \lTri \ldots \lTri G^{(1)} \lTri G^{(0)}
	\] 
	eine Subnormalreihe für $G$ (mit umgekehrter Indesreihenfolge).
	Des Weiteren sind die Quotienten $\sfrac{G^{(k)}}{G^{(k+1)}}$ auf Grund der letzten Proposition abelsch.
	Also ist $G$ auflösbar.

	Sei nun umgekehrt $G$ auflösbar und $\{e\} = G_0 \lTri G_1 \lTri \ldots \lTri G_{n} = G$ eine Subnormalreihe mit abelschen Faktorgruppen.
	Da $\sfrac{G}{G_{n-1}} = \sfrac{G_{n}}{G_{n-1}}$ abelsch ist, gilt $[G,G] = G^{(1)} < G_{n-1}$.
	Mittels Induktion können wir analog $G^{(k)} < G_{n-k}$.
	Für $k = n$ erhalten wir also $G^{(n)} < G_0 = \{e\}$.
\end{proof}

\section{Satz von Sylow}
Für eine endliche Gruppe $G$ besagt der Satz von Lagrange, dass für $H < G$ sowohl die Ordnung $\abs{H}$ als auch der Index $[G:H]$ Teiler von  $\abs{G}$ sind.

\begin{theorem}[Sylow]
	Sei $G$ eine endliche Gruppe, $p \in \N$ prim und $n = \abs{G} = p^{k} m$ für $k \geq 1$ und $m$ teilerfremd zu $p$.
	\begin{enumerate}[1)]
		\item Es existiert eine maximale $p$-Untergruppe $H_{p}$ mit $\abs{H_{p}} = p^{k}$, welche \emph{Sylow $p$-Untergruppen} genannt werden.
		\item Falls $H < G$ eine $p$-Untergruppe ist, so existiert eine $p$-Sylow Untergruppe $H_{p}$ mit $H < H_{p}$.
		\item Je zwei Sylow $p$-Untergruppen sind konjugiert.
	\end{enumerate}
\end{theorem}

\begin{lemma}
	Sei $p \in \N$ prim, $n = p^{k}m$ mit $m$ teilerfremd zu $p$.
	Dann ist $\binom{n}{p^{k}}$ nicht durch $p$ teilbar.
\end{lemma}

\begin{proof}
	Sei $S = \sfrac{\Z}{(p^{k})} \times \{1,\ldots,m\}, G =  \sfrac{\Z}{(p^{k})}$, und definiere eine Wirkung von $G$ auf $S$ durch Addition 
	in der ersten Komponente: $g \cdot (a,j) = (a+g,j)$.
	Wir bemerken dass die $G$-Bahnen in $S$ genau die Mengen der Form $G \times \{i\}$ für ein $j \in \{1,\ldots,m\} $ sind.

	Wir definieren $T = \{A \subseteq S: \text{Teilmenge mit } \abs{A} = p^{k}\}$ und
	lassen $G$ auf $A \in T$ mittels $g \cdot A = \{g \cdot (a,j) \mid (a,j) \in A\} $ wirken.
	Damit ist $T $ eine $G$-Menge.

	Da $G$ eine $p$-Gruppe ist, können wir das frühere Lemma über $p$-Gruppen und Fixpunkte verwenden:
	\[
		\binom{n}{p^{k}} = \abs{T} = \abs{\on{Fix}_{G}(T)} = m \not\equiv 0 \mod p
	x.\] 
	nach Annahme im Lemma.
	\begin{align*}
		A \in \on{Fix}_{G}(T) &\Leftrightarrow A \subseteq S, \abs{A} = p^{k} \text{ und } g \cdot A = A \text{ für alle } g \in G\\
							  &\Leftrightarrow A \subseteq S, \abs{A} = p^{k} \text{ und } A \text{ ist eine Vereinigung von $G$-Bahnen} \\%TODO missing
							  &\Leftrightarrow A \subseteq S \text{ ist eine $G$-Bahn}\\
							  &\Leftrightarrow A = G \times \{j\} \text{ für ein } j \in \{1,\ldots,m\} 
	.\end{align*}
	\begin{eg}[Sylow-Untergruppe]
		Sei $G = \SL_{2}(\F_{p})$ mit Ordnung $p(p^2-1)$. Dann ist $H_{p} = \{ \begin{pmatrix} 
				1 & 0\\ 0 &1
		\end{pmatrix} \mid a \in \F_{p} \} \cong \F_{p}$ eine Sylow $p$-Untergruppe.
	\end{eg}
\end{proof}

\begin{proof}[Beweis von 1) Satz]
	Sei $T = \{ A \subseteq G: \abs{A} = p^{k}\} $. Dann ist $T$ eine $G$-Menge mittels Linksmultiplikation ($A \in T, g \in G, g \cdot A = \{g \cdot a \mid a \in A\}$)
	mit $\abs{T} = \binom{n}{p^{k}} \not\equiv 0 \mod p$.
	Auf Grund eines Korollars zu dem Satz über Bahn und Stabilisator gilt
	\[
		\abs{T} = \abs{\on{Fix}_{G}(T)} + \sum_{\abs{G \cdot t} > 1} [G: \on{Stab}_{G}(A)] \tag{$*$}
	.\] 
	Falls $n = p^{k}$ ist, so ist $H_{p} = G$ selbst die gesuchte Sylow $p$-Untergruppe.
	Ansonsten ist $p^{k} < n$ und es gibt keine $G$-invariante Teilmenge $A \subseteq G$ mit $\abs{A} = p^{k}$.
	Also ist $\on{Fix}_{G}(T) = \{\} $. Da $\abs{T} \not\equiv 0 \mod p$, folgt aus $(*)$, dass es ein $A \in T$ gibt, so dass $[G: \on{Stab}(A)] \not\equiv 0 \mod p$ ist.

	\begin{remark}
		$H_{p} = \on{Stab}_{G}(A_0)$ ist eine Sylow $p$-Untergruppe mit $\abs{H_{p}} = p^{k}$.
	\end{remark}
		Da $\abs{G} = \abs{H_{p}} [G: H_{p}] = p^{k} m $ und $p \not\in [G: H_{p}]$, folgt $p^{k} \mid \abs{H_{p}}$.
		Des Weiteren wissen wir $H_{p} \cdot A_0 = A_0$. Sei $a_0 \in A_0$ beliebig, dann gilt also
		$h \cdot a_0 \in H_{p} \cdot A_0 = A_0$ für alle $h \in H_{p}$.
		Also gilt $H_{p} = a_0 \subseteq A_0$ und $\abs{H_{p}} = \abs{H_{p} \cdot a_0} \leq \abs{A_0} = p^{k}$.
		Dies beweist die Behauptung und damit $1)$ im Satz.
\end{proof}

\begin{proof}[Beweis von 2)]
	Sei $H$ eine beliebige $p$-Untergruppe und $H_{p}$ eine beliebige Sylow $p$-Untergruppe.
	Wir definieren $T = \sfrac{G}{H_{p}}$ und lassen $H$ mittels linksmultiplikation auf $T$ wirken.
	Wegen dem Lemma über Fixpunkte von $p$-Gruppen gilt
	\[
		\abs{\on{Fix}_{H}(T)} \equiv \abs{T} = [G: H_{p}] = \frac{n}{p_{k}} = m \not\equiv 0 \mod p
	.\] 
	Insbesondere gibt es einen Fixpunkt $g H_{p} \in T$ (für die Wirkung von $H$ ).
	D.h. 
	\begin{align*}
		&hg H_{p} = g H_{p} \text{ für alle } h \in H\\
		&\implies hg \in g H_{p} \in \text{ für alle } h \in H\\
		&\implies h \in g H_{p} g^{-1} \text{ für alle } h \in H\\
		&\implies H < g H_{p} g^{-1} \text{ \& } g H_{p} g^{-1} \text{ ist eine Sylow $p$-Untergruppe}
	\end{align*}
\end{proof}

\begin{proof}[Beweis von 3)]
	Angenommen $H, H_{p}$ sind zwei Sylow $p$-Untergruppen. Dann ist $H$ eine $p$-Untergruppe und obiger Beweis von 2) zeigt, dass es ein $g \in G$ mit
	$H < g H_{p} g^{-1}$ gibt. Da aber $\abs{H} = \abs{H_{p}} = p^{k}$ ist, gilt $H = g H_{p} g^{-1}$.
\end{proof}

\section{Symmetrische und Alternierende Gruppen}

\begin{definition}
	Sei $n \geq 1$ natürlich, dann ist $S_{n} = \bij(\{1,\ldots,n\})$.
	Die Elemente von $S_{n}$ heißen \emph{Permutationen}.
\end{definition}

\begin{theorem}
	Sei $n \geq 1$. Auf $S_{n}$ gibt es einen Homomorphismus $\sgn: S_{n} \to \{\pm 1\}$, der jeder Permutation ein \emph{Vorzeichen} zuordnet
	und einer \emph{Vertauschung} $\tau_{ij}$ für $i \neq j$ das Vorzeichen $-1$ mit
	\[
		\tau_{ij}(k) = \begin{cases}
			i &\text{ für } k = j\\
			j &\text{ für } k = i\\
			k &\text{ sonst}
		\end{cases}
	.\] 
\end{theorem}

\begin{definition}
	$\sigma \in S_{n}$ heißt \emph{gerade} falls $\sgn(\sigma) = 1$, ungerade falls $\sgn(\sigma) = -1$.
	Die \emph{alternierende Gruppe} $A_{n} = \ker(\sgn)$ ist die Gruppe aller geraden Permutationen.
\end{definition}

\begin{proof}
	Siehe lineare Algebra. Alternative Beweis-Skizze:
	Für $F \in \Z[X_1,\ldots,X_{n}]$ definieren wir $\prescript{\sigma}{}{F} = F(X_{\sigma(1)},\ldots, X_{\sigma(n)})$.
	Dies definiert eine Gruppenwirkung von $S_{n}$ auf $\Z[X_1,\ldots,X_{n}]$ mittels Ringhomomorphismen.
	Wir definieren $P = \prod_{1 \leq i < j \leq n} (X_{i} - X_{j})$ und erhalten
	\[
		\prescript{\sigma}{}{P} = \prod_{1 \leq i < j \leq n} (X_{\sigma(i)} - X_{\sigma(j)} = \sgn(\sigma) P
	.\] 
	kann als Definition von $\sgn(\sigma)$ verwendet werden.
\end{proof}

\begin{notation}[für $\sigma \in S_{n}$]
	\[
	\sigma = \begin{pmatrix} 
		1 & 2 & \ldots & n\\
		\sigma(1) & \sigma(2) & \ldots & \sigma(n)
	\end{pmatrix} 
	.\] 
\end{notation}
Besser:
\begin{notation}[mittels Zyklen für $\sigma \in S_{n}$]
	Falls $\sigma = \id$ schreiben wir einfach $\sigma = \id$.
	Sei nun $\sigma \neq \id$ und $i_1 \in \{1,\ldots,n\} $ der erste Nichtfixpunkt (also $i_{1}$ minimal mit $\sigma(i_1) \neq i_1$ ).
	Wir bestimmen
	\[
		\sigma(i_1) , \sigma^2(i_1),\ldots, \sigma^{k_1}(i_1) = i_1 \qq{für $k_1 > 1$ minimal}
	.\] 
	Falls dies alle Nichtfixpunkte von $\sigma$ sind, so nennen wir $\sigma$ einen \emph{($k$-)Zyklus} und schreiben
	 \[
		 \sigma = (i_1,\sigma(i_1),\sigma^2(i_1),\ldots,\sigma^{k-1}(i_1))
	.\] 
	Falls nicht, so sei $i_2 > i_1$ der nächste Nichtfixpunkt (der noch nicht gefunden wurde) und bestimme
	\[
		i_2, \sigma(i_2),\ldots, \sigma^{k_2}(i_2) = i_2 \qq{für $k_2 > 1$ minimal}
	\]
	etc. Nach endlich vielen Schritten haben wir alle Nichtfixpunkte gefunden  und schreiben 
	 \[
		 \sigma = (i_1,\sigma(i_1),\ldots,\sigma^{k_1-1}(i_1))(i_2,\sigma(i_2),\ldots,\sigma^{k_2-1}(i_2))\ldots(i_{r},\sigma(i_{r}),\ldots,\sigma^{k_{r}-1}(i_{r}))
	.\] 
	In diesem Fall sagen wir auch, dass $\sigma$ \emph{Zyklentyp}(Struktur) $k_1,k_2,\ldots,k_{r}$ hat (wobei
	die Zahlen $k_1,\ldots,k_{r}$ auch in einer anderen Reihenfolge auftreten dürfen).
\end{notation}

\begin{eg}
	$n = 5$
	\[
		(3,2,5) = (2,5,3) = \sigma : \begin{cases}
			1 \mapsto  1\\ 2 \mapsto  5\\ 3 \mapsto  2\\ 4 \mapsto 4\\ 5 \mapsto 3
		\end{cases} = \begin{pmatrix} 
			1 & 2 & 3 & 4 & 5\\
			1 & 5 & 2 & 4 & 3
		\end{pmatrix} 
	.\] 
\end{eg}

\begin{proposition}
	Zwei Permutationen sind in $S_{n}$ genau dann konjugiert, falls sie dieselbe Zyklenstruktur haben.
\end{proposition}

\begin{eg}
	$n=5$ : $(2,5,3)$ und $(1,2,3)$ sind konjugiert.
	\begin{gather*}
		\sigma = \begin{pmatrix} 
			1 & 2 & 3 & 4 & 5\\
			2 & 5 & 3 & 1 & 4
		\end{pmatrix} \in S_{n}\\ 
		\sigma (1,2,3) \sigma^{-1} = \begin{cases}
			1 \mapsto  4 \mapsto 4 \mapsto 1\\
			2 \mapsto 1 \mapsto 2 \mapsto 5\\
			3 \mapsto 3 \mapsto 1 \mapsto 2\\
			4 \mapsto 5 \mapsto 5 \mapsto 4\\
			5 \mapsto 2 \mapsto 3 \mapsto 3
		\end{cases} = (2,5,3)
	\end{gather*}
\end{eg}

\begin{proof}[Beweis-Skizze (Seite 122)]
	Sei $\sigma \in S_{n}$ beliebig und $(i_1,\ldots,i_{k})$ ein Zyklus.
	Dann ist 
	\[
		\sigma(i_1,\ldots,i_{k}) \sigma^{-1} = (\sigma(i_1),\sigma(i_2),\ldots,\sigma(i_{k}))
	\]
	mit einer kleinen Rechnung wie im Beispiel.
	
	Dies gilt analog auch für Produkte von Zyklen für zwei Permutationen mit demselben Zyklentyp kann man $\sigma$ finden:
	\begin{align*}
		&\tau_{1} = (i_{1,1},\ldots,i_{1,k_1})(i_{2,1},\ldots,i_{2,k_2})\ldots (i_{r,1},\ldots,i_{r,k_{r}}) \overbrace{(i_{r+1}) \ldots (i_{s})}^{\text{Fixpunkte von } \tau_1}\\
		&\tau_{2} = (j_{1,1},\ldots,j_{1,k_1})(j_{2,1},\ldots,j_{2,k_2})\ldots (j_{r,1},\ldots,j_{r,k_{r}}) \underbrace{(j_{r+1}) \ldots (j_{s})}_{\text{Fixpunkte von } \tau_2}
	.\end{align*}
	wobei in beiden Zeilen jede Zahl von $1,\ldots,n$ einmal auftritt.
	Definiere man nun $\sigma$ mittels $\sigma(i_{*}) = j_{*}$.
\end{proof}

\begin{theorem}
	$A_{n}$ und $S_{n}$ sind auflösbar für $n \leq 4$.
	$A_{n}$ ist einfach für $n \geq 5$.
\end{theorem}

\begin{proof}[Beweis für $n \leq 4$ ]
	$A_1 \cong A_2 \cong \{e\}$\\
	$A_3 \cong \sfrac{\Z}{(3)}$ ist abelsch.\\
	$A_{4}: V_4 = \left< (1,2) (3,4), (1,3)(2,4) \right> = \{e, (1,2)(3,4), (1,3)(2,4), (1,4)(2,3)\} $ (Kleinsche Vierergruppe) 
	ist eine Untergruppe (wo jedes nichttriviale Element Ordnung $2$ hat).
	Dies bedarf einer kleinen Nebenrechnung,  z.B.
	\[
		(1,2)(3,4) \cdot (1,3)(2,4) = \begin{cases}
			1  \mapsto 3 \mapsto 4\\
			4 \mapsto 2 \mapsto 1\\
			2 \mapsto 4 \mapsto 3\\
			\quad\;\;\,\ldots
		\end{cases} = (1,4) (2,3)
	\] 
	insbesondere ist $V_4 \cong \sfrac{\Z}{(2)} \times \sfrac{\Z}{(2)}$.

	$V_4$ enthält neben $e$ genau die Elemente vom Zyklentyp $2,2$. $\implies V_4 \lTri A_4$.
	\[
		\abs{A_4} = \frac{4!}{2} = 4 \cdot 3 = 12, \abs{V_4} = 4 \implies \abs{\sfrac{A_4}{V_4}} = 3 \quad \& \quad \sfrac{A_4}{V_4} \cong \sfrac{\Z}{(3)}
	.\]
	Folgt $A_4$ ist auflösbar. Weiters ist $\sfrac{S_4}{A_4} \cong \sfrac{\Z}{(2)}$ also auch $S_4$ auflösbar.
\end{proof}

Für $n \geq 5$ wollen wir die Gruppenwirkung von $A_{n}$ auf $\{1,\ldots,n\}$ und folgende Lemmas verwenden.

\begin{lemma}
	Sei $n \geq 3$. Dann ist die Wirkung von $A_{n}$ auf $\{1,\ldots,n\} $ transitiv.
\end{lemma}

\begin{proof}
	$(1,2,3) \in A_{n}$ bildet $1$ auf $2$ ab. Für $i \geq 3$ bildet $(1,i,2) \in A_{n}$ die $1$ auf $i$ ab.
	Also ist in beiden Fällen die Bahn von $1$ ganz $\{1,\ldots,n\}$.
\end{proof}

\begin{lemma}
	Sei $n \geq 5$ und $H \lTri A_{n}$ nicht die triviale Gruppe. Dann enthält $H$ eine Permutation $\sigma \neq e$ mit mindestens einem Fixpunkt.
\end{lemma}

\begin{proof}
	Sei $\sigma \in H$ und $\tau \in A_{n}$. Dann gilt $[\tau, \sigma] = \tau \sigma \tau^{-1} \sigma^{-1} \in H$, da $\tau \sigma \tau^{-1} \sigma^{-1} \in H$.
	Angenommen $\sigma \in H \setminus \{ e\} $. Falls $\sigma$ einen Fixpunkt hat, so gilt das Lemma in diesem Fall.
	Wir nehmen also an, dass $\sigma$ keinen Fixpunkt hat und werden je nach Zyklentyp von $\sigma$ immer ein $\sigma' \in H \setminus \{e\}$ finden,
	dass ein Fixpunkt besitzt (Meist $\sigma' = [\tau, \sigma]$ für ein geeignetes $\tau \in A_{n}$).
	\begin{itemize}
		\item $\sigma$ enthält ein Zyklus der Länge $k \geq 4$.
			Sei $\sigma = (i_1,i_2,i_3,\ldots,i_{k}) \ldots$ und $\tau = (i_1,i_2,i_3) \in A_{n}$. Folgt 
			\[
				\sigma' = [\tau,\sigma] = (i_1,i_2,i_3) \sigma (i_1,i_2,i_3)^{-1} \sigma^{-1}: \begin{cases}
					i_1 \stackrel{\sigma^{-1}}{\mapsto} i_{k} \mapsto \stackrel{\tau^{-1}}{\mapsto} i_{k} \stackrel{\sigma}{\mapsto} i_1 \stackrel{\tau}{\mapsto} i_2\\
					i_3 \stackrel{\sigma^{-1}}{\mapsto} i_{2} \mapsto \stackrel{\tau^{-1}}{\mapsto} i_{1} \stackrel{\sigma}{\mapsto} i_2 \stackrel{\tau}{\mapsto} i_3
				\end{cases}
			\]
			Nichttrivial (wegen $k\geq 4$), ein Fixpunkt.
		\item $\sigma$ enthält sowohl Zyklen der Länge $2$ als auch Zyklen der Länge $3$.
			$\sigma' = \sigma^2 \in H$ hat Zyklen der Länge $3$ und Fixpunkte.
		\item $\sigma$ enthält nur Zyklen der Länge $3$ (mind. $2$ wegen $n\geq 5$ )
			Sei $\sigma = (i_1,i_2,i_3)(i_4,i_5,i_6) \ldots$ und $\tau = (i_1,i_2,i_4)$. Folgt 
			\[
				\sigma' = [\tau,\sigma] = (i_1,i_2,i_4) \sigma (i_1,i_2,i_4)^{-1} \sigma^{-1}: \begin{cases}
					i_1 \stackrel{\sigma^{-1}}{\mapsto} i_{3} \mapsto \stackrel{\tau^{-1}}{\mapsto} i_{3} \stackrel{\sigma}{\mapsto} i_1 \stackrel{\tau}{\mapsto} i_2\\
					i_6 \stackrel{\sigma^{-1}}{\mapsto} i_{5} \mapsto \stackrel{\tau^{-1}}{\mapsto} i_{5} \stackrel{\sigma}{\mapsto} i_6 \stackrel{\tau}{\mapsto} i_6
				\end{cases}
			\]
			Nichttrivial, ein Fixpunkt.
		\item $\sigma$ enthält nur Zyklen der Länge $2$ (mind. wegen $N \geq 5$ )
			Sei $\sigma = (i_1,i_2)(i_3,i_4)(i_5,i_6) \ldots$ und $\tau = (i_1,i_2,i_3)$. Folgt 
			\[
				\sigma' = [\tau,\sigma] = (i_1,i_2,i_3) \sigma (i_1,i_2,i_3)^{-1} \sigma^{-1}: \begin{cases}
					i_1 \stackrel{\sigma^{-1}}{\mapsto} i_{2} \mapsto \stackrel{\tau^{-1}}{\mapsto} i_{1} \stackrel{\sigma}{\mapsto} i_2 \stackrel{\tau}{\mapsto} i_3\\
					i_5 \stackrel{\sigma^{-1}}{\mapsto} i_{6} \mapsto \stackrel{\tau^{-1}}{\mapsto} i_{6} \stackrel{\sigma}{\mapsto} i_5 \stackrel{\tau}{\mapsto} i_5
				\end{cases}
			\]
			Nichttrivial (wegen $k\geq 4$). Es existiert Fixpunkt.
	\end{itemize}
	Dies deckt alle Fälle ab.
\end{proof}

\begin{proof}[Beweis, dass $A_5$ einfach ist]
	Sei $\{e\} \neq H \lTri A_5$ und $\sigma \in H \setminus \{e\} $ eine Permutation mit einem Fixpunkt wie im Lemma.
	Insbesondere ist $\sigma$ kein $5$-Zyklus und wegen $\sigma \in H$ auch kein $4$-Zyklus.
	Also ist $\sigma$ entweder ein $3$-Zyklus oder mit Zyklentyp $2,2$.
	Angenommen $\sigma = (i_1, i_2) (i_3,i_4)$ und $\tau = (i_1,i_2,i_3)$ für $i_5 \neq i_1,i_2,i_3,i_4$.
	Dann ist $\tau \sigma \tau^{-1} = (i_2,i_5)(i_3,i_4)$ und 
	\[
		\underbrace{\sigma}_{\in H} \underbrace{\tau \sigma \tau^{-1}}_{\in H} = (i_1,i_2)(i_3,i_4)(i_2,i_5)(i_3,i_4) = \begin{cases}
			i_1 \mapsto i_1 \mapsto i_1 \mapsto i_1 \mapsto i_2\\
			i_2 \mapsto i_2 \mapsto i_5 \mapsto i_5 \mapsto  i_5\\
			i_5 \mapsto i_5 \mapsto  i_2 \mapsto i_2\mapsto i_1\\
			i_3 \mapsto i_4 \mapsto i_4 \mapsto i_3 \mapsto i_3\\
			\qquad \qquad\;\,\ldots
		\end{cases} = (i_1,i_2,i_5) \in H
	.\]
	Also enthält $H$ auch einen $3$-Zyklus.

	\begin{claim}
		Alle $3$-Zyklen sind in $A_5$ konjugiert. Also enthält die normale Untergruppe $H$ alle $3$-Zyklen.
		Sei $\sigma = (i_1,i_2,i_3)$ für beliebige verschiedene $i_1,i_2,i_3 \in \{1,\ldots,5\}$.
		Wir definieren $\tau = \begin{pmatrix} 
			1 &2 &3 &4 &5\\
			i_2 &i_2 &i_3 &* &\diamond 
		\end{pmatrix} $,
		wobei wir die verbleibenden Eintragungen $* \neq \diamond $ in $\{1,\ldots,5\} \setminus \{i_1,i_2,i_3\} $ wählen.
		Falls $\sgn(\tau) = -1$ vertauschen wir $*$ und $\diamond $ und erhalten $\tau \in A_{n}$.
		Damit gilt dann $\tau (1,2,3) \tau^{-1} = (i_1,i_2,i_3)$, was die Behauptung beweist.
		Wir berechnen
		 \[
			 (i_1,i_2,i_3)(i_2,i_3,i_4) = (i_1,i_2)(i_3,i_4) \qq{und} \underbrace{(i_1,i_2,i_3)(i_3,i_4,i_5)}_{\in H} = (i_1,i_2,i_3,i_4,i_5)
		\] 
		für eine beliebige Aufzählung $i_1,\ldots,i_5$ von $1,\ldots,5$.
		Folgt $H = A_5$ enthält alle $5$-Zyklen und alle Elemente vom Zyklentyp $2,2$.
	\end{claim}
\end{proof}

\begin{proof}[Beweis für $n > 5$ mittels Induktion]
	Angenommen $\{e\} \neq H \lTri A_{n}$ und $\sigma \in H \setminus \{e\}$ hat einen Fixpunkt.
	Wegen dem ersten Lemma dürfen wir auch ohne Beschränkung annehmen, dass $\sigma(n) = n$.
	\[
	\implies \{e\}  \neq H \cap A_{n-1} \lTri A_{n-1}
	.\] 
	Nach Induktionsannahme gilt also $H \cap A_{n-1} = A_{n-1}$.
	Wegen dem ersten Lemma folgt, dasss jedes Element von $A_{n}$ mit einem Fixpunkt zu einem Element von $A_{n-1}$ konjugiert ist.
	Zusammengenommen erhalten wir, dass $H$ jedes Element mit einem Fixpunkt enthält.
	
	Sei $\sigma \in A_{n}$ beliebig.
	\begin{itemize}
		\item Falls $\sigma$ einen Fixpunkt hat so ist $\sigma \in H$.
		\item Ansonsten schreiben wir $\sigma = (1,\sigma(1),i)((1,\sigma(1),i)^{-1} \sigma)$
			wobei $i \in \{1,\ldots,n\} \setminus \{1,\sigma(1)\} $ und der erste Zyklus $n-3$ Punkte fixiert und der zweite einen fixiiert
			und daher beide in $H$ sind.
	\end{itemize}
	Es folgt also $\sigma \in H$.

	Da $\sigma \in A_{n} $ beliebig war, gilt $H = A_{n}$.
\end{proof}

\section{Gruppen kleiner Ordnung \& Klassifikation}

\begin{theorem}
	Sei $G$ eine Gruppe der Ordnung $n = \abs{G} < 100$. 
	Dann ist entweder $G$ auflösbar der $n = 60$ und $G \simeq A_5$.
\end{theorem}

Für den Beweis des Satzes bedienen wir uns vieler bereits bewiesenen kleinen Lemmas,
dem Sylowsatz und weiteren Leamms mit zunehmender Komplexität.
Des Weiteren verwenden wir Induktion nach $n$ und einen grundlegene Eigenschaft von Auflösbarkeit.

\begin{definition}[Wiederholung]
	Sei $G$ eine Gruppe. Wir sagen $G$ ist \emph{auflösbar} falls es einen Subnormalreihe
	\[
	\{e\} = G_0 \lTri G_1 \lTri \ldots \lTri G_{k} = G
	\]
	gibt für die die Faktorgruppen $\frac{G_{j}}{G_{j-1}}$ für $j = 1,\ldots,k$ alle abelsch sind.
\end{definition}

\begin{proposition}[Legoeigenschaft und Auflösbarkeit]
	Sei $G$ eine Gruppe und $N \lTri G$. Falls $N$ und $\sfrac{G}{N}$ auflösbar sind, so gilt dasselbe für $G$.
\end{proposition}

\begin{proof}
	Seien $N \lTri G$ und $\sfrac{G}{N}$ auflösbar. Dann existieren Subnormalreihen
	\begin{align*}
		\{e\} = G_0 \lTri G_1 \lTri \ldots \lTri G_{l} = N \tag{$*_{1}$}\\
		\{eN\} = H_0 \lTri H_1 \lTri \ldots \lTri H_{m} = \sfrac{G}{N} \tag{$*_{2}$}
	\end{align*}
	mit abelschen Faktorgruppen. Sei $\pi: G \to \sfrac{G}{N}$ die kanonische Projektion.
	Wir definieren
	\[
		G'_{j} = \pi^{-1}(H_{i}) < G
	\] 
	und erhalten
	\[
		G_{l} = N = \pi^{-1}(e N) = G'_{0} < G'_{1} < \ldots < G'_{m} = G
	.\] 
	\begin{claim}
		$G'_{j-1} \lTri G'_{j}$ und $\sfrac{G'_{j}}{G'_{j-1}} \cong \sfrac{H_j}{H_{j-1}}$ für $j = 1,\ldots,m$.
	\end{claim}
	Gemeinsam mit $(*_{1})$ beweist die Behauptung die Proposition, da damit die Subnormalreihe
	\[
		\{e\}  = G_{0} \lTri G_1 \lTri \ldots \lTri G_{l} = N = G'_{0} \lTri \ldots \lTri G'_{m} = G
	\] 
	alle Eigenschaften wie in der Definition erfüllt.

	Seien $h \in G_{}$
\end{proof}

%TODO missing rest of section see gruppen 21 and gruppenmoduln 22

\section{Freie Gruppen und Relationen}
\begin{definition}
	Sei $n \geq 1$ eine natürliche Zahl. Dann wird $\Z^{n}$ als die \emph{freie abelsche Gruppe} mit $n$ Erzeugenden  
	$b_1 = (1,0,\ldots,0)^{T}, \ldots , b_{n} = (0,\ldots,0,1)^{T}$ bezeichnet.
\end{definition}

\begin{lemma}
	Sei $G$ eine abelsche Gruppe und $a_1,\ldots,a_{n} \in G$. Dann gibt es einen eindeutig bestimmten 
	Gruppenhomomorphismus $\phi: \Z^{n} \to G$ mit $\phi(b_{j}) = a_{j}$ für $j = 1,\ldots,n$.
\end{lemma}

\begin{proof}[Beweis Idee]
	Verwende $\phi(m) = \phi(\sum_{j=1}^{n} m_{j} b_{j}) = \sum_{j=1}^{n} m_{j} \phi(b_{j}) =\sum_{j=1}^{n} m_{j} a_{j}$
\end{proof}

\begin{theorem}
	Sei $n \geq 1$ und $b_1,\ldots,b_{n}$ paarweise verschieden. Dann existiert eine \emph{\enquote{freie Gruppe} $F_{n}$},
	welche von $b_1,\ldots,b_{n}$ erzeugt wird, mit folgender \enquote{universeller} Eigenschaft:
	Für jede Gruppe $G$ und Elemente $a_1,\ldots,a_{n} \in G$ gibt es einen eindeutig bestimmten Homomorphismus 
	$\phi: F_{n} \to G$ mit $\phi(b_{j}) = a_{j}$ für $j=1,\ldots,n$.
\end{theorem}




































