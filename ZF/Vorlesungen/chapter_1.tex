%! TEX root=../algebra.tex
\graphicspath{{Images/}}

\chapter{Einführung}
\section{Algebra}
\paragraph{Was ist Algebra?}
\begin{enumerate}
	\item Gruppen und Wirkungen
	\item Ringe und Module
	\item Körpertheorie
\end{enumerate}
\paragraph{Wozu ist Algebra gut?}
Zentrale Grundlage für die reine Mathematik.

\paragraph{Was sind die möglichen Ziele dieser Vorlesung?}
Grundlagen schaffen. Viele tolle Sätze und Zusammenhänge schaffen.

\section{Geschichtlicher Überblick}
Algebra vom Arabischen al-gabr, \enquote{das Zusammenfügen gebrochener Teile} (Äquivalenzumformungen).

Epochen:
\begin{enumerate}
	\item Babylonier ca. $2000$ v.Chr.
	\item Griechen $600$ v. - $300$ v.Chr.
	\item Inder 5. - 7. Jhdt.
	\item Araber 8. - 13. Jhdt.
	\item Italiener 16. Jhdt.
	\item Leibniz
	\item Lineare Algebra 
	\item E. Galois
	\item E. Noether ($\sim 1930$)
\end{enumerate}

\section{Polynomielle Gleichungen}
\paragraph{Quadratische Gleichung}
\begin{align*}
	x^2 + p x + q = 0 \implies x^2 + 2 \frac{p}{2} x + \frac{p^2}{4} = \frac{p^2}{4} - q \implies (x+\frac{p}{2})^2 = \frac{p^2}{4}-q
	\implies x = -\frac{p}{2} \pm \sqrt{\frac{p^2}{4} - q} 
\end{align*}

\paragraph{Kubische Gleichung}
$a,b,c$ gegeben.
\begin{align*}
	x^3 + a x^2 + b x + c = 0 
\end{align*}
Wir setzen $x = y - \frac{a}{3}$, dann ist
\begin{align*}
	&x^3 = (y-\frac{a}{3})^3 = y^3 - 3 \frac{a}{3} y^2 + 3 (\frac{a}{3})^3 y - (\frac{a}{3})^3\\
	&a x^2 = a(y-\frac{a}{3})^2 = a (y^2 - 2 \frac{a}{3} y + (\frac{a}{3})^2)\\
	&bx = \ldots\\
	\cline{1-2}
	&y^3 + py + q = 0
\end{align*}
Ansatz: $y = g+h$.
\begin{align*}
	g^3 + 4g^2 h + 4 g h^2 + h^3 + p (g+h)  + q = 0\\
	\underbrace{g^3 + h^3 + q}_{=0} + \underbrace{(3gh + p)}_{=0} (g+h) = 0\\
	g^3 + h^3 = -q \qq{und} gh = -\frac{p}{3}
.\end{align*}
Sei $G = g^3, H = h^3$. Dann folgt $G+H = -q$ und $GH = -(\frac{p}{3})^3$
Folgt $G(-q-G) = -\frac{p^3}{27}$. Dies können wir mit der Quadratischen Gleichung lösen.
Die sich daraus ergebende Formel wird auch die Cardano-Formel genannt.
Hat eine Kubischegleichung drei reelle Lösungen so verwendet jede Lösungsformel zur Berechnung dieser Komplexe Zahlen (Casus Irreduzibiles).

\paragraph{Quadratische Gleichung}
Diese wurde kurz danach von Ferrari gelöst. Herleitung siehe Buch.

\paragraph{Gleichung 5. Grades}
1824 Abel: es kann keine Formel mit Wurzelausdrücken geben.\\
1830 Galois: Vollständige Erklärung und Erfindung der Gruppentheorie zu diesem Zweck.

\section{Zahlentheorie}
\[
\N \subseteq \Z \subseteq \Q \subseteq \R \subseteq \C
\] 
wobei $\N$ die $0$ enthält. In $\N$ und in $\Z$ gibt es Primzahlen.
\begin{theorem}
	Es gibt in $\Z$ eine eindeutige Primfaktorzerlegung.
	Jede Zahl $n \in \Z \setminus \{0\}$ lässt sich als Produkt
	\[
	n = \epsilon \cdot p_1^{k_1} \cdot \ldots \cdot p_{l}^{k_{l}}
	.\] 
	schreiben, wobei $\epsilon = \pm 1$, $l \in \N$, $p_1,\ldots,p_{l} > 0$ Primzahlen sind und $k_1,\ldots,k_{l} \in \N_{> 0}$ sind.
	Diese Darstellung ist bis auf die Reihenfolge der Primzahlen eindeutig.
\end{theorem}

Welche Zahle in $\N$ sind Summen von zwei Quadratzahlen?
z.B. $3$ geht nicht, $5 = 4+1$

Was sind die Primzahlen in $\Z[i] = \{a + ib: a,b \in \Z\} \subseteq \C$?
z.B. $5 = (2+i)(2-i)$ ist keine Primzahl in $\Z[i]$, $3$ ist eine.

Sehr nützlich für diese Frage ist:
\begin{definition}
	Wir schreiben für $a,b,m \in \Z$, dass $a \equiv b \mod m$ falls $m$ die Differenz $b-a$ teilt (also falls es ein  $k \in \Z$ gibt mit $b-a = k m $)
\end{definition}

Dies definiert eine \emph{Äquivalenzrelation} (für festes $m \in \Z$).
Der Quotientenraum wird mit 
\[
	\Z_{m} = \Z / \Z_{m} = \{a + \Z_{m}: a \in \Z\} 
\]
bezeichnet.

\begin{lemma}
	Addition und Multiplikation auf $\Z$ definieren auch wohldefinierte Addition und Multiplikation auf $\Z_{m}$.
	D.h. für festes $m$ gilt: $a_1 \equiv a_2$ und $b_1 \equiv b_2 \mod m \implies a_1 + b_1 \equiv a_2 + b_2$ und $a_1 \cdot b_1 \equiv a_2 \cdot b_2$ modulo $m$.
\end{lemma}


\begin{lemma}
	Die einzigen Quadratzahlen in $\Z_4$ sin $0$ und $1$.
	Daher sind die Zahlen $3,7,11,15,19,\ldots$ keine Summen von zwei Quadratzahlen in $\N$.
\end{lemma}


\subsection{Kurze Diskussion zu ggT}
\begin{proposition}
	Für je zwei natürliche Zahlen $a,b \in \N_{>0}$ gibt es einen größten gemeinsamen Teiler $d > 0$.
	Dieser erfüllt $\Z a + \Z b = \Z d$.
\end{proposition}


\section{Geometrie}
Alte Griechen sahen Geometrie und Zahlentheorie getrennt.

Descartes (1986-1649): Kann man $\sqrt[3]{2}$ oder $20^{\circ}$ mit Zirkel und Lineal konstruieren.

Klassifikation von Geometrie im Erlanger Programm Klein $1872$














