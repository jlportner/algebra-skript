%! TEX root=../algebra.tex
\graphicspath{{Images/}}

\chapter{Kommutative Ringe}

\section{Ringe}

\begin{definition}
	Ein \emph{Ring} ist eine Menge $R$ ausgestattet mit Elementen $0 \in R$, $1 \in R$ und drei Abbildungen
	\[
		\begin{cases}
			+: R \times R \to R\\
			-: R \to R\\
			\cdot: R \times R \to R
		\end{cases}
	\]
	so dass folgende Axiome gelten.

	$(R,+)$ ist eine abelsche Gruppe mit neutralem Element $0$ und Inversem $-$ d.h.
	\begin{gather*}
		(a+b) + c = a + (b+c)\\
		0 + a = a\\
		(-a) + a = 0\\
		a + b = b + a
	\end{gather*}
	für alle $a,b,c \in R$.

	$(R,\cdot)$: Assoziativität $(a\cdot b) \cdot c = a \cdot (b \cdot c)$ und Einselement $1 \cdot a = a = a \cdot 1$.

	Distributivität: $a ( b + c) = ab + ac$ und $(b+c) a = ba + ca$.

	Falls zusätzlich Kommutativität von $\cdot$ gilt: $ab = ba$, dann sprechen wir von einem \emph{kommutativen Ring}.
\end{definition}

\begin{remark}
	\begin{itemize}	
		\item $0$ ist eindeutig durch die Axiome bestimmt. 
		\item Ebenso ist $-a$ durch die Axiome für jedes $a \in R$ eindeutig bestimmt.
		\item $0 \neq 1$ wurde nicht verlangt.
		\item $0 \cdot a = 0$ für jedes $a \in R$ :
			 \begin{align*}
				 0 \cdot a = (0+0) \cdot a = 0\cdot a + 0 \cdot a \implies 0 = 0 \cdot a
			.\end{align*}
	\end{itemize}
\end{remark}

\begin{convention}
	\begin{itemize}
		\item Klammern bei $+$ (und ebenso bei $\cdot$) lassen wir auf Grund der Assoziativität der Addition (Mult.) weg also $a+b+c+d$.
		\item Punktrechnung vor Strichrechnung, d.h. $a\cdot b +c = (a\cdot b) + c$.
		\item Den Multiplikationspunkt lässt man oft weg.
	\end{itemize}
\end{convention}

\begin{notation}
	\begin{gather*}
		0\cdot a = 0 \quad 1\cdot a = a \quad 2\cdot a = a+a \quad 3\cdot a = a+a+a\\
		(n+1) = n\cdot a + a, (-n) \cdot a = -(n\cdot a) \text{ für } n \in \N
	.\end{gather*}
	Dies definiert eine Abbildung $\Z \times R \to R, (n,a) \mapsto n\cdot a$.
	Diese erfüllt:  $(m+n) \cdot a = m\cdot a + n \cdot a$, $n \cdot (a+b) = n\cdot a + n \cdot b$.

	Ebenso definieren wir
	\begin{gather*}
		a^{0} = 1_{R} \quad a^{1} = a \quad a^2 = a\cdot a \quad a^{n+1} = a^{n} \cdot a \text{ für } n \in \N
	\end{gather*}
	Diese erfüllt
	\[
		a^{m+n} = a^{m} + a^{n} \quad (a^{m})^{n} = a^{m\cdot n} \quad (ab)^{n} = a^{n} b^{n}
	\] 
	in kommutativen Ringen.
\end{notation}

\begin{definition}
	Angenommen $R,S$ sind Ringe und $f: R \to S$ ist eine Abbildung.
	Wir sagen $f$ ist ein \emph{Ringhomomorphismus} falls
	\begin{gather*}
		f(1_{R}) = 1_{S} \quad f(a+b) = f(a) + f(b) \quad f(a\cdot b) = f(a) \cdot f(b)
	\end{gather*}
	für alle $a,b \in R$.
	Falls $f$ invertierbar ist, so nennen wir $f$ einen \emph{Ringisomorphismus}.
\end{definition}

\begin{remark}
	$f(0_{R} = 0_{S}$ denn $f(0_{R}) = f(0+0) = f(0) + f(0) \geq 0_{S} = f(0_{R})$.\\
	$f(-a) = -f(a)$ für $a \in R$ (ähnlicher Beweis).
\end{remark}

\begin{definition}
	Sei $R$ ein Ring und $S \subseteq R$ auch ein Ring. 
	Wir sagen $S$ ist ein \emph{Unterring}, falls $\id: S \to R, s \mapsto s$ ein Ringhomomorphismus ist.
\end{definition}



\begin{lemma}
	Falls in einem Ring $R$ gilt $0=1$, dann ist  $R = \{0\}$.
\end{lemma}


\begin{lemma}[Binomialformel]
	Sei $R$ ein Ring und $a,b \in R$ mit $ab = ba$ (z.B. weil $R$ kommutativ ist).
	Dann gilt für jedes $n \in \N$ $(a+b)^{n} = \sum_{k=0}^{n} \binom{n}{k} a^{k} b^{n-k}$.
\end{lemma}


Falls $n = 2$ ist und $(a+b)^2 = a^2 + 2ab + b^2$ gilt. Dann folgt $ab = ba$.

\begin{attention}
Ab nun werden wir nur kommutative Ringe betrachten.
\end{attention}

\section{Einheiten, Teilbarkeit, Quotientenkörper (Seite 34)}

\begin{definition}
	Sei $R$ ein Ring. Ein Element $a \in \R \setminus \{0\}$ heißt ein Nullteiler falls es ein $b \in \R \setminus \{0\} $ mit $ab = 0$ gibt.
\end{definition}

\begin{definition}
	Ein kommutativer Ring heißt ein Integritätsbereich falls $0 \neq 1$ und falls aus $ab = ac$ und $a\neq 0$ $b = c$ folgt (Kürzen).
\end{definition}

\begin{lemma}
	Sei $R$ ein kommutativer Ring mit $0 \neq 1$. Dann ist $R$ ein Integritätsbereich gdw. $R$ keine Nullteiler besitzt.
\end{lemma}



\begin{definition}
	Sei $R$ ein kommutativer Ring und  $a,b \in R$.
	Wir sagen \emph{$a$ teilt  $b$}, $a \vert b$ [in $R$] falls es ein $c$ in $R$ gibt mit $b = a \cdot c$.
\end{definition}

\begin{definition}
	Wir sagen $a \in R$ ist eine \emph{Einheit} falls $a \vert 1 \Leftrightarrow \exists b	$ mit $ab = 1 \Leftrightarrow \exists a^{-1} \in R$.
	Einheiten mit $R^{x} = \{a \in R \mid a \vert 1\} $
\end{definition}

\begin{remark}
	$R^{x}$ bildet eine Gruppe, $1 \in R^{x}$, $a,b \in R^{x} \implies (ab)(a^{-1} b^{-1}) = a a^{-1} b b^{-1} = 1 \implies a b \in R^{x}$.
\end{remark}


\begin{definition}
	Ein \emph{Körper (field)} $K$ ist ein kommutativer Ring in dem $0 \neq 1$ und jede Zahl ungleich Null eine multiplikative Inverse besitzt.
\end{definition}

\begin{lemma}
	Ein Körper ist ein Integritätsbereich.
\end{lemma}


\begin{proposition}
	Sei $m \geq 1$ eine natürliche Zahl.
	Dann ist $\Z_{m}$ ein Körper genau dann wenn $m$ eine Primzahl ist.
\end{proposition}


\begin{theorem}[Quotientenkörper (S.38)]
	Sei $R$ ein Integritätsbereich. Dann gibt es einen Körper $K$, der $R$ enthält und so dass $K = \{\frac{p}{q}: p,q \in R , q \neq 0\}$.
	z.B. für $R = \Z$ haben wir $K = \Q$.
\end{theorem}


	Ab sofort schreiben wir $\frac{a}{b} = [(a,b)]_{\sim}$.
	Wir identifizieren $a \in R$ mit $\frac{a}{1} \in K$. Hierzu bemerken wir, dass $\iota: a \in R \mapsto \frac{a}{1} \in K $ ein injektiver Ringhomomorphismus ist.


\begin{definition}
	Sei $K$ ein Körper und $L \subseteq K$ ein Unterring der auch ein Körper ist. Dann nennen wir $L$ auch einen \emph{Unterkörper}.
\end{definition}


\section{Ring der Polynome (Seite 41)}
Im Folgenden ist $R$ immer ein kommutativer Ring. Wir wollen einen neuen Ring, den Ring $R[X]$ der Polynome
in der Variablen $X$ und Koeffizienten in $R$ definieren.


\begin{definition}
	Sei $R$ ein kommutativer Ring. Wir definieren den \emph{Ring der formalen Potentreihen} (in einer Variable über dem Ring $R $) als 
	\begin{enumerate}
		\item die Menge aller Folgen $\left(a_{n}\right)^{\infty}_{n=0} \in R^{\N}$
		\item $0 = \left(0\right)^{\infty}_{n=0}, 1 = (1,0,0,\ldots)$ 
		\item $+: \left(a_{n}\right)^{\infty}_{n=0} + \left(b_{n}\right)^{\infty}_{n=0} = (a_{n} + b_{n})^{\infty}_{n=0}$ 
		\item $\cdot: \left(a_{n}\right)^{\infty}_{n=0} \cdot \left(b_{n}\right)^{\infty}_{n=0} = \left(c_{n}\right)^{\infty}_{n=0} $ wobei
			\[
				c_{n} = \sum_{i=0}^{n} a_{i} b_{n-i} = \sum_{\substack{i+j=n \\ i,j \geq 0}}^{\infty} a_{i} b_{j}
			.\] 
	\end{enumerate}
	Die Menge aller Folgen mit $a_{n} = 0$ für alle hinreichend großen $n \geq 0$ wird als der \emph{Polynomring} (in einer Variable und über $R$ ) bezeichnet.
\end{definition}


\begin{notation}
	Wir ühren ein neues Symbol, eine Variable, z.B. $X$ ein und identifizieren $X$ mit
	\begin{align*}
		X^{0} = 1 = (1,0,0,\ldots) \quad X^{1} = (0,1,0,0,\ldots) \quad X^2 = (0,0,1,0,\ldots) \quad \ldots
	.\end{align*}
	Allgemeiner: Sei $a$ ein Polynom, dann ist
	\[
		X \cdot a = (0,a_0,a_1,a_2,\ldots)
	\]
	denn $(X\cdot a)_{n} = \sum_{i+j=n} X_{i} a_{j} = a_{n-1}$ 
	da $X = 0$ außer wenn $i = 1$ ist. $(X\cdot a)_{0} = X_0\cdot a_0 = 0$.

	Wir schreiben $R[X] = \{\sum_{i=0}^{n} a_{i} X^{i} : n \in \N, a_0,\ldots,a_{n} \in R\}$ ($R$-adjungiert-$X$) für den \emph{Ring der Polynome in der Variablen $X$}
	und $R[\![ X ]\!] = \{\sum_{n=0}^{\infty} a_{i} X^{i} : a_0,a_1,\ldots \in R\} $ für den \emph{Ring der formalen Potenzreihen in der Variable $X$}
\end{notation}

\begin{definition}
	Sei $p \in R[X] \setminus \{0\}$. Der Grad von $p$ $\deg(p)$ ist gleich $n \in \N$ falls $p_{n} \neq 0$ ist und
	$p_{k} = 0$ für $k> n$. In diesem Fall nennen wir $p_{n}$ auch den \emph{führenden Koeffizienten}.
	
	Wir definieren $\deg(0) = - \infty$.
\end{definition}

\begin{proposition}
	Sei $R$ ein Integritätsbereich. Dann ist $R[X]$ auch ein Integritätsbereich.
	Des weiteren gilt für $p,q \in R[X] \setminus \{0\} $ 
	\begin{itemize}
		\item $\deg(pq) = \deg(p) + \deg(q)$ und der führende Koeffizient von $pq$ ist das Produkt der führenden Koeffizienten von $p$ und $q$.
		\item $\deg(p+q) \leq \max(\deg(p), \deg(q))$ 
		\item Falls $p \mid q$, dann gilt $\deg(p) \leq \deg(q)$.
	\end{itemize}
\end{proposition}


\begin{definition}
	Sei $K$ ein Körper. Dann wird der Quotientenkörper von $K[X]$ als der \emph{Körper der rationalen Funktionen}
	$K(X) = \{\frac{f}{g} : f,g \in K[x], g \neq 0\} $ bezeichnet.
\end{definition}

Wenn wir obige Konstruktion (des Polynomrings) iterieren, erhalten wir den Ring der Polynome in mehreren Variablen
\[
	R[X_1,X_2,\ldots,X_{d}] := (R[X_1])[X_2][X_3]\ldots[X_{d}]
.\] 
Falls $R = K$ ein Körper ist, definieren wir auch
\[
	K(X_1,X_2,\ldots,X_{d}) = \operatorname{Quot}(K[X_1,\ldots,X_{d}])
.\] 

\begin{remark}
	Auf $R[X_1,\ldots,X_{d}]$ gibt es mehrere Grad-Funktionen
	\begin{align*}
		&\deg(x_1), \deg(x_2), \ldots \deg(x_{d})\\
		&\deg_{\text{total}}(f) = \max \{m_1+\ldots+m_{d} \mid f_{m_1,\ldots,m_{d}} \neq 0\} 
	\end{align*}
	für $f = \sum_{m_1,\ldots,m_{d}} f_{m_1,\ldots,m_{d}} X_1^{m_1}\ldots X_{d}^{m_{d}}$.
	z.B.
	\begin{align*}
		\deg_{\text{total}}(1+X_1^{3} + X_2 X_3) = 3 \qquad \deg_{X_2}(1+X_1^{3} + X_2 X_3) = 1
	.\end{align*}
\end{remark}

\begin{theorem}
	Seien $R,S$ zwei kommutative Ringe. Ein Ringhomomorphismus $\Phi$ von $R[x]$ nach $S$ ist eindeutig durch seine Einschränkung
	$\varphi = \Phi \mid_{R}$ und durch das Element $x = \Phi(X) \in S$ bestimmt. Des weiteren definiert
	\[
		\Phi(\sum_{n=0}^{\infty} a_{n} X^{n} = \sum_{n=0}^{\infty} \phi(a_{n}) x^{n} \tag{$*$}
	\] 
	einen Ringhomomorphismus falls $\varphi: R \to S$ ein Ringhomomorphismus ist und $x \in S$ beliebig ist.
\end{theorem}


\begin{notation}
	Wir schreiben für zwei kommutative Ringe $R,S$ 
	\[
		\hom_{Ring}(R,S = \{ \varphi: R \to S \mid \varphi \text{ ist ein Ringhomomorphismus}\} 
	\] 
	in dieser Notation können wir obigen Satz in der Form
	\[
		\hom_{Ring}(R[X],S) \cong \hom_{Ring}(R,S) \times S
	\] 
	schreiben. Dies kann iteriert werden:
	\[
		\hom_{Ring} (R[x_1,\ldots,x_{d}],S) \cong \hom_{Ring}(R,S) \times \underbrace{S\times \ldots \times S}_{d-\text{mal}}
	.\] 
\end{notation}




\section{Ideale und Faktorringe}

\begin{definition}
	Sei $R$ ein kommutativer Ring.
	Ein Ideal in $R$ ist eine Teilmenge $I \subseteq R$ so dass
	\begin{enumerate}[(i)]
		\item $0 \in I$ 
		\item $a,b \in I \implies a + b \in I$
		\item $a \in I, x \in R \implies xa \in I$
	\end{enumerate}
\end{definition}


\begin{theorem}
Sei $R$ ein kommutativer Ring un $I \subseteq R$ ein Ideal.
\begin{enumerate}
	\item Die Relation $a \sim b \Leftrightarrow a - b \in I$ ist eine Äquivalenzrelation auf $R$.
		Wir schreiben auch $a \equiv b \mod I$ für die Äquivalenzrelation und $\sfrac{R}{I}$ für den Quotienten, den wir Faktorring nennen wollen.
	\item Die Addition, Multiplikation, das Negative induzieren wohldefinierte Abbildungen
		\[
			\sfrac{R}{I} \times \sfrac{R}{I} \to \sfrac{R}{I} \qq{bzw.} \sfrac{R}{I} \to \sfrac{R}{I}
		.\] 
	\item Mit diesen Abbildungen, $0_{\sfrac{R}{I}} = [0]_{\sim}, 1_{\sfrac{R}{I}} = [1]_{\sim}$ ist $\sfrac{R}{I}$ ein Ring und die kanoische Projektion
		$p: R \to \sfrac{R}{I}$ mit $a \in R \mapsto [a]_{\sim} = a + I$ ist ein surjektiver Ringhomomorphismus.
\end{enumerate}
\end{theorem}



\begin{lemma}
	Sei $I \subseteq R$ ein Ideal in einem kommutativen Ring. Dann gilt 
	\[
	I = R \Leftrightarrow 1 \in I \Leftrightarrow I \cap R^{X} \neq \emptyset
	.\] 
\end{lemma}



\begin{definition}
	Sei $R$ ein kommutativer Ring und seien $a_1,\ldots,a_{n} \in R$. Dann wird
	\[
		I = (a_1,\ldots,a_{n}) = \{x_1 a_1 + x_2 a_2 + \ldots + x_{n} a_{n} : x_1,\ldots,x_{n} \in R\} 
	\] 
	das von $a_1,\ldots,a_{n}$ \emph{erzeugte Ideal} genannt.

	Für $a \in I$ wird $I = (a) = Ra$ das von $a$ \emph{erzeugte Hauptideal} genannt.
\end{definition}

\begin{lemma}
	Sei $R$ ein kommutativer Ring.
	\begin{enumerate}[1)]
		\item $(a) \subseteq (b) \Leftrightarrow b \mid a$ 
		\item Falls $R $ ein Integritätsbereich ist, dann gilt $(a) = (b) \Leftrightarrow \exists u \in R^{x}$ mit $b = ua$
	\end{enumerate}
\end{lemma}



Falls $I \subseteq R$ ein Ideal ist und $a \in R$, dann ist die Restklasse für Äuivalent modulo $I$ gleich
\[
	[a]_{N} = \{x \in R: x \sim a\} = a + I
.\] 

\begin{theorem}[Erster Isomorphiesatz]
	Angenommen $R,S$ sind kommutative Ringe und $\varphi: R  \to S$ ist ein Ringhomomorphismus.
	\begin{enumerate}
		\item Dann induziert $\varphi$ einen Ringisomorphismus
			\[
				\overline{\varphi}: \sfrac{R}{\ker(\varphi)} \to \Im(\varphi) = \varphi(R) \subseteq S
			\] 
			so dass $\varphi = \overline{\varphi} \circ p$ wobei $p : R \to \sfrac{R}{\ker(\varphi)}$ die kanonische Projektion ist (Diagramm links).
		\item Sei $I \subseteq \ker(\varphi)$ ein Ideal in $R$.
			Dann induziert $\varphi$ einen Ringhomomorpismus $\overline{\varphi}: \sfrac{R}{I} \to S$ mit $\varphi = \overline{\varphi} \circ p_{I}$ (Diagramm rechts).
			Des weiteren gilt $\ker(\overline{\varphi}) = \sfrac{\ker(\varphi)}{I}$ und $\Im(\overline{\varphi}) = \Im(\varphi)$
	\end{enumerate}
	\[
	\begin{tikzcd}
		R \arrow[d, "p"] \arrow[r, "\varphi"]                      & S \\
		\sfrac{R}{\ker(\varphi)} \arrow[ru, "\overline{\varphi}"'] &  
	\end{tikzcd}\qquad\qquad
	\begin{tikzcd}
		R \arrow[d, "p_{I}"] \arrow[r, "\varphi"]      & S \\
		\sfrac{R}{I} \arrow[ru, "\overline{\varphi}"'] &  
	\end{tikzcd}
	\]
\end{theorem}


\begin{remark}
	Sei $I_{0} \subseteq R$ ein Ideal in einem kommutativen Ring.
	Dann gibt es eine Korrespondenz (kanonische Bijektion) zwischen Idealen in $\sfrac{R}{I_{0}}$ und Idealen in $R$, die $I_0$ enthalten.
	\begin{align*}
		I \subseteq R, I_0 \subseteq I \quad &\mapsto \quad \sfrac{I}{I_0} = \{x+ I_0: x \in I\} \subseteq \sfrac{R}{I_0}\\
		J \subseteq \sfrac{R}{I_0} \quad &\mapsto \quad p_{I_0}^{-1}(J) \subseteq R \qquad (p_{I_0}: \begin{cases}
			R \to \sfrac{R}{I_0}\\
			x \mapsto x + I_0
		\end{cases})
	.\end{align*}
\end{remark}

\begin{definition}
	Wir sagen zwei Ideale $I, J$ in einem kommutativen Ring sind \emph{coprim}, falls $I+J = R$ ist.
	D.h. $\exists a \in I, b \in J $ mit $1 = a + b$.
\end{definition}


\begin{proposition}[Chinesischer Restsatz]
	Sei $R$ ein kommutativer Ring und seien $I_1, \ldots, I_{n}$ paarweise coprime Ideale.
	Dann ist der Ringhomomorphismus $\varphi: R \to \sfrac{R}{I_1} \times  \ldots \times \sfrac{R}{I_{n}}$ mit
	$x \mapsto (x+I_1,\ldots,x+I_{n})$ surjektiv mit $\ker(\varphi) = I_1 \cap \ldots \cap I_{n}$.

	Dies induziert einen Ringisomorphismus $\sfrac{R}{I_1 \cap \ldots \cap I_{n}} \to  \sfrac{R}{I_1} \times \ldots \times \sfrac{R}{I_{n}}$.
\end{proposition}


\section{Charakteristik eines Körpers}
Sei $K$ ein Körper. Dann gibt es einen Ringhomomorphismus $ \varphi: \Z \to K$ mit $\begin{cases}
	n \in \N \mapsto \underbrace{1+\ldots+1}_{n-\text{mal}}\\
	-n \in \N \mapsto -(\underbrace{1+\ldots+1}_{n-\text{mal}})
\end{cases}$

Sei $I = \ker(\varphi)$ so, dass $\sfrac{\Z}{I} \equiv \Im(\varphi) \subseteq K$.
Da $K$ ein Körper ist, ist $\Im(\varphi)$ ein Integritätsbereich.

\begin{lemma}
	Sei $I \subseteq \Z$ ein Ideal. Dann gilt $I = (m)$ für ein $m \in \N$. 
	Der Quotient ist ein Integritätsbreich genau dann wenn $m = 0$ oder $m$ eine Primzahl ist.
\end{lemma}


\begin{definition}
	Sei $K$ ein Körper. Wir sagen, dass $K$ Charakteristik $0$ hat, falls $\varphi: \Z \to K$ injektiv ist.
	Wir sagen, dass $K$ Charakteristik $p \in N_{> 0} $ hat falls $\varphi: \Z \to K$ den Kern $(p)$ hat.
\end{definition}


\begin{proposition}
	Sei $K$ ein Körper mit Charakteristik $p > 0$.
	Dann ist die \emph{Frobeniusabbildung} $F: x \in K \to x^{p} \in K$ ein Ringhomomorphismus.
	Falls $\abs{K} < \infty$, dann ist $F$ ein Ringautomorphismus.
\end{proposition}


\section{Primideale und Maximalideale}
\begin{definition}
	Sei $R$ ein kommutativer Ring, und sei $I \subseteq R$ ein Ideal.
	Wir sagen $I$ ist ein \emph{Primideal}, falls $\sfrac{R}{I}$ ein Integritätsbreich ist.
	Wir sagen $I$ ist ein \emph{Maximalideal}, falls $\sfrac{R}{I}$ ein Körper ist.
\end{definition}

\begin{proposition}
	Sei $I \subseteq R$ ein Ideal in einem kommutativen Ring.
	\begin{enumerate}[1)]
		\item Dann ist $I$ ein Primideal genau dann wenn $I \neq R$ und für alle $a,b \in R$ gilt $ab \in I \implies a \in I $ oder $b \in I$.
		\item Dann ist $I$ ein Maximalideal genau dann wenn $I \neq R$ und es gibt kein Ideal $J$ mit $I \subsetneq J \subsetneq R$.
	\end{enumerate}
\end{proposition}




\begin{remark}
	Der Hilbert'sche Nullstellensatz besagt, dass jedes Maximalideal in $\C[X_1,\ldots,X_{n}]$ von dieser Gestalt ist.
\end{remark}

\begin{theorem}
	Sei $R$ ein kommutativer Ring, und $I \subsetneq R$ ein Ideal. Dann existiert ein Maximalideal $m \supseteq I$.
	Insbesondere existiert in jedem Ring $R \neq [0]$ ein Maximalideal.
\end{theorem}


\section{Unterring}
\begin{definition}
	Sei $R$ ein Ring und $S \subseteq R$ auch ein Ring. Wir sagen $S$ ist ein \emph{Unterring} falls $\id: S \to R, s \mapsto s$ 
	ein Ringhomomorphismus ist.
	
	\textbf{Alternativ Definition:}
	Sei $R$ ein Ring und $S \subseteq R$. Dann ist $S$ ein Unterring falls
	\begin{enumerate}
		\item $0,1 \in S$.
		\item $a-b \in S$ für alle $a,b \in S$.
		\item $a\cdot b \in S$ für alle $a,b \in S$.
	\end{enumerate}
\end{definition}

\begin{notation}
	Sei $S \subseteq R$ ein Unterring in einem Ring $R$.
	Seien $a_1,\ldots,a_{n} \in R$. Wir definieren
	\[
		S[a_1,\ldots,a_{n}] = \bigcap_{\substack{T \subseteq R \text{ Unterring}\\ T \supseteq S\\ a_1,\ldots,a_{n} \in T}} T 
	.\] 
	genannt \enquote{s-adjungiert $a_1,\ldots,a_{n}$}.
	\[
	= \ev_{a_1,\ldots,a_{n}}(S[x_1,\ldots,x_{n}]) = \{\sum_{k_1,\ldots,k_{n} \in M} c_{k_1,\ldots,k_{n}} a_1^{k_1} \ldots a_{n}^{k_{n}} \} 
	.\] 
	mit $\abs{M} < \infty, M \subseteq \N^{n}, c_{k_1,\ldots,k_{n}} \in S$.
\end{notation}




\section{Matrizen}
Sei $R$ ein kommutativer Ring, $m,n \in N_{> 0}$.
Dann bezeichnen wir die Menge $\mat_{mn}(R)$ als die Menge aller $m \times n$-Matrizen
\[
\begin{pmatrix} 
a_{11} &\ldots &a_{1n}\\
\vdots &\ddots &\vdots\\
a_{m_1} &\ldots &a_{mn}
\end{pmatrix} 
.\] 
mit Koeffizienten oder Eintragungen $a_{11},\ldots,a_{mn} \in R$.
Für $m=n$ i definieren wir auch auf $\mat_{mm}(R)$ auf übliche Weise die Addition und Multipliaktion.
Dies definiert auf $\mat_{mm}(R)$ gemeinsam mit dem Einselement $I_{m} = (\delta_{ij})_{i,j}$ eine Ringstruktur.
Sobald $m > 1$ sit, ist dieser Ring nichtkommutativ.

Die Einheiten in $\mat_{mm}(R)$ werden auch als invertierbare Matrizen bezeichnet.
Die Menge wird auch die allgemeine lineare Gruppe vom Grad $m$ über $R$ genannt:
\[
	\operatorname{Gl}_{m}(R) = \mat_{mm}(R)^{\times} = \{A \in \mat_{mm}(R) \mid \text{es existiert ein } B \in \mat_{mm}(R) \text{ mit } AB = BA = I_{n}\}   
.\] 


\begin{proposition}[Meta]
	Jede Rechenregel für Matrizen über $\R$ die nur $+,-,\cdot,0,1$ beinhalten, gilt auch über einem beliebigen kommutativen Ring.
\end{proposition}

\begin{proposition}
	Sei $R$ ein kommutativer Ring
	\begin{itemize}
		\item $\mat_{mm}(R)$ erfüllt die Ringaxiome, also z.B. $A(BC) = (AB)C$
		\item $\det(AB) = \det(A) \det(B)$
		\item  $A \widetilde{A} = \widetilde{A} A = \det(A) I_{m}$, wobei $\widetilde{A}$ die komplementäre Matrix
			\[
				\widetilde{A} = ((-1)^{i+j} \det(A_{ji}))_{i,j}
			.\] 
		\item $\charak_{A}(A) = 0$ für das charakteristische Polynom $\charak_{A}(X) = \det(X I_{m} - A)$ einer Matrix $A$.
	\end{itemize}
\end{proposition}

\begin{remark}
	$\det(A)$ , jeder Koeffizient von $A(BC), (AB)C, A  \widetilde{A}, \widetilde{A} A, \det(A) I, \charak_{A}(X)$, $ \charak_{A}(A)$ 
	hängt polynomiell von den Eintragungen von $A,B,C$ ab, wobei die Koeffizienten in $\Z$ liegen z.B.
	\[
		\det(A) = \sum_{\sigma \in S_{n}} \underbrace{\sgn(\sigma)}_{\in \Z}
			a_{1 \sigma(1)} a_{2 \sigma(2)} \ldots a_{n \sigma(n)}
		\]
		welche Monome in den Eintragungen von $A$ sind.
\end{remark}

\begin{lemma}
	Wenn ein Polynom $f \in \R[X_1,\ldots,X_{n}]$ auf ganz $\R^{n}$ verschwindet, dann ist $f=0$.
\end{lemma}


\begin{remark}
	Das Lemma gilt analog für jeden Körper $K$ mit $\abs{K} = \infty$.
\end{remark}






















