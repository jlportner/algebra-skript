%! TEX root=../algebra.tex
\graphicspath{{Images/}}

\chapter{Auswahlaxiom und das Zornsche Lemma}

\textbf{Auswahlaxiom} (in der Mengenlehre)\\
	Sei $I$ eine nichtleere Menge und seien $X_{i}$ für $i \in I$ nichtleere Mengen.
	Dann ist $\prod_{i \in I} X_{i} \neq \emptyset$, d.h. es existiert eine Funktion
	\[
	f: I \to \bigcup_{i \in I} X_{i}
	\] 
	mit $f(i) \in X_{i}$ für alle $i \in I$.

\begin{remark}
	\begin{itemize}
		\item unabhängig von den anderen ZF-Axiomen der Mengenlehre
		\item kritisiert wegen der Nichtkonstruktivität des Axioms und mancher scheinbar paradoxen Folgerung
		\item notwendig für einen großen Teil der Mathematik
	\end{itemize}
\end{remark}

Häufig wird nicht das Auswahlaiom sondern ein dazu äquivalentes Lemma, das Zornsche Lemma, verwendet. Für dieses benötigen wir etwas mehr Begriffe:

\begin{definition}
	Sei $X$ eine Menge. Eine \emph{Ordnung} auf $X$ ist eine Relation $\leq$ so dass
	\begin{enumerate}[1)]
		\item reflexivität: $x \leq x$
		\item antisymmetrie: $x \leq y$ und $y \leq x$ 
		\item transitivität: $x \leq y$ und $y \leq z \implies x \leq z$ für alle $x,y,z \in X$.
	\end{enumerate}
	Die Ordnung heißt \emph{total} oder \emph{linear} falls zusätzlich
	\begin{enumerate}[1)]
		\setcounter{enumi}{3}
		\item linearität: $x \leq y$ oder $y \leq x$
	\end{enumerate}
	gilt. Ansonsten heißt sie \emph{partiell}.
\end{definition}


\begin{definition}
	Sei $\leq$ eine Ordnung auf einer Menge $X$. Ein Element $x \in X$ heißt \emph{maximal} falls für alle $y \in X$ gilt $x \leq y \implies x = y$.
	Ein Element $m \in X$ ein \emph{Maximum} falls $x \leq m$ für alle $x \in X$ gilt.
\end{definition}

\begin{definition}
	Sei $\leq$ eine Ordnung auf einer Menge $X$ und sei $A \subseteq X$.
	Ein Element $x \in X$ heißt eine \emph{obere Schranke von $A$} falls $a \leq x$ für alle $a \in A$.
	Analog definiert man \emph{untere Schranke von $A$}.
\end{definition}

\begin{definition}
	Sei $\leq$ eine Ordnung auf einer Menge $X$. Eine Teilmenge $K \subseteq X$ heißt eine \emph{Kette} falls für alle $x,y \in K$ gilt
	$x \leq y$ oder $y \leq x$. Wir sagen die Ordnung $\leq$ sei \emph{induktiv} falls jede Kette in $X$ eine obere Schranke besitzt.
\end{definition}

\begin{theorem}[Zornsches Lemma]
	Sei $\leq$ eine induktive Ordnung auf einer Menge $X$.
	Dann existiert ein maximales Element in $X$.
\end{theorem}

\textbf{Typische Anwendung:}
Jeder Vektorraum über $K$ hat eine Hamel-Basis.


Vorerst einige Definitionen und Lemmata:
\begin{definition}
	Für eine Teilmenge $C \subseteq X$ definieren wir
	\[
	\widehat{C} = \{x \in X \setminus C \mid x \text{ ist eine obere Schranke}\} 
	.\] 
	Um die Beweisidee umzusetzen verwenden wir eine \emph{Auswahlfunktion} auf der Menge $\{\widehat{C} : C \subseteq X \text{ s.d. } \hat{C} \neq \emptyset\}$
\end{definition}

\begin{definition}
	Eine Teilkette $K \subseteq X$ heißt eine \emph{f-Kette} falls für jede Teilmenge $C \subseteq K$ mit $\widehat{C} \cap K \neq \emptyset$ das Element
	$f(\widehat{C})$ zu $K$ gehört und eine minimale obere Schranke von $C$ in $K$ ist, also $f(\widehat{C}) \leq y$ für alle $y \in \widehat{C} \cap K$ gilt.
	Dies vermeidet \enquote{unnötige Zwischenschritte}, die zu Problemen bei einer Vereinigung von Ketten führen würde.
\end{definition}


\begin{lemma}[Verlängerung]
	Falls $K$ eine $f$-Kette ist und $\widehat{K} \neq \emptyset$ ist, so ist $K_{neu} = K \cup \{f(\widehat{K})\} $ wieder eine $f$-Kette.
\end{lemma}


\begin{lemma}[Zwei $f$-Ketten]
	Angenommen $K,K'$ sind zwei $f$-Ketten und $K' \setminus K \neq \emptyset$.
	Dann ist $K \subseteq K'$ und es gilt $x \leq x'$ für alle $x \in K$ und $x' \in K' \setminus K$.\\
	Informell: \enquote{$K$ ist eine Anfangsabschrift von $K'$}.
\end{lemma}


\begin{lemma}[Vereinigung]
Wir definieren $K_{max} = \bigcup_{\substack{K \text{ ist eine} \\f-\text{Kette}}} K$.
	Dann ist $K_{max}$ eine $f$-Kette.
\end{lemma}





























