%! TEX root=../algebra.tex
\graphicspath{{Images/}}

\chapter{Auswahlaxiom und das Zornsche Lemma}

\textbf{Auswahlaxiom} (in der Mengenlehre)\\
	Sei $I$ eine nichtleere Menge und seien $X_{i}$ für $i \in I$ nichtleere Mengen.
	Dann ist $\prod_{i \in I} X_{i} \neq \emptyset$, d.h. es existiert eine Funktion
	\[
	f: I \to \bigcup_{i \in I} X_{i}
	\] 
	mit $f(i) \in X_{i}$ für alle $i \in I$.

\begin{remark}
	\begin{itemize}
		\item unabhängig von den anderen ZF-Axiomen der Mengenlehre
		\item kritisiert wegen der Nichtkonstruktivität des Axioms und mancher scheinbar paradoxen Folgerung
		\item notwendig für einen großen Teil der Mathematik
	\end{itemize}
\end{remark}

Häufig wird nicht das Auswahlaiom sondern ein dazu äquivalentes Lemma, das Zornsche Lemma, verwendet. Für dieses benötigen wir etwas mehr Begriffe:

\begin{definition}
	Sei $X$ eine Menge. Eine \emph{Ordnung} auf $X$ ist eine Relation $\leq$ so dass
	\begin{enumerate}[1)]
		\item reflexivität: $x \leq x$
		\item antisymmetrie: $x \leq y$ und $y \leq x$ 
		\item transitivität: $x \leq y$ und $y \leq z \implies x \leq z$ für alle $x,y,z \in X$.
	\end{enumerate}
	Die Ordnung heißt \emph{total} oder \emph{linear} falls zusätzlich
	\begin{enumerate}[1)]
		\setcounter{enumi}{3}
		\item linearität: $x \leq y$ oder $y \leq x$
	\end{enumerate}
	gilt. Ansonsten heißt sie \emph{partiell}.
\end{definition}

\begin{eg}
	\begin{itemize}
		\item $\leq$ in $\R$ ist total
		\item $\mid$ in $\Z$ partiell, da $2 X 3$ und $3 X 2$.
		\item $\subseteq$ auf $\mathcal{P}(A) = \{B \subseteq A\} $
	\end{itemize}
\end{eg}

\begin{definition}
	Sei $\leq$ eine Ordnung auf einer Menge $X$. Ein Element $x \in X$ heißt \emph{maximal} falls für alle $y \in X$ gilt $x \leq y \implies x = y$.
	Ein Element $m \in X$ ein \emph{Maximum} falls $x \leq m$ für alle $x \in X$ gilt.
\end{definition}

\begin{definition}
	Sei $\leq$ eine Ordnung auf einer Menge $X$ und sei $A \subseteq X$.
	Ein Element $x \in X$ heißt eine \emph{obere Schranke von $A$} falls $a \leq x$ für alle $a \in A$.
	Analog definiert man \emph{untere Schranke von $A$}.
\end{definition}

\begin{definition}
	Sei $\leq$ eine Ordnung auf einer Menge $X$. Eine Teilmenge $K \subseteq X$ heißt eine \emph{Kette} falls für alle $x,y \in K$ gilt
	$x \leq y$ oder $y \leq x$. Wir sagen die Ordnung $\leq$ sei \emph{induktiv} falls jede Kette in $X$ eine obere Schranke besitzt.
\end{definition}

\begin{theorem}[Zornsches Lemma]
	Sei $\leq$ eine induktive Ordnung auf einer Menge $X$.
	Dann existiert ein maximales Element in $X$.
\end{theorem}

\textbf{Typische Anwendung:}
Jeder Vektorraum über $K$ hat eine Hamel-Basis.

\begin{proof}[Beweisidee]
	Ausgehend von der leeren Menge (die eine Kette darstellt) wollen wir
	Elemente einer immer Länger werdenden Kette finden, wobei wir immer wieder eine obere Schranke hinzufügen wollen - sofern dies möglich ist.\\
	\ldots\\
	$\implies$ eine Art Induktion

	\textbf{Problem:} Die Vereinigung von Ketten muss keine Kette sein.
\end{proof}

Vorerst einige Definitionen und Lemmata:
\begin{definition}
	Für eine Teilmenge $C \subseteq X$ definieren wir
	\[
	\widehat{C} = \{x \in X \setminus C \mid x \text{ ist eine obere Schranke}\} 
	.\] 
	Um die Beweisidee umzusetzen verwenden wir eine \emph{Auswahlfunktion} auf der Menge $\{\widehat{C} : C \subseteq X \text{ s.d. } \hat{C} \neq \emptyset\}$
\end{definition}

\begin{definition}
	Eine Teilkette $K \subseteq X$ heißt eine \emph{f-Kette} falls für jede Teilmenge $C \subseteq K$ mit $\widehat{C} \cap K \neq \emptyset$ das Element
	$f(\widehat{C})$ zu $K$ gehört und eine minimale obere Schranke von $C$ in $K$ ist, also $f(\widehat{C}) \leq y$ für alle $y \in \widehat{C} \cap K$ gilt.
	Dies vermeidet \enquote{unnötige Zwischenschritte}, die zu Problemen bei einer Vereinigung von Ketten führen würde.
\end{definition}

\begin{eg}
	$K_{min} = \emptyset, \widehat{K}_{min} = X$ ist eine $f$-Kette, die in jeder anderen $f$-Kette enthalten ist.\\
	$K_1 = \{f(\widehat{K}_{min}\} = K_{min} \cup \{f(\widehat{K}_{min})\}$ ist eine weitere $f$-Kette, die in jeder anderen nichtleeren $f$-Kette enthalten ist.
	\begin{itemize}
		\item $\widehat{\emptyset} = X \implies f(x) \in X$ ist definiert
		\item $K_{min}$ ist eine $f$-Kette: $C=\emptyset$ und $f(\widehat{\emptyset}) \in K_{min}$ ist minimal $C=K_{min}$ erfüllt $\widehat{C} \cap K_{min} = \emptyset$
		\item Falls $K$ eine $f$-Kette ist, so können wir $C=\emptyset$ in der Definition verwenden und erhalten $f(\widehat{\emptyset}) \in K$, also $K_{min} \subseteq K_1 \subseteq K$.
	\end{itemize}
\end{eg}

\begin{lemma}[Verlängerung]
	Falls $K$ eine $f$-Kette ist und $\widehat{K} \neq \emptyset$ ist, so ist $K_{neu} = K \cup \{f(\widehat{K})\} $ wieder eine $f$-Kette.
\end{lemma}

\begin{proof}
	Sei $C \subseteq K_{neu}$.
	\begin{itemize}
		\item Falls $\widehat{C} \cap K \neq \emptyset$ ist, so gilt $C \subseteq K$, $f(\widehat{C}) \in K$ und $f(\widehat{C})$ ist eine minimales Element von $\widehat{C} \cap K$ 
			(da $K$ eine $f$-Kette ist). Damit ist aber auch $f(\widehat{C}) \in K_{neu}$ und $f(\widehat{C})$ ist ein minimales Element von $\widehat{C} \cap K_{neu}$ 
			(da $f(\widehat{K})$ eine obere Schranke von $K$ ist).
		\item Falls $C \subseteq K$ und $\widehat{C} \cap K = \emptyset$, dann ist $\widehat{C} = \widehat{K}$, da $K$ eine Kette ist gilt $\widehat{C} \supseteq \widehat{K}$.
			Sei $x \in \widehat{C}, k \in K \implies k \neq \widehat{C}$, also existiert ein $c \in C$ mit $k \leq c \leq x \implies x$ ist eine obere Schranke
			von $K$ und $x \not\in  K$ also $x \in \widehat{K}$ und somit $\widehat{C} \in \widehat{K}$.\\
			Folglich isst $f(\widehat{C}) = f(\widehat{K}) \in K_{neu}$ eine minimale obere Schranke von $C$ in $K_{neu}$.
		\item Falls $f(\widehat{K}) \in C$, so ist $\widehat{C} \cap K_{neu} = \emptyset$ und es gibt nichts zu beweisen.
	\end{itemize}
\end{proof}

\begin{lemma}[Zwei $f$-Ketten]
	Angenommen $K,K'$ sind zwei $f$-Ketten und $K' \setminus K \neq \emptyset$.
	Dann ist $K \subseteq K'$ und es gilt $x \leq x'$ für alle $x \in K$ und $x' \in K' \setminus K$.\\
	Informell: \enquote{$K$ ist eine Anfangsabschrift von $K'$}.
\end{lemma}

\begin{proof}
	Sei $x' \in K' \setminus K$. Wir definieren 
	\[
	C = \{x \in K \cap K' : x \leq x'\} \subseteq K' \cap K
	\] 
	und verwenden, dass $K'$ eine $f$-Kette ist.
	Da $x' \in \widehat{C} \cap K'$ ist, folgt $f(\widehat{C}) \in K'$ und $f(\widehat{C}) \leq x'$.\\
	Falls $\widehat{C} \cap K \neq \emptyset$ wäre, so wäre $f(\widehat{C}) \in K$ (da $K$ eine $f$-Kette ist)
	womit aber $f(\widehat{C}) \in C \cap \widehat{C}$ der Definition von $\widehat{C}$ widerspricht.\\
	Also ist $\widehat{C} \cap K = \emptyset$. In anderen Worten bedeutet dies, dass es zu jedem $x \in K$ ein $c \in C$ mit $x \leq c$ geben muss.
	Nach Definition von $C$ folgt daher $x \leq c \leq x'$. Also $x \leq x'$ für alle $x \in K$.\\
	Unsere Annahmen an $K$ und $K'$ war bloss, dass es $x' \in K' \setminus K$ gibt.
	Daraus folgt nun auch $K \subseteq K'$, denn ansonsten könnten wir die Rollen von $K$ und $K'$ vertauschen.
\end{proof}

\begin{lemma}[Vereinigung]
Wir definieren $K_{max} = \bigcup_{\substack{K \text{ ist eine} \\f-\text{Kette}}} K$.
	Dann ist $K_{max}$ eine $f$-Kette.
\end{lemma}

\begin{proof}
	Da für je zwei Ketten $K,K'$ $K \subseteq K'$ oder $K' \subseteq K$ gilt, sehen wir, dass $K_{max}$ wieder eine Kette ist.
	Wir müssen noch zeigen, dass $K_{max}$ eine $f$-Kette ist und nehmen hierzu eine Teilmenge $C \subseteq K_{max}$
	mit  $\widehat{C} \cap K_{max} \neq \emptyset$.
	Sei $x' \in \widehat{C} \cap K_{max}$ und sei $K'$ eine $f$-Kette mit $x' \in K'$.

	\textbf{Behauptung:} $C \subseteq K'$.\\
	Sei also $x \in C$, dann existiert eine $f$-Kette $K$ mit $x \in K$.
	Nach obigem Lemma gilt $K \subseteq K'$ ($\implies x \in K' \checkmark$) oder $K' \subseteq K$. 
	Woraus folgt $K'$ enthält wegen obigem Lemma alle Elemente von $K$ unterhalbt von $x'$.
	Da $x' \in \widehat{C}$ und $x \in C$ folgt $x \leq x'$ und $x \in K'.$

	Da $C \subseteq K', c \in \widehat{C} \cap K'$ und da $K'$ eine $f$-Kette ist, folgt nun $f(\widehat{C}) \in K' \subseteq K_{max}$ und $f(\widehat{C}) \leq x'$.
	Da $x' \in \widehat{C} \cap K_{max}$ beliebig war, sehen wir, dass $f(\widehat{C}) \in K_{max}$ ein minimales Element von $\widehat{C} \cap K_{max}$ ist.
	Dies zeigt aber, dass $K_{max}$ eine $f$-Kette ist.
\end{proof}

\begin{proof}[Beweis vom Zornschen Lemma]
	$K_{max}$ ist nach Definition eine größte $f$-Kette in $X$.
	Insbesondere ist also das erste Lemma nicht anwendbar, d.h. $\widehat{K}_{max} = \emptyset$.

	Des Weiteren ist aber $K_{max}$ eine Teilkette, die nach Annahme an $\leq$ eine obere Schranke $x_{max}$ besitzen muss.
	Es folgt also $x_{max} \in K_{max}$ ist ein Maximum von $K_{max}$ und auch, dass $x_{max}$ ein maximales Element von $X$ ist.
	%TODO: a01.choice Image S12
\end{proof}



























