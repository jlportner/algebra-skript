%! TEX root=../algebra.tex
\graphicspath{{Images/}}

\chapter{Faktorisierungen von Ringen}
\emph{Buch Seiten 83-114}.
Wir wollen in diesem Kapitel Ringe mit eindeutiger Primfaktorzerlegung betrachten.
Im Folgenden ist $R$ immer ein Integritätsbereich.

\begin{definition}[Wiederholung]
	$a \mid b \Leftrightarrow \exists  c $ mit $b=ac$ für $a,b \in R$.\\
	$a \in R^{\times}$ ist eine Einheit $\Leftrightarrow a \mid 1$.
\end{definition}

\begin{definition}
	Wir sagen $p \in R \setminus \{0\}$ ist \emph{irreduzibel}, falls $p \not\in R^{\times}$ und für alle $a,b \in R$ gilt
	$p = ab \implies a \in R^{\times}$ oder $b \in R^{\times}$.
\end{definition}

\begin{definition}
	Wir sagen $p \in R \setminus \{0\}$ ist \emph{prim} falls $(p)$ ein Primideal ist, in anderen Worten
	falls $p \not\in R^{\times}$ und für alle $a,b \in R$ gilt $p \mid ab \implies p \mid a$ oder $p \mid b$.
\end{definition}

\begin{lemma}
	Sei $R$ ein Integritätsbereich. Dann ist jedes prim $p \in R$ auch irreduzibel.
\end{lemma}


\begin{remark}
	Die Umkehrung des Lemmas stimmt im Allgemeinen nicht.
	Wenn sie doch stimmt, so hilft dies für die Eindeutigkeit in einer Primfaktorzerlegung.
	Siehe später in 3.3.
\end{remark}

\section{Euklidische Ringe}
\begin{definition}
	Ein Integritätsbereich $R$ heißt ein \emph{Euklidischer Ring} falls es eine Gradfunktion $N: R \setminus \{0\}  \to \N$ gibt,
	so dass die beiden folgenden Eigenschaften gelten:
	\begin{itemize}
		\item \emph{Gradungleichung}: $N(f) \leq N(fg)$ für alle $f,g \in R \setminus \{0\} $.
		\item \emph{Division mit Rest}: Für $f,g \in R$ mit $f \neq  0$ gibt es $q,r \in R$ mit $g = q\cdot f + r$ wobei $r = 0$ oder $N(r) < N(f)$ ist.
			Wir nennen $r$ den \emph{Rest} (bei Division durch $f $).
	\end{itemize}
\end{definition}






\begin{theorem}
	In einem Euklidischen Ring ist jedes Ideal ein Hauptideal.
\end{theorem}


\section{Hauptidealring}
\begin{definition}
	Sei $R$ ein Integritätsbereich. Dann heißt $R$ ein \emph{Hauptidealring} falls jedes Ideal in $R$ ein Hauptideal ist.
\end{definition}


\begin{remark}
	Der Ring $\Z[\frac{1}{2}(1+i\cdot \sqrt{163})]$ ist ein Hauptidealring und kann nicht zu einem Euklidischen Ring gemacht werden.
\end{remark}

\begin{proposition}
	Sei $R$ ein Hauptidealring. Für je zwei Elemente $f,g \in R \setminus \{0\} $ gibt es einen größten gemeinsamen Teiler $d$ mit $(d) = (f) + (g)$.
\end{proposition}

\begin{definition}
	Seien $f,g,d \in R \setminus \{0\} $. Wir sagen $d $ ist ein gemeinsamer Teiler von $f$ und $g$ falls $d \mid f$ und $d \mid g$.
	Wir sagen $d$ ist ein größter gemeinsamer Teiler falls $d$ ein gemeinsamer Teiler ist und jeder gemeinsame Teiler $t$ auch $d$ teilt.
\end{definition}

\begin{remark}
	Zwei ggT's unterscheiden sich um eine Einheit (wenn $R$ ein Integritätsbereich ist).
\end{remark}


In einem Euklidischen Ring kann man einen ggT von $f,g \in R \setminus \{0\} $ durch den \emph{euklidischen Algorithmus} bestimmen.
\begin{enumerate}[1)]
	\setcounter{enumi}{-1}
	\item Falls $N(f) > N(g)$, so vertauschen wir $f$ und $g$. 
		Also dürfen wir annehmen, dass $N(f) \leq N(g)$.
	\item Dividiere $g$ durch $f$ mit Rest: $g = q f + r$
	\item Falls $r = 0$ ist, so ist $f$ ein ggT und der Algorithmus stoppt.
	\item Falls $r\neq 0$ ist, so ersetzen wir $(f,g)$ durch $(r,f)$ und springen nach $1)$.
\end{enumerate}

\begin{lemma}
	Der Euklidische Algorithmus (wie oben beschrieben) endet nach endlich vielen Schritten und berechnet einen ggT.
\end{lemma}


\begin{theorem}[Prime Elemente]
	Sei $R$ ein Hauptidealring.
	\begin{enumerate}[1)]
		\item Dann ist $p \in R \setminus \{0\} $ prim genau dann wenn $p$ irreduzibel ist.
		\item Jedes $f \in R \setminus \{0\} $ lässt sich als Produkt einer Einheit und endlich vielen primen Elementen schreiben.
	\end{enumerate}
\end{theorem}


\begin{theorem}
	Sei $R$ ein Hauptidealring und $p \in R$ irreduzibel.
	Dann ist $(p)$ ein Maximalideal. Insbesondere ist $p$ prim.
\end{theorem}


Für den Beweis vom Satz über Prime Elemente Eigenschaft 2 verwenden wir:
\begin{proposition}
	Sei $R$ ein Hauptidealring und seien $J_0 \subseteq J_1 \subseteq J_2 \subseteq \ldots$ eine aufsteigende Kette von Idealen in $R$.
	Dann gibt es ein $n \in \N$ mit $J_{m} = J_{n}$ für alle $m \geq n$.
\end{proposition}



%ZFE begin include
\begin{eg}
	Einige Primzahlen in $\Z[i]$, z.B. sind $1\pm i,3,2\pm i$ Primzahlen in $\Z[i]$.
	%TODO maybe das über Z[i] 09.faktor S5

	$2$ ist keine Primzahl in $\Z[i]$, da $2 = (1+i)(1-i)$.
	$5$ ist auch keine Primzahl in $\Z[i]$, da $5 = (2+i)(2-i)$.

	Nach dem ersten folgenden Lemma ergibt sich nun, dass $1 \pm i$, $2 \pm i$ Primzahlen in $\Z[i]$ sind.
	Nach dem zweiten Lemma sind $3,7$ Primzahlen in $\Z[i]$.
\end{eg}
%ZFE end include

\begin{lemma}
	Sei $z \in \Z[i]$ so dass $N(z) = p \in \N$ eine Primzahl in $\N$ ist.
	Dann ist $z$ irreduzibel (also prim) in $\Z[i]$.
\end{lemma}


\begin{lemma}
	Angenommen $p \in \N$ ist eine Primzahl in $\N$, die sich \emph{nicht} als Summe zweier Quadratzahlen schreiben lässt.
	Dann ist $p$ auch eine Primzahl in $\Z[i]$.
\end{lemma}



\section{Faktorielle Ringe}

\begin{definition}
	Ein Integritätsbereich $R$ heißt ein \emph{faktorieller Ring} falls jedes $a \in R \setminus \{0\}$ sich als ein Produkt von einer Einheit
	und endlich vielen Primelementen von $R$ schreiben lässt: $a = u \cdot  p_1 \cdot \ldots \cdot p_{m}$ für $u \in R^{\times}, m \in \N, p_1,\ldots p_{m} \in R$ prim.
\end{definition}


\begin{proposition}
	Sei $R$ ein faktorieller Ring.
	Dann ist $p \in R \setminus \{0\} $ irreduzibel gdw. $p$ prim ist.
\end{proposition}


\begin{corollary}
	Sei $R$ ein Integritätsbereich. Dann ist $R$ faktoriell gdw. jedes Element von $R \setminus \{0\} $ eine Zerlegung als ein Produkt
	von einer Einheit und endlich vielen irreduziblen Elementen besitzt und jedes irreduzible Element auch ein Primelement ist.
\end{corollary}

\begin{definition}
	Sei $R$ ein kommutativer Ring und $a,b \in R$.
	Wir sagen $a,b$ sind \emph{assoziiert} und schreiben $a \sim b$ falls es eine Einheit
	$u \in R^{\times}$ gibt mit $a = u b$.
\end{definition}

\begin{lemma}
	Dies definiert eine Äquivalenzrelation auf $R$.
\end{lemma}


\begin{lemma}
	Sei $R$ ein Integritätsbereich. Seien $p,q \in R \setminus \{0\} $ irreduzibel und $p \mid q$.
	Dann gilt $p \sim q$.
\end{lemma}


\begin{definition}[Wh.]
	Für $n \in \N_{> 0}$. sei $S_{n}$ die \emph{symmetrische Gruppe} auf der Menge $\{1,\ldots,n\}$, d.h.
	\[
	S_{n} = \{ \sigma: \{1,\ldots, n\} \to \{1,\ldots,n\} \text{ bijektiv}\} 
	.\] 
\end{definition}

\begin{theorem}[Eindeutige Primfaktorzerlegung]
	Sei $R$ ein faktorieller Ring, dann besitzt jedes nichttriviale Element von $R$ eine bist auf Permutation und Assoziierung 
	eindeutige Primfaktorzerlegung.

	Genauer gilt also für jedes $a \in R \setminus \{0\} $ gibt es eine Einheit $u \in R^{\times}$, $m \in \N$, und
	Primelemente $p_1,\ldots,p_{m}$ mit $a = u p_1 \ldots p_{m}$.\\
	Falls $a = v q_1 \ldots q_{n}$ eine weitere Zerlegung ist, wobei $v \in R^{\times}$, $n \in \N$ und $q_1,\ldots,q_{n}$ prim sind,
	dann gibt es $\sigma \in S_{n}$ so dass $q_{j} \sim p_{\sigma(j)}$ für $j = 1,\ldots,n$ und $m = n$.
\end{theorem}

	Die Existenz der Zerlegung ist die Definition von \enquote{faktorieller Ring}.
	Wir nennen $p_1,\ldots p_{m}$ die Primfaktorzerlegung von $a$.


\begin{definition}
	Sei $R$ ein faktorieller Ring.
	Wir sagen $P \subseteq R$ ist eine \emph{Repräsentantenmenge} (der Primelemente) falls jedes $p \in P$ ein Primelement in $R$ ist
	und es zu jedem Primelement $q \in R$ ein eindeutig bestimmtes $p \in P$ gibt mit $q \sim p$.
\end{definition}


\begin{lemma}
	Sei $R$ ein faktorieller Ring.
	Dann existiert eine Repräsentantenmenge.
\end{lemma}


\begin{theorem}[Eindeutige Primfaktorzerlegung]
	Sei $R$ ein faktorieller Ring und $P \subseteq R$ eine Repräsentantenmenge.
	Dann besitzt jedes $a \in R \setminus \{0\}$ eine eindeutige Primfaktorzerlegung
	der Form 
	\[
		a = u \prod_{p \in P} p^{n_p} \left[ = u \prod_{\substack{p \in P\\ n_{p} > 0}} p^{n_{p}}  \right]
	\] 
	wobei $n_{p} = 0$ für alle bis auf endlich viele $p \in P$.
\end{theorem}



\begin{lemma}
	Sei $R$ ein faktorieller Ring und $P \subseteq R$ eine Repräsentantenmenge.
	Sei $a = u \prod_{p \in P} p^{m_{p}}$ und $b = v \prod_{p \in P} p^{n_{p}}$.
	Dann gilt $a \mid b$ gdw. $m_{p} \leq n_{p}$ für alle $p \in P$.
\end{lemma}


\begin{proposition}[ggT]
	Sei $R$ ein faktorieller Ring mit Repräsentantenmenge $P$.
	Dann existiert für jedes Paar $a,b \in R$, nicht beide $0$,
	ein ggT. Falls $a = u \prod_{p \in P} p^{m_{p}}, b = v \prod_{p \in P} p^{n_{p}}$ ist, so ist
	$\prod_{p \in P} p^{\min(m_{p},n_{p})}$ ein ggT von $a$ und $b$.
\end{proposition}


Wir können analog den ggT von mehreren Elementen $a_1,\ldots, a_{l} \in R$ definieren und die obige Proposition gilt analog.

\begin{definition}
	Sei $R$ ein faktorieller Ring. Wir sagen $a_1,\ldots,a_{l} \in R$ sind \emph{coprim} falls $1$ ein ggT von
	$a_1,\ldots,a_{l}$ ist, oder äquivalenterweise falls es zu jedem Primelement $p$ in $R$ 
	ein $a_{j}$ gibt so dass $a_{j}$ nicht durch $p$ teilbar ist.
\end{definition}

\begin{corollary}
	Sei $R$ ein faktorieller Ring mit Quotientenkörper $K$.
	Dann hat jedes $x \in K$ eine Darstellung $x = \frac{a}{b}$ mit $a,b \in R$ coprim, $b \neq 0$.
\end{corollary}


\begin{corollary}
	Sei $R$ faktoriell und $K = \operatorname{Quot}(R)$. Dann hat jedes $x \in K$ eine Darstellung der Form 
	\[
	x = u \prod_{p \in P} p^{n_{p}}
	,\]
	wobei $n_{p} \in \Z$ und gleich $0$ für alle bis auf endlich viele $p \in P$ ist.
\end{corollary}


\section{Einige algebraische Euklidische Ringe}
Alle Beispiele, die wir hier betrachten wollen,leben in einem quadratischen Zahlenkörper:
$ K = \Q[\sqrt{d}] = \{a + b\sqrt{d} :a,b \in \Q \} $ mit $d \in \Z$, das kein Quadrat ist.
Isomorph dazu $\sfrac{\Q[x]}{(x^2-d)}$.

Wir definieren auf $K$ die Konjugation $\tau: K \to K, a+ b\sqrt{d} \mapsto a- b \sqrt{d}$.
Dies definiert einen Körperautomorphismus.


%TODO maybe copy slide 1

Auf $K$ definieren wir die Normfunktion
\[
	N(a+b \sqrt{d} ) = (a+b \sqrt{d} )(a-b \sqrt{d}) = a^2 - d b^2 
\]
so dass $N : K \to \Q$ multiplikativ ist, daher 
\[
	N(zw) = (zw) \underbrace{\tau(zw)}_{\tau(z) \tau(w)} = N(z) N(w) \qq{ für } z,w \in K
.\] 
Weiters $N(z) = 0 \Leftrightarrow z = 0$ für alle $z = a + b + \sqrt{d} \in K $.

Wir werden den Ring $R = \Z[\sqrt{d}]$ betrachten und wollen $\phi(z) = \abs{N(z)}$ als Gradfunktion verwenden.

\begin{theorem}
	Für $d = -1, -2, 2, 3$ ist  $R = \Z[\sqrt{d}]$ ein Euklidischer Ring, wobei wir $\phi(z) = \abs{N(z)}$ als Gradfunktion verwenden.
\end{theorem}


Sei $R = \Z[\sqrt{d}]$.
\begin{lemma}
	Es gilt $u \in R^{\times} \Leftrightarrow N(u) = \pm 1$.
\end{lemma}

\begin{lemma}
	Falls $z \in R$ eine Primzahl in $\Z$ als Norm hat, so ist $z$ in $R$ irreduzibel.
\end{lemma}

\begin{lemma}
	Falls $p \in \Z$ eine Primzahl in $\Z$ ist, so dass weder  $p$ noch $-p$ eine Norm von einem Element
	in $R$ ist, so ist $p$ ein irreduzibles Element in $R$.
\end{lemma}




\begin{theorem}[Gausssche ganze Zahlen]
	Sei $R = \Z[i]$ der Ring der Gausschen ganzen Zahlen.
	Dann ist $R$ ein Euklidischer Ring. Wir können in $R$ die Repräsentantenmenge
	\[
	p = \{z = a + ib \in R \mid z \text{ prim, } -a < b \leq a\} 
	\] 
	verwenden. Diese Menge $P$ enthält
	\begin{itemize}
		\item (Ramified): $z = 1+i$ mit $2 = -i (1+i)^2$
		\item (Inert): $p \in \N$ prim mit $p \equiv 3 \mod 4$,
			z.B. $3,7,11,\ldots$.
		\item (Split): $z = a \pm b i$ prim in $R$, wobei $a,b \in \N, b < a$ und
			$a^2 + b^2 = p = 1 \mod 4$ mit $p \in \N$ prim. $p = (a+ib)(a-ib)$ 
			z.B. $5,13,\ldots$
	\end{itemize}
\end{theorem}

\begin{lemma}
	Sei $p \in \N$ prim. Dann ist $(p-1)! \equiv -1 \mod p$ .
\end{lemma}


\begin{proposition}
	Sei $p \in \N$ kongruent $1 \mod 4$.
	Dann gibt es in  $\F_{p}$ zwei Lösungen der quadratischen Gleichung $x^2 = -1$.
\end{proposition}



\begin{corollary}
	Sei $p \in \N$ kongruent $1 \mod 4$. Dann ist $p$ keine Primzahl in $\Z[i]$.
\end{corollary}




\begin{theorem}
	Im $R_{falsch} = \Z[\sqrt{3}i]$ funktioniert Division mit Rest nicht wie in den obigen Fällen.
	Aber in $R_{richtig} = \Z[\zeta] = \{a+b \zeta : a,b \in \Z\} $ für $\zeta = \frac{1+\sqrt{3} i}{2}$ funktioniert dies wieder.
\end{theorem}


%TODO maybe übersicht 12.faktor S1


\section{Polynomringe}
\emph{Seite 108}

\begin{theorem}[Gauss]
	Falls $R$ ein faktorieller Ring ist, so ist auch $R[x]$ ein faktorieller Ring.
\end{theorem}

\begin{corollary}
	Der Ring $\Z[x_1,\ldots,x_{n}]$ und der Ring $K[x_1,\ldots,x_{n}]$ für einen Körper $K$ sind faktoriell,
\end{corollary}

\begin{definition}
	Sei $R$ ein faktorieller Ring und $f \in \R[x] \setminus \{0\}$.
	Dann nennen wir den ggT der Koeffizienten von $f$ den \emph{Inhalt $I(f)$ von $f$ }
	(welcher bis auf Einheiten in $R$ eindeutig bestimmt ist).

	Wir sagen $f$ ist \emph{primitv} falls $I(f) \sim 1$.
\end{definition}


\paragraph{Beobachtungen}
\begin{itemize}
	\item Jedes normierte Polynom ist primitiv.
	\item Für $a \in R \setminus \{0\}, f \in R[x] \setminus \{0\} $ gilt $I(af) \sim a I(f)$.
	\item Falls  $f \in R[x]$ irreduzibel ist, so ist entweder
		$f \in R$ oder $f$ ist primitiv.
		(Grad $f = 0 \implies f \in R$, Grad $f > 0 \implies f = a f^{*}, a \in R, f^{*}$ primitiv.
		Folgt $a$ oder $f^{*}$ ist eine Einheit $\implies$ $\deg(f^{*}) = \deg(f) > 0$ also $f^{*}$ ist keine Einheit)
\end{itemize}

\begin{lemma}
	Sei $R$ ein faktorieller Ring und $K = \operatorname{Quot}(R)$.
	Dann hat jedes  $f \in K[x] \setminus \{0\} $ eine Darstellung $f = d f^{*}$ 
	wobei $d \in K^{\times}$ und $f^{*} \in R[x]$ ist primitiv.
	Diese Darstellung ist bis auf Assoziierung eindeutig:\\
	Falls $f = d_1 f_1^{*} = d_2 f_2^{*}$, $d_1,d_2 \in K^{\times}$,
	$f_1^{*},f_2^{*} \in R[x]$ primitiv, dann ist $d_1 \sim_{R} d_2, f_1^{*} \sim_{R} f_2^{*}$.
	
	Wobei $\sim_{R}$  assoziiert über eine Einheit in $R$ bedeutet.
\end{lemma}


\begin{definition}
	Für $f \in K[x] \setminus \{0\} $ nennen wir das $d \in K^{\times}$ mit $f = d f^{*}, f^{*} \in R[x]$ primitiv, wieder den \emph{Inhalt von $f$}.
\end{definition}

\begin{proposition}[Gauss]
	Sei $R$ faktoriell. Für $f,g \in R[x]$ gilt $I(fg) \sim I(f) I(g)$.
	Insbesondere ist das Produkt von primitiven Elementen von $R[x]$ wieder primitiv.
\end{proposition}

\b
Im folgenden werden wir die \enquote{Reduktion der Koeffizienten} verwenden:
Für ein $p \in R$ gibt es einen Ringhomomorphismus $f \in R[x] \mapsto f_{\mod p} \in \sfrac{R}{(p)}[x], \sum_{i=0}^{n} a_{i} X^{i} \mapsto \sum_{i=0}^{n} (a_{i} + (p)) X^{i}$.
Dies folgt aus dem Satz von 4. VO %TODO proper reference,
(wobei $\varphi(a) = a + (p)$ und $\Phi(X) = X$).


\begin{theorem}[Gauss]
	Sei $R$ ein faktorieller Ring. Dann ist auch $R[x]$ faktoriell.
	Des Weiteren hat $R[x]$ genau die beiden Typen von Primelementen:
	\begin{itemize}
		\item $p \in R$ prim ist auch ein Primelement von $R[x]$.
		\item $f \in R[x]$ primitiv so dass $f$ irreduzibel als Element von $K[x]$ ist,
			ist ein Primelement von $R[x]$.
	\end{itemize}
\end{theorem}

\begin{corollary}
	Sei $f \in R[x]$ primitiv. Dann ist $f$ irreduzibel als Element von $R[x]$ gdw. $f$ ist irreduzibel als Element von $K[x]$.
\end{corollary}


\begin{lemma}
	Sei $K$ ein Körper und $a \in K$. Dann gilt für jedes $f \in K[x]$ 
	\[
		f(x) = (x-a)g(x) + r \qq{ für } g(x) \in K[x], r \in K
	.\]
	Daher gilt $f(a) = 0 \Leftrightarrow (x-a) \mid f(x)$.
\end{lemma}

\begin{proposition}
	Sei $K$ ein Körper. Dann sind lineare Polynome der Form $x-a$ für $a \in K$ irreduzibel als Elemente von $K[x]$.
	Für quadratische ($\deg(f) = 2$ ) und kubische ($\deg(f) = 3$) Polynome  $f \in K[x]$ gilt
	\begin{center}
		$f$ ist irreduzibel $\Leftrightarrow$ $f$ hat keine Nullstelle ($\forall a \in K$ gilt $f(a) \neq 0$ )
	\end{center}
\end{proposition}


\begin{theorem}[Fundamentalsatz der Algebra]
	Jedes Polynom $f \in \C[x]$ mit $\deg(f) > 0$ hat eine Nullstelle in $\C$.

	Die irreduziblen Elemente von $\C[x]$ sind genau die linearen Polynome.
	Insbesondere hat jedes $f \in \C[x]$ eine Faktorisierung in Linearfaktoren
	\[
		f(x) = a \prod_{j=1}^{\deg(f)} (x-z_{j})
	.\] 
	für gewisse $a \in C \setminus \{0\}$ und $z_1,\ldots,z_{\deg(f)} \in \C$.
\end{theorem}

\begin{corollary}[Fundamentalsatz für $\R$ ]
	Ein Polynom in $\R[x]$ ist irreduzibel gdw. entweder $\deg(f) = 1$ ist oder $\deg(f) = 2$ ist und $f$ keine Nullstellen in $\R$ besitzt.
\end{corollary}


\begin{proposition}
	Sei $R$ ein faktorieller Ring. Sei $f \in R[x]$ und $\frac{a}{b} \in K$ mit $b \neq 0, (a,b)$ coprim.
	Falls $f(\frac{a}{b}) = 0$ ist, so ist $b$ ein Teiler von führenden Koeffizienten von $f$ und $a$ ein Teiler vom konstanten Term von $f$.
\end{proposition}




\begin{proposition}
	Sei $R$ ein faktorieller Ring und $p \in R$ ein Primelement.
	Angenommen $f \in R[x]$ erfülle:
	\begin{itemize}
		\item $f$ primitiv
		\item $\deg(f) = \deg(f_{\mod p})$ mit $f_{\mod p} \in \sfrac{R}{(p)}[x]$
		\item $f_{\mod p} \in \frac{R}{(p)} [x]$ ist irreduzibel
	\end{itemize}
	Dann ist $f \in R[x]$ ein Primelement.
\end{proposition}



\begin{theorem}[Eisenstein-Kriterium]
	Sei $R$ ein faktorieller Ring und $p \in R$ ein Primelement.
	Sei $f(x) = \sum_{i=0}^{n} a_{i} x^{i}$  primitiv mit $n \geq 1, p \nmid a_{n}$, $p \mid a_{i}$ für $i = 0,\ldots,n-1$ und $p^2 \nmid a_0$.
	Dann ist $f$ irreduzibel.
\end{theorem}



\begin{corollary}
	Für jede Primzahl $p \in \N$ ist das $p$-te Kreisteilungspolynom
	\[
		\Phi_{p}(x) = 1 + x+ x^2 + \ldots + x^{p-1} = \frac{x^{p} -1}{x-1}
	\] 
	in $\Z[x]$ irreduzibel.
\end{corollary}



\begin{remark}
	Für $p \in \N$ prim gilt allerdings
	\[
		(x+y-z)^{p} = x^{p} + y^{p} - z^{p} \in \F_{p}[x,y,z]
	.\]
	nicht irreduzibel.
\end{remark}


















