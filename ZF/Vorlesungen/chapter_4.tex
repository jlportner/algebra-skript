%! TEX root=../algebra.tex
\graphicspath{{Images/}}

\chapter{Gruppentheorie}
\section{Definition und Beispiele}

\begin{definition}
	Eine Menge $G$ gemeinsam mit einer Abbildung $\cdot : G \times G \to G$ heißt eine Gruppe falls folgende Axiome erfüllt sind:
	\begin{enumerate}[1)]
		\item Assoziativität: $\forall a,b \in G: (a \cdot  b) \cdot  c = a \cdot (b \cdot c)$
		\item Einheit: $\exists e \in G \forall a \in G: e \cdot a = a \cdot e = a$
		\item Inverse: $\forall a \in G \exists x \in G: a \cdot  x = x \cdot a = e$ (wobei $e$ wie in 2) ist)
	\end{enumerate}
\end{definition}

\begin{lemma}
	Sei $G$ eine Gruppe.  Die Einheit $e$ wie in 2) ist eindeutig bestimmt durch $e \cdot  a = a$
	für alle $a \in G$, oder  auch durch $e \cdot e = e$. Für jedes $a \in G$ ist die Inverse $x \in G$ durch
	$a \cdot x = e$ eindeutig bestimmt, wie schreiben $a^{-1} = x$.
	Insbesondere gilt $e^{-1} = e$, $(a^{-1})^{-1} = a$ und $(ab)^{-1} = b^{-1} a^{-1}$ für alle $a,b \in G$.
\end{lemma}

\begin{remark}
	Wir bezeichnen die Einheit auch als das Einselement und schreiben $e = e_{G} = 1 = 1_{G}$.
\end{remark}


\begin{definition}
	Sei $G$ eine Gruppe und $a,b \in G$. Falls $ab = ba$ gilt, so sagen wir, dass $a$ und $b$ kommutieren.
	Falls alle Paare in $G$ \emph{kommutieren}, so heißt $G$ \emph{kommutativ} oder auch \emph{abelsch}.
\end{definition}

\begin{remark}
	Für abelsche Gruppen verwenden wir manchmal auch additive Notation $+ : G \times G \to G$.
\end{remark}

\begin{definition}
	Für eine Gruppe $G$ und $a \in G$ definiere wir die Potenzen von $a$ durch
	\[
	a^{k} := \begin{cases}
		\underbrace{a \cdot \ldots \cdot a}_{k-\text{fache}} &\text{für } k > 0\\
		e &\text{für } k=0\\
		\underbrace{a^{-1} \cdot \ldots \cdot ^{-1}}_{\abs{k}-\text{fache}} &\text{für } k < 0
	\end{cases}
	\qq{für alle} k \in Z
	.\] 
\end{definition}

\begin{lemma}[Potenzregel]
	\begin{enumerate}[a)]
		\item $a^{k} a^{l} = a^{k+l}$ für $k \in \Z$.
		\item $(a^{k})^{l} = a^{k l}$ für  $k \in \Z$.
		\item Falls $a,b \in G$ kommutieren so kommutieren auch $a^{k}$ und $b^{l}$ und es gilt $(ab)^{k} = a^{k} b^{k}$.
	\end{enumerate}
\end{lemma}


\begin{lemma}[Gleichungen und Kürzen]
	Für alle $a,b \in G$ existiert ein eindeutig bestimmtes $x \in G$ mit $ax = b$, nämlich $x = a^{-1} b$.
	Für alle $a,b,c \in G$ gilt $a=b \Leftrightarrow ac = bc \Leftrightarrow ca = cb$.
\end{lemma}


\begin{definition}
	Angenommen $G_1, G_2$ sind Gruppen.
	Ein \emph{Homomorphismus} von $G_1$ nach $G_2$ ist eine Abbildung $\varphi: G_1 \to G_2$ mit
	$\varphi(ab) = \varphi(a) \varphi(b)$ für alle $a,b \in G$.
	Wir definieren den \emph{Kern} $\ker(\varphi) = \varphi^{-1} \{e_{G_2}\} = \{a \in G \mid \varphi(a) = e_{G_2}\} $
	und das \emph{Bild} $\Im(\varphi) = \varphi(G_1) = \{b \in G_2 \mid \exists a \in G \text{ mit } \varphi(a) = b\} $.
	Falls $\varphi$ bijektiv ist, so sprechen wir auch von einem \emph{Isomorphismus} der Gruppen
	und sagen $G_1$ und $G_2$ sind \emph{isomorph }.
\end{definition}

\begin{definition}
	Sei $G$ eine Gruppe. Eine Untergruppe von $G$ ist eine nichtleere Teilmenge $H \subseteq G$ mit $a b^{-1} \in H$ für alle $a,b \in H$. Wir schreiben $H < G$.
\end{definition}

\textbf{Übung.} Sei $G$ eine Gruppe und $H \subseteq G$. Äquivalent sind:
\begin{enumerate}[1)]
	\item $H$ ist eine Untergruppe
	\item $e \in H$, und $a,b \in H \implies ab \in H$ und $a^{-1} \in H$ 
	\item $H$ ist eine Gruppe und $\iota: H \to G$ ist ein Homomorphismus.
\end{enumerate}
Falls $\abs{H} < \infty$, so ist auch folgende Aussage mit obigen Aussagen äquivalent:
\begin{enumerate}[1)]
	\setcounter{enumi}{3}
	\item $H$ ist nichtleer, und $a,b \in H \implies ab \in H$.
\end{enumerate}



\begin{lemma}
	Sei $G$ eine Gruppe und $a \in G$. Dann definiert $k \in \Z \mapsto a^{k} \in G$ einen Gruppenhomomorphismus.
	Entweder ist $\varphi$ injektiv oder es gibt ein $n_0 > 0$ mit $\ker(\varphi) = (n_0) = \Z n_0$.
\end{lemma}

\begin{definition}
	Falls $\varphi$ wie im Lemma injektiv ist, so sagen wird, dass $a$ \emph{unendliche Ordnung} hat.
	Falls $\ker(\varphi) = (n_0)$ mit $n_0 > 0$ ist, so sagen wir, dass $a$ \emph{Ordnung $n_0$} hat.
\end{definition}



\section{Konjugation}

\begin{lemma}
	Sei $G$ eine Grupee.
	\begin{enumerate}[a)]
		\item Für jedes $g \in G$ ist $\gamma_{g}: G \to G, x \mapsto g x g^{-1}$ ein Automorphismus von $G$, welche ein \emph{innerer Automorphismus} genannt wird.
		\item Die Abbildung $g \in G \mapsto \gamma_{g} \in \aut(G)$ ist ein Homomorphismus.
			Der Kern von $\Phi$ ist das Zentrum $Z_{G} = \{g \in G \mid gx = xg \forall x \in G\} $.
	\end{enumerate}
\end{lemma}


\begin{definition}
	Sei $G$ ein Gruppe und $g \in G$. Dann ist die Menge der Fixpunkte $\gamma_{g}$ gleich dem Zentralisator von $g$ :
	\[
		\operatorname{Cent}_{g} = \{x \in G \mid gx = xg\} 
	.\] 
\end{definition}

\begin{definition}
	Sei $G$ eine Gruppe und $x,y \in G$.
	Wir sagen $x,y$ sind \emph{zueinander konjugiert}, falls es ein $g \in G$ mit $g x g^{-1} = y$.
\end{definition}

\begin{lemma}
	\enquote{Konjugiert sein} definiert eine Äquivalenzrelation auf jeder Gruppe.
\end{lemma}



Manchmal ist $G$ sehr kompliziert und unüberschaubar aber die Konjugationsklassen sind einfacher zu verstehen.



\section{Untergruppen und Erzeuger}
\textbf{Wiederholung:}
$H\subseteq G$ nichtleer ist eine \emph{Untergruppe} ($H<G$ ) falls für alle $a,b \in H$ gilt $ab^{-1} \in H$.



\begin{lemma}
	Eine Untergruppe von einer Untergruppe ist eine Untergruppe.
\end{lemma}

\begin{lemma}
	Sei $G$ eine Gruppe und $I$ eine Menge und $H_{i} < G$ für jedes $i \in I$.
	Dann ist $\bigcap_{i \in I} H_{i} < G$.
\end{lemma}

\begin{definition}
	Sei $G$ eine Gruppe und $X \subseteq G$ eine Teilmenge. Die Untergruppe, die von $X$ erzeugt wird ist definiert als
	\[
		\left< X \right> = \bigcap_{\substack{H < G\\ X \subseteq H}} H
	.\] 
	Wir nennen $X$ die \emph{Erzeugendenmenge} von $\left< X \right>$. Falls $\left< X \right> = G$ sagen wir, dass $G$ \emph{durch $X$ erzeugt} wird.
	Falls $X = \{g\}$ dann nennen wir $\left< X \right> = \left< g \right>$ die von $g$ erzeugte \emph{zyklische Untergruppe} von $G$. 
\end{definition}

\begin{lemma}
	Sei $G$ eine Gruppe und $X \subseteq G$. Dann ist $\left< X \right> = 
	\{x_{1}^{\epsilon_1}\ldots x_{n}^{\epsilon_{n}} \mid n \in \N, x_1,\ldots,x_{n} \in X, \epsilon_1,\ldots,\epsilon_{n} \in \{\pm 1\} \} $.
\end{lemma}


\begin{lemma}
	Sei $G$ eine Gruppe und $a \in G$. Dann gilt $\left< a \right> \cong \sfrac{\Z}{(n_0)}$ für ein $n_0 \in \N$.
\end{lemma}



\begin{remark}
	Es gibt keinen \enquote{Basis- oder Diemensionsbegriff}:
	Denn is $S_{6}$ gibt es eine Untergruppe, die von $3$ oder mehr Elementen erzeugt wird, aber nicht von weniger:
	\[
	H = \left< \tau_{1,2}, \tau_{3,4}, \tau_{5,6} \right> \cong \F_{2}^{3}
	.\] 
\end{remark}

\begin{definition}
	Sei $G$ eine Gruppe. Der \emph{Kommutator} von $a,b \in G$ ist 
	\[
		[a,b] = ab a^{-1} b^{-1}
	.\]
	Die \emph{Kommutatorgruppe} ist
	\[
		[G,G] = \left< [a,b]: a,b \in G \right>
	.\] 
\end{definition}

\section{Nebenklassen und Quotienten}

\begin{definition}
	Sei $G$ eine Gruppe und $H < G$.
	Wir definieren zwei Relationen auf $G$ 
	\[
		a \sim_{H} b \Leftrightarrow b^{-1} a \in H \qquad a \prescript{}{H}{\sim} b \Leftrightarrow b a^{-1} \in H
	.\] 
	Wir nennen die Menge $a H = \{a h \mid h \in H\} $ die Linksnebenklasse mit Linksrepräsentanten $a$ und schreiben auch
	\[
	\sfrac{G}{H} = \{a H \mid a \in G\} 
	.\] 
	Außerdem nennen wir die Menge $H a = \{h a \mid h \in H\} $ die Rechtsnebenklasse mit Rechtsrepräsentanten $a$ und schreiben 
	\[
	\sfrac{H}{G} = \{H a \mid a \in G\} 
	.\] 
\end{definition}

\begin{lemma}
	Sei $G$ eine Gruppe und $H < G$. Dann ist $\sim_{H}$ eine Äquivalentrelation und $[ a ]_{\sim_{H}}$ und $G / H$ ist der Quotient von $G$ bzgl. $\sim_{H}$.
	Dies gilt analog für $\prescript{}{H}{\sim}$
\end{lemma}



\begin{theorem}
	Sei $G$ eine Gruppe und $H < G$.
	\begin{enumerate}[(1)]
		\item $\sfrac{G}{H}$ und $\sfrac{H}{G}$ sind (auf natürliche Weise) gleichmächtig.
		\item {[Lagrange]} Falls $\abs{G} < \infty$, dann gilt $\abs{G} = \abs{\sfrac{G}{H}} \cdot \abs{H}$.
			Insbesondere gilt $\abs{H}$ ist ein Teiler von $\abs{G}$.
	\end{enumerate}
\end{theorem}

\begin{definition}
	Die Kardinalität von $G$ wird auch die \emph{Ordnung} von $G$ genannt.
	Die Kardinalität von $\sfrac{G}{H}$ wird der \emph{Index $[G:H]$} von $H$ in $G$ genannt.
\end{definition}


\begin{corollary}
	Sei $G$ eine endliche Gruppe und $g \in G$. Dann teilt die Ordnung von $g$ die Ordnung von $G$.
	Des Weiteren gilt $g^{\abs{G}} = e$.
\end{corollary}


\begin{corollary}
	In $\F_{p} = \sfrac{\Z}{(p)}$ gilt $a^{p-1} = \begin{cases}
		0 &a = 0\\
		1 &\text{für alle } a \in \F_{p}^{\times}
	\end{cases}$
\end{corollary}


\begin{corollary}[Erste Klassifikation von Gruppen]
	Sei $G$ eine endliche Gruppe und $\abs{G} = p \in \N$ prim. Dann ist $G$ isomorph zu $\sfrac{\Z}{(p)}$.
\end{corollary}


$\implies$ Es gibt bis auf Isomorphie nur eine Gruppe der Ordnung $2,3,5,7,\ldots$.

Im Allgemeinen haben $\sfrac{G}{H}$ und $\sfrac{H}{G}$ keine natürliche Gruppenstruktur.

\begin{theorem}
	Sei $G$ eine Gruppe und $H < G$. Die folgenden Bedingungen sind äquivalent
	\begin{enumerate}[(1)]
		\item Für alle $x \in G$ ist $x H = H x$.
		\item Für alle  $x \in G$ ist $x H x^{-1} = H$.
		\item Es existiert eine gruppe $G_1$ und ein Gruppenhomomorphimus $\varphi: G \to G_1$ mit $H = \ker(\varphi)$.
		\item Für alle $x,y \in G$ gilt $(xH)(yH) = (xy) H$.
		\item $\sfrac{G}{H}$ ist (auf natürliche Weise) eine Gruppe so dass $\varphi: G \to \sfrac{G}{H}, g \mapsto g H$ ein Gruppenhomomorphismus ist.
	\end{enumerate}
\end{theorem}


\begin{definition}
	Sei $G$ eine Gruppe und $H < G$.
	Wir sagen $H$ ist \emph{normal} in $G$ oder ein \emph{Normalteiler} von $G$ falls $H$ die Bedingungen in obigem Satz erfüllt.
	Wir schreiben in diesem Fall auch $H \lTri G$.
	Falls $H \lTri G$ so nennen wir $\sfrac{G}{H}$ die \emph{Faktorgruppe} von $G$ modulo $H$.
\end{definition}

\begin{definition}
	Sei $G \neq \{e\} $ eine Gruppe. Wir sagen $ G$ ist \emph{einfach} falls $G$ nur $\{e\} $ und $G$ als Normalteiler besitzt.
\end{definition}



\begin{theorem}[Erster Isomorphiesatz]
	Sei $\varphi: G \to H$ eine Homomorphismus zwischen zwei Gruppen $G$ und $H$.
	Dann induziert $\varphi$ einen Isomorphismus $\abs{\varphi}: \sfrac{G}{\ker(\varphi)} \to  \Im(\varphi)$ so dass 
	folgendes Diagram komutiert
	\[
	\begin{tikzcd}
		G \arrow[d, "\pi"'] \arrow[r, "\varphi"]                 & H                                    \\
	\sfrac{G}{\ker(\varphi)} \arrow[r, "\overline{\varphi}"] & \Im(\varphi) < H \arrow[u, "\iota"']
		\end{tikzcd}
	\] 
	mit $\pi$ als der kanonischen Projektion und $\iota$ der Einbettung.
	Also gilt $\varphi = \iota \circ \overline{\varphi} \circ \pi$.
\end{theorem}



\begin{corollary}[Zweiter Isomorphiesatz]
	Sei $G$ eine Gruppe, $H \lTri G$. und $K < G$.
	Dann gilt $K H = H K < G, H \lTri KH, H \cap K \lTri K $ und 
	\[
	\sfrac{K}{H \cap K} \cong \sfrac{KH}{H}
	.\] 
	mit $x H \cap K \leftrightarrow x H$ für $x \in K$
\end{corollary}


\textbf{Übung:}
 Das Produkt von zwei Untergruppen ist im Allgemeinen keine Untergruppen.
 Das Produkt von zwei normalen Untergruppen ist eine normale Untergruppe.

 \begin{corollary}[Dritter Isomorphiesatz]
 	Sei $G$ eine Gruppe, $H \lTri G$, $K \lTri G$ und $K < H$.
	Dann ist $\sfrac{H}{K} \lTri \sfrac{G}{K}$ und es gilt
	\[
	\sfrac{\sfrac{G}{K}}{\sfrac{H}{K}} \cong \sfrac{G}{H}
	\] 
	wobei $(xK) \, \sfrac{H}{K} \; \widehat{=} \; x H$ einander im Isomorphismus entsprechen.
 \end{corollary}


\begin{corollary}
	Sei $G$ eine Gruppe und $H \lTri G$.
	Für eine beliebige weitere Gruppe $K$ gibt es eine natürliche Bijektion zwischen 
	\[
		\hom(\sfrac{G}{H},K) = \{ \varphi: \sfrac{G}{H} \to K \text{ Homomorphismus}\} \qq{und} \{\varphi: \hom(G,K) \mid \varphi \vert_{H} \equiv e_{K}\} 
	.\] 
\end{corollary}

\begin{corollary}
	Sei $G$ eine Gruppe und $H \lTri G$.
	Dann sind die folgenden beiden Abbildungen invers zueinander:
	\[
		(K < G \text{ mit } H < K) \mapsto  \sfrac{K}{H} < \sfrac{G}{H} \qq{und}(\pi^{-1}(\overline{K}) < G \text{ mit } H < \pi^{-1}(\overline{K})) \mapsfrom \overline{K} < \sfrac{G}{H}
	.\] 
\end{corollary}


\textbf{Übung:}
Sei $G$ eine Gruppe und $H < G$ mit Index $2$.
Dann gilt $H \lTri G$.

\textbf{Übung:} Klassifizieren/Beschreiben Sie alle Gruppen der Ordnung $\leq 7$ / $\leq 8$ / $\leq 10$.

\section{Gruppenwirkungen}

\begin{definition}
	Sei $G$ eine Gruppe und $T$ eine Menge.
	Eine \emph{Gruppenwirkung} (Linkswirkung, Linksaktion) von $G$ auf $T$ ist eine Abbildung $\cdot: G \times T \to T, (g,t) \mapsto g \cdot t$, so dass
	\begin{itemize}
		\item $e\cdot t = t$ für $t \in T$ 
		\item $g_1 \cdot (g_2 \cdot t) = (g_1 g_2) \cdot t$ für $g_1,g_2 \in G$ und $t \in T$.
	\end{itemize}
	Wir sagen in diesem Fall auch kurz, dass $T$ eine \emph{$G$-Menge} ist.
\end{definition}

\begin{remark}
	Obige Definition können wir äquivalent auch in folgender Form formulieren:\\
	Es gibt einen Gruppenhomomorphismus $\alpha: G \to \bij(T), g \in G \mapsto \alpha_{g}$.

	Der Zusammenhang zur obigen Definition ergibt sich durch die Formel $\alpha_{g}(t) = g \cdot t$
\end{remark}


\begin{definition}
	Sei $G$ eine Gruppe und $T$ eine $G$-Menge.
	\begin{itemize}
		\item $S \subseteq T$ heißt \emph{invariant} falls $g \cdot S = S$ für alle $g \in G$.
		\item $t_0 \in T$ heißt \emph{Fixpunkt} falls $g \cdot t_0 = t_0$ für alle $g \in G$.
			Die Menge der Fixpunkte wird mit $\operatorname{Fix_{G}}(T) = \{t_0 \in T \mid t_0 \text{ ist ein Fixpunkt}\} $ bezeichnet.
		\item Für $t_0 \in T$ wird $G \cdot t_0 = \{g \cdot t_0 : g \in G\}$ als die \emph{Bahn ($G$-Bahn)} bezeichnet.
		\item Für $t_0 \in T$ heißt $\operatorname{Stab}_{G}(t_0) = \{g \in G \mid g \cdot t_0 = t_0\}$ der \emph{Stabilisator von $t_0$}.
		\item Falls $g \in G \mapsto \alpha_{g} \in \bij(T)$ wie in obiger Bemerkung injektiv ist, so heißt die Gruppenwirkung \emph{treu}.
		\item Die Gruppenwirkung heißt \emph{transitiv} falls es zu jedem Paar $t_1,t_2 \in T$ ein $g \in G$ mit $g \cdot  t_1 = t_2$ gibt.
			Die Gruppenwirkung heißt \emph{scharf transitiv} falls es zu jedem Paar $t_1,t_2 \in T$ genau ein $g \in G$ mit $g \cdot t_1 = t_2$ gibt.
		\item Die Menge der $G$-Bahnen wird mit $G \setminus T = \{G \cdot t_0 \mid t_0 \in T\} $ bezeichnet.
	\end{itemize}
\end{definition}

%TODO check all sfracs before for right orientation and if that is even important

\begin{lemma}
	Sei $G$ eine Gruppe und $T$ eine $G$-Menge.
	Dann definiert $t_1 \sim_{G} t_2 \Leftrightarrow \exists g \in G$ mit $g \cdot t_1 = t_2$ eine Äquivalenzrelation auf $T$. Die Bahnen sind genau die Äquivalenzklassen
	und $\sfrac{G}{\sim_{G}} = G \setminus T$ ist der Quotientenraum.
\end{lemma}


\begin{definition}
	Sei $G$ eine Gruppe und $T_1, T_2$ zwei $G$-Mengen.
	Ein $G$-Morphismus von $T_1$ nach $T_2$ ist eine Abbildung $f: T_1 \to T_2$ mit
	\[
		f(g \underbrace{\cdot }_{\text{in } T_1} t) = g \underbrace{\cdot}_{\text{in } T_2} f(t)
	\]
	für alle $t \in T_1$ und $g \in G$.
	$g$ ist ein \emph{$G$-Isomorphismus} falls $f$ zusätzlich bijektiv ist.
\end{definition}

\begin{theorem}[Satz (über Bahnen und Stabilisator)]
	Sei $G$ eine Gruppe und $T$ eine $G$-Menge.
	Sei $t_0 \in T$, $T_0 = G \cdot t_0$ und $H = \on{Stab}_{G}(t_0)$.
	Dann ist $H < G$, $T_0$ ist invariant und 
	\[
		f: \sfrac{G}{H} \to T_0, gH \mapsto g\cdot t_0
	\]
	ist ein wohldefinierter $G$-Isomorphismus.
	In diesem Satz ist also die Bahn isomorph zu $G$ modulo Stabilisator.
\end{theorem}


\begin{corollary}
	Sei $G$ eine Gruppe und $T$ eine $G$-Menge. Falls $\abs{G} < \infty$, dann gilt 
	\[
		\abs{G} = \abs{G \cdot t_0} \cdot \abs{\on{Stab}_{G}(t_0)}
	\]
\end{corollary}


\begin{corollary}
	Sei $G$ eine Gruppe und $T$ eine endliche $G$-Menge. Dann gilt
	\[
		\abs{T} = \abs{\on{Fix}_{G}(T)} + \sum_{\abs{G\cdot t} > 1} [G: \on{Stab}_{G}(t)]
	,\] 
	also die summe über die nicht trivialen Bahnen.
\end{corollary}


\begin{theorem}[Cayley]
	Sei $G$ eine endliche Gruppe.
	Dann ist $G$ isomorph zu einer Untergruppe einer symmetrischen Gruppe $S_{n}$ für $n \in \N$.
\end{theorem}


\begin{remark}
	Falls $H < G$ mit endlichem Index, so gibt es einen Homomorphismus $\alpha: G \to S_{n}$ mit $n = [G:H]$ und 
	$\ker(\alpha) < H$.
\end{remark}

\section{Nilpotente und auflösbare Gruppen}

\begin{definition}
	Sei $G$ eine Gruppe. Wir sagen $G$ ist \emph{nilpotent mit Nilpotenzgrad $1$} falls $G$ abelsch ist.
	Wir sagen $G$ ist nilpotent mit \emph{Nilpotenzgrad $n+1$} (für $n \in \N_{\geq 1}$) falls
	$\sfrac{G}{Z_{G}}$ nilpotent mit Nilpotenzgrad $n$ ist.

	Wir sagen $G$ ist \emph{nilpotent} falls es ein $n \in \N$ gibt so dass $G$ nilpotent mit Nilpotenzgrad $n$ ist.
\end{definition}


\begin{definition}
	Sei $G$ eine Gruppe und $p \in \N$ eine Primzahl.
	Wir sagen $G$ ist eine $p$-Gruppe falls $\abs{G} = p^{k}$ für ein $k \in \N$.
\end{definition}

\begin{lemma}[Fixpunkte von $p$-Gruppen]
	Sei $p \in \N$ eine Primzahl und $G$ eine $p$-Gruppe.
	Sei $T$ eine $G$-Menge. Dann gilt $\abs{\on{Fix}_{G}(T)} \equiv \abs{T} \mod p$.
\end{lemma}


\begin{theorem}
	Eine $p$-Gruppe ist nilpotent.
\end{theorem}


\begin{corollary}
	Sei $p \in \N$ eine Primzahl und $G$ eine Gruppe mit $\abs{G} = p^2$.
	Dann ist $G$ abelsch.
\end{corollary}


\begin{definition}
	Sei $G$ eine Gruppe. Eine \emph{Subnormalreihe in $G$} ist eine Folge von Untergruppen so dass
	 \[
		 \{e\} = G_0 \lTri G_1 \lTri G_2 \lTri \ldots \lTri G_{n} = G
	\] 
	jede Untergruppe in der nächsten normal ist.
\end{definition}

\begin{definition}
	Sei $G$ eine Gruppe. Wir sagen $G$ ist \emph{auflösbar} falls es eine Subnormalreihe in $G$ (wie oben) gibt, so dass
	$\sfrac{G_{k+1}}{G_{k}}$ eine abelsche Gruppe (für $k = 0, \ldots, n-1$) ist.
\end{definition}


\begin{proposition}
	Sei $G$ eine Gruppe. Dann ist $[G,G] = \left< \{ [a,b] \mid a,b \in G \} \right> \lTri G$, und $\sfrac{G}{[G,G]}$ ist abelsch.
	Falls $H$ eine ablesche Gruppe ist und $\varphi: G \to H$ ein Homomorphismus ist, so ist $\varphi([G,G]) = \{e_{H}\}$
	und $\varphi$ induziert einen Gruppenhomomorphismus $\overline{\varphi}: \sfrac{G}{[G,G]} \to H$.
	In diesem Sinne ist $\sfrac{G}{[G,G]}$ die größte abelsche Faktorgruppe von $G$.
\end{proposition}


\begin{proposition}
	Sei $G$ eine Gruppe. Dann ist $G$ auflösbar genau dann wenn die folgende induktiv definierten
	höheren Kommutatorgruppen nach endlich vielen Schritten die triviale Untergruppe $\{e\}$ erreicht:
	\begin{align*}
		&G^{(0)} = G \\
		&G^{(1)} = [G^{(0)}, G^{(0)}] \;(\text{Kommutatorgruppe})\\
		&G^{(2)} = [G^{(1)}, G^{(1)}] \;(\text{2. Kommutatorgruppe})\\
		&\;\;\vdots \\
		&G^{(n+1)} = [G^{(n)}, G^{(n)}]
	\end{align*}
\end{proposition}


\section{Satz von Sylow}
Für eine endliche Gruppe $G$ besagt der Satz von Lagrange, dass für $H < G$ sowohl die Ordnung $\abs{H}$ als auch der Index $[G:H]$ Teiler von  $\abs{G}$ sind.

\begin{theorem}[Sylow]
	Sei $G$ eine endliche Gruppe, $p \in \N$ prim und $n = \abs{G} = p^{k} m$ für $k \geq 1$ und $m$ teilerfremd zu $p$.
	\begin{enumerate}[1)]
		\item Es existiert eine maximale $p$-Untergruppe $H_{p}$ mit $\abs{H_{p}} = p^{k}$, welche \emph{Sylow $p$-Untergruppen} genannt werden.
		\item Falls $H < G$ eine $p$-Untergruppe ist, so existiert eine $p$-Sylow Untergruppe $H_{p}$ mit $H < H_{p}$.
		\item Je zwei Sylow $p$-Untergruppen sind konjugiert.
	\end{enumerate}
\end{theorem}

\begin{lemma}
	Sei $p \in \N$ prim, $n = p^{k}m$ mit $m$ teilerfremd zu $p$.
	Dann ist $\binom{n}{p^{k}}$ nicht durch $p$ teilbar.
\end{lemma}





\section{Symmetrische und Alternierende Gruppen}

\begin{definition}
	Sei $n \geq 1$ natürlich, dann ist $S_{n} = \bij(\{1,\ldots,n\})$.
	Die Elemente von $S_{n}$ heißen \emph{Permutationen}.
\end{definition}

\begin{theorem}
	Sei $n \geq 1$. Auf $S_{n}$ gibt es einen Homomorphismus $\sgn: S_{n} \to \{\pm 1\}$, der jeder Permutation ein \emph{Vorzeichen} zuordnet
	und einer \emph{Vertauschung} $\tau_{ij}$ für $i \neq j$ das Vorzeichen $-1$ mit
	\[
		\tau_{ij}(k) = \begin{cases}
			i &\text{ für } k = j\\
			j &\text{ für } k = i\\
			k &\text{ sonst}
		\end{cases}
	.\] 
\end{theorem}

\begin{definition}
	$\sigma \in S_{n}$ heißt \emph{gerade} falls $\sgn(\sigma) = 1$, ungerade falls $\sgn(\sigma) = -1$.
	Die \emph{alternierende Gruppe} $A_{n} = \ker(\sgn)$ ist die Gruppe aller geraden Permutationen.
\end{definition}


\begin{notation}[für $\sigma \in S_{n}$]
	\[
	\sigma = \begin{pmatrix} 
		1 & 2 & \ldots & n\\
		\sigma(1) & \sigma(2) & \ldots & \sigma(n)
	\end{pmatrix} 
	.\] 
\end{notation}
Besser:
\begin{notation}[mittels Zyklen für $\sigma \in S_{n}$]
	Falls $\sigma = \id$ schreiben wir einfach $\sigma = \id$.
	Sei nun $\sigma \neq \id$ und $i_1 \in \{1,\ldots,n\} $ der erste Nichtfixpunkt (also $i_{1}$ minimal mit $\sigma(i_1) \neq i_1$ ).
	Wir bestimmen
	\[
		\sigma(i_1) , \sigma^2(i_1),\ldots, \sigma^{k_1}(i_1) = i_1 \qq{für $k_1 > 1$ minimal}
	.\] 
	Falls dies alle Nichtfixpunkte von $\sigma$ sind, so nennen wir $\sigma$ einen \emph{($k$-)Zyklus} und schreiben
	 \[
		 \sigma = (i_1,\sigma(i_1),\sigma^2(i_1),\ldots,\sigma^{k-1}(i_1))
	.\] 
	Falls nicht, so sei $i_2 > i_1$ der nächste Nichtfixpunkt (der noch nicht gefunden wurde) und bestimme
	\[
		i_2, \sigma(i_2),\ldots, \sigma^{k_2}(i_2) = i_2 \qq{für $k_2 > 1$ minimal}
	\]
	etc. Nach endlich vielen Schritten haben wir alle Nichtfixpunkte gefunden  und schreiben 
	 \[
		 \sigma = (i_1,\sigma(i_1),\ldots,\sigma^{k_1-1}(i_1))(i_2,\sigma(i_2),\ldots,\sigma^{k_2-1}(i_2))\ldots(i_{r},\sigma(i_{r}),\ldots,\sigma^{k_{r}-1}(i_{r}))
	.\] 
	In diesem Fall sagen wir auch, dass $\sigma$ \emph{Zyklentyp}(Struktur) $k_1,k_2,\ldots,k_{r}$ hat (wobei
	die Zahlen $k_1,\ldots,k_{r}$ auch in einer anderen Reihenfolge auftreten dürfen).
\end{notation}


\begin{proposition}
	Zwei Permutationen sind in $S_{n}$ genau dann konjugiert, falls sie dieselbe Zyklenstruktur haben.
\end{proposition}



\begin{theorem}
	$A_{n}$ und $S_{n}$ sind auflösbar für $n \leq 4$.
	$A_{n}$ ist einfach für $n \geq 5$.
\end{theorem}


Für $n \geq 5$ wollen wir die Gruppenwirkung von $A_{n}$ auf $\{1,\ldots,n\}$ und folgende Lemmas verwenden.

\begin{lemma}
	Sei $n \geq 3$. Dann ist die Wirkung von $A_{n}$ auf $\{1,\ldots,n\} $ transitiv.
\end{lemma}


\begin{lemma}
	Sei $n \geq 5$ und $H \lTri A_{n}$ nicht die triviale Gruppe. Dann enthält $H$ eine Permutation $\sigma \neq e$ mit mindestens einem Fixpunkt.
\end{lemma}




\section{Gruppen kleiner Ordnung \& Klassifikation}

\begin{theorem}
	Sei $G$ eine Gruppe der Ordnung $n = \abs{G} < 100$. 
	Dann ist entweder $G$ auflösbar der $n = 60$ und $G \simeq A_5$.
\end{theorem}

Für den Beweis des Satzes bedienen wir uns vieler bereits bewiesenen kleinen Lemmas,
dem Sylowsatz und weiteren Leamms mit zunehmender Komplexität.
Des Weiteren verwenden wir Induktion nach $n$ und einen grundlegene Eigenschaft von Auflösbarkeit.

\begin{definition}[Wiederholung]
	Sei $G$ eine Gruppe. Wir sagen $G$ ist \emph{auflösbar} falls es einen Subnormalreihe
	\[
	\{e\} = G_0 \lTri G_1 \lTri \ldots \lTri G_{k} = G
	\]
	gibt für die die Faktorgruppen $\frac{G_{j}}{G_{j-1}}$ für $j = 1,\ldots,k$ alle abelsch sind.
\end{definition}

\begin{proposition}[Legoeigenschaft und Auflösbarkeit]
	Sei $G$ eine Gruppe und $N \lTri G$. Falls $N$ und $\sfrac{G}{N}$ auflösbar sind, so gilt dasselbe für $G$.
\end{proposition}


%TODO missing rest of section see gruppen 21 and gruppenmoduln 22

\section{Freie Gruppen und Relationen}
\begin{definition}
	Sei $n \geq 1$ eine natürliche Zahl. Dann wird $\Z^{n}$ als die \emph{freie abelsche Gruppe} mit $n$ Erzeugenden  
	$b_1 = (1,0,\ldots,0)^{T}, \ldots , b_{n} = (0,\ldots,0,1)^{T}$ bezeichnet.
\end{definition}

\begin{lemma}
	Sei $G$ eine abelsche Gruppe und $a_1,\ldots,a_{n} \in G$. Dann gibt es einen eindeutig bestimmten 
	Gruppenhomomorphismus $\phi: \Z^{n} \to G$ mit $\phi(b_{j}) = a_{j}$ für $j = 1,\ldots,n$.
\end{lemma}


\begin{theorem}
	Sei $n \geq 1$ und $b_1,\ldots,b_{n}$ paarweise verschieden. Dann existiert eine \emph{\enquote{freie Gruppe} $F_{n}$},
	welche von $b_1,\ldots,b_{n}$ erzeugt wird, mit folgender \enquote{universeller} Eigenschaft:
	Für jede Gruppe $G$ und Elemente $a_1,\ldots,a_{n} \in G$ gibt es einen eindeutig bestimmten Homomorphismus 
	$\phi: F_{n} \to G$ mit $\phi(b_{j}) = a_{j}$ für $j=1,\ldots,n$.
\end{theorem}




































