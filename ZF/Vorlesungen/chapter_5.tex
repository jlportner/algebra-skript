%! TEX root=../algebra.tex
\graphicspath{{Images/}}

\chapter{Modultheorie}
(siehe Seite $288$, aber \enquote{kommutativ})

\section{Definition \& Beispiel}
\enquote{Modukb verhalten sich zu Ringen wie Vektorräume zu Körpern.}

\begin{definition}
	Sei $R$ ein Ring. Ein \emph{$R$-Modul $M$} ist eine abelsche gruppe gemeinsam mit einer Skalarmultiplikation $R \times M \to M, (a,m) \mapsto a \cdot m$
	mit folgenden Eigenschaften:
	\begin{itemize}
		\item $a \cdot (m_1 + m_2) = a m_1 + a m_2$ für $a \in R, m_1,m_2 \in M$.
		\item $(a+b) \cdot m = a m + b m$ für $a,b \in R, m \in M$.
		\item $a \cdot (b \cdot m) = (a b ) \cdot m$ für $a,b \in R, m \in M$.
		\item $1 \cdot m = m$ für $m \in M$.
	\end{itemize}
\end{definition}

\begin{definition}
	Seien $R$ ein Ring und $M,N$ $R$-Moduln. Wir sagen $\phi: M \to N$ ist \emph{$R$-linear} (ein \emph{Modulmorphismus über $R$})
	falls $\phi$ ein Gruppenmorphismus ist und $\phi(a m) = a \phi(m)$ für alle $a \in R$ und $m \in M$.
\end{definition}

\begin{definition}
	Sei $R$ ein Ring und $M$ ein $R$-Modul. Ein \emph{Untermodul} ist eine Untergruppe $N < M$ mit $a \cdot n \in N$ für alle $a \in R$ und $n \in N$.
\end{definition}

\begin{lemma}
	Sei $R$ ein Ring, $M$ ein $R$-Modul und $N < M$ ein Untermodul. Dann induziert die $R$-Modulstruktur auf $M$ eine $R$-Modulstruktur auf $\sfrac{M}{N}$ so dass
	die kanonische Projektion $\begin{cases}
		\pi: M \to \sfrac{M}{N}\\ m \mapsto m+N
	\end{cases}$ $R$-linear ist.
\end{lemma}



\begin{proposition}[Erster Isomorphiesatz]
	Seien $R$ ein Ring, $M,N$ $R$-Moduln, $\phi: M \to N$ $R$-linear.
	Dann sind $\ker(\phi) < M, \Im(\phi) < N$ Untermoduln und $\phi$ induziert einen $R$-linearen Isomorphismus
	\[
		\overline{\phi}: \sfrac{M}{\ker(f)} \to \Im(f)
	.\] 
\end{proposition}

\begin{lemma}
	Seien $R$ ein Ring und $M_1,\ldots,M_{n}$ $R$-Moduln.
	Dann ist auch $M_1 \times \ldots \times M_{n}$ ein $R$-Modul mit koordinatenweiser Skalarmultiplikation
	\[
		a \cdot (m_1,\ldots,m_{n}) = (a m_1,\ldots,a m_{n}) \qq{für} a \in R, (m_1,\ldots,m_{n}) \in M_1 \times \ldots \times M_{n}
	.\] 
\end{lemma}


\begin{lemma}
	Seien $R,S$ zwei Ringe, $M$ ein $R$-Modul und $N$ ein $S$-Modul.
	Dann ist $M \times N$ ein $R \times S$-Modul mit koordinatenweiser Skalarmultiplikation
	\[
		(a,b) \cdot (m,n) = (a m, b n) \qq{für} (a,b) \in R \times S, (m,n) \in M \times N
	.\] 
\end{lemma}


\textbf{Übung:}
Charakterisiere die Untermoduln von $M \times N$ (über $R \times S$ ).

Welche Ringe könnten interessant sein?
\[
	\text{Körper} \to \text{Vektorräume} \qquad \Z \to \text{Abelsche Gruppen} \qquad K[X] \to ?
\] 
\begin{theorem}
	Sei $K$ ein Körper und $M$ ein Vektorraum über $K$.
	Die Definition einer Modulstruktur auf $M$ über $K[X]$ (die mit der Vektorraumstruktur von $M$ über $K$ kompatibel ist)
	ist gleichzusetzen mit der Auswahl einer $K$-linearen Abbildung $\varphi: M \to M$.
	Formaler formuliert sind die folgenden beiden Abbildungen invers zueinander:
	\begin{center}
		\begin{tabu} to \linewidth {X[2] X[0.2] X[2]}
			Eine Skalarmultiplikation auf $M$ über $K[X]$ dessen Einschränkung auf $K \times M$ die Skalarmultiplikation von $M$ über $K$ ist. & & 
			Eine $K$-lineare Abbildung $ \varphi: M \to M$ \\
			\qquad \qquad \qquad \qquad \qquad  $\cdot $ & $\longmapsto $ & $\varphi(m) = X \cdot m$ für $m \in M$ \\
			$f \cdot m = (f(\varphi))(m) = (\sum_{k} a_{k} \varphi^{k}) (m)$ für $f = \sum_{k} a_{k} X^{k} \in K[X]$ & $\longmapsfrom$ & $\varphi$
		\end{tabu}
	\end{center}
\end{theorem}


Wir wollen endlich erzeugte Moduln über Hauptidealringen klassifizieren!\\
$\overset{\Z}{\longrightarrow}$ Klassifikation von endlich erzeugten abelschen Grupppen.\\
$\overset{K[X]}{\longrightarrow}$ Satz über Jordan Normalform.

\section{Freie Moduln}

\begin{definition}
	Sei $I$ eine Menge und $R$ ein Rnig. Wir bezeichnen
	\[
		R^{(I)} = \{x : I \to R \mid x_{i} = 0 \text{ für alle bis auf endlich viele } i \in I\} 
	\] 
	als den \emph{freien $R$-Modul} (über der Indexmenge $I$ ). Wir nennen
	\[
		e_{i} = \mathbbm{1}_{\{i\}} \qq{für} i \in I
	\]
	die \emph{Standardbasis} von $R^{(I)}$.
	Ein \emph{freier Modul} $M$ ist ein Modul isomorph zu $R^{(I)}$ für eine Menge $I$.
	Die Kardinalität von $I$ wird als der \emph{Rang} von $M \cong R^{(I)}$ bezeichnet.
\end{definition}

\begin{lemma}
	Sei $R \neq \{0\} $ ein Ring. Dann ist der Rang eines Moduls wohldefiniert.
\end{lemma}


\begin{claim}
	Freie Moduln verhalten sich am ehesten wie Vektorräume \ldots
\end{claim}

\begin{proposition}
	Seien $m,n \geq 1$ natürliche Zahlen und $R$ ein Ring. Dann gilt
	\[
		\hom(R^{n},R^{m}) \cong \mat_{mn}(R)
	\] 
	wie in der Linearen Algebra.
\end{proposition}


\begin{definition}
	Sei $M$ ein $ R$-Modul über einem Ring $R$. Wir sagen $x_1,\ldots,x_{n} \in M$ sind \emph{frei} oder \emph{linear unabhängig} (l.u.) falls die Abbildung
	$a \in R^{n} \mapsto \sum_{i = 1}^{n} a_{i} x_{i}$ injektiv ist.

	Falls $x_1,\ldots,x_{n} \in M$ l.u. sind, so ist das Bild der Abbildung ein freier Untermodul von $M$.
\end{definition}

\section{Torsionsmoduln}
\begin{definition}
	Sei $R$ ein Ring und $M$ ein $R$-Modul. Wir sagen $m \in M$ ist ein \emph{Torsionselement}, falls es ein $a \in R \setminus \{0\} $ gibt
	mit $a \cdot m = 0$. Wir sagen $M$ ist ein \emph{Torsionsmodul} falls jedes $m \in M$ ein Torsionselement ist.
	Wir sagen $M$ ist \emph{torsionsfrei} falls $m = 0$ das einzige Torsionselement von $M$ ist.
\end{definition}


\section{Struktur von endlich erzeugten Moduln über Hauptidealringe}

\begin{definition}
	Sei $R$ ein Ring und $M $ ein $R$-Modul.
	Für eine Teilmenge $X \subseteq M$ wird
	\[
	\left< X \right>_{R} = \{\sum_{x \in E} a_{x} x \mid a_{x} \in R \text{ für $x \in E$ und $E \subseteq X$ endlich}\} 
	\]
	als die \emph{$R$-lineare Hülle von $X$} oder als der \emph{von $X$ erzeugte Untermodul} bezeichnet.
	Falls es eine Teilmenge $X \subseteq M$ mit $\abs{X} < \infty$ und $\left< X \right>_{R} = M$ gibt, so heißt $M$ \emph{endlich erzeugt}.
\end{definition}


Wir wollen ab nun nur Hauptidealreinge betrachten - dort wäre jeder Untermodul von $R$ wieder frei mit Rang $0$ oder $1$.

\begin{theorem}[Klassifikationssatz (1. Teil)]
	Sei $R$ ein Hauptidealring und $M$ ein endlich erzeugter Modul über $R$.
	Dann ist $M$ isomorph zu einem direkten Produkt $R^{n} \times T$ wobei
	\[
		T = M_{\on{tors}} = \{m \in M \mid m \text{ ist ein Torsionselement von } M\}
	\]
	und $n$ ist der Rang von $\sfrac{M}{M_{\on{tors}}}$.
	Insbesondere ist $M$ ein freier Modul genau dann wenn $M_{\on{tors}} = \{0\}$.
\end{theorem}

\begin{proposition}
	Sei $R$ ein Hauptidealring und $n \geq 1$. Dann ist jeder Untermodul $M \subseteq R^{n}$ ein freier $R$-Modul mit Rang $\leq n$.
\end{proposition}



\begin{theorem}[Klassifikationssatz (2. Teil)]
	Sei $R$ ein Hauptidealring und $M_{\on{tors}}$ ein endlich erzeigter Torsionsmodul. 
	Dann existieren $d_1 \mid d_2 \mid \ldots \mid d_{n}$ in $R \setminus \{0\} $ so dass
	\[
		M_{\on{tors}} = \sfrac{R}{(d_1)} \times \ldots \times \sfrac{R}{(d_{n})}
	.\] 
	Alternativ gilt
	\[
		M_{\on{tors}} \cong \prod_{j=1}^{k} M_{\on{tors}}^{(p_{j})}
	\]
	wobei $p_1,\ldots,p_{k} \in R$ inäquivalente Primzahlen in $R$ sind und %TODO was ist inäquivalent ist das richtig
	\[
		M_{\on{tors}}^{(p_{j})} = \{m \in M_{\on{tors}} \mid \text{ es existiert ein $l \in \N$ mit } p_{j}^{l} m = 0\} \cong 
		\sfrac{R}{(p_{j}^{n_{j,1}})} \times \ldots \times \sfrac{R}{(p_{j}^{n_{j,n}})}
	.\] 
\end{theorem}

\begin{theorem}[Smith Normalform]
	Sei $R$ ein Hauptideal ring, $k,l \geq 1$ natürliche Zahlen und $A \in \mat_{kl}(R)$.
	Dann existieren $g \in \GL_{k}(R)$ und $h \in \GL_{l}(R)$ so dass
	\[
	g A h^{-1} = \begin{pmatrix} 
		d_1 \\
		& \ddots \\
		& & d_{n} \\
		& & & 0 \\
		& & & & \ddots
	\end{pmatrix} 
	\]
	für $d_1 \mid d_2 \mid \ldots \mid d_{n}$ in $R \setminus \{0\} $.
\end{theorem}

\begin{itemize}
	\item Wir beweisen diesen Satz nur für Euklidische Ringe.
	\item Im Gauss'schen Eliminationsalgo entsprechen Zeilenoperationen einer Linksmultiplikation und Spaltenoperationen einer Rechtsmulitplikation.
	\item Wir kombinieren Gauss mit Division mit Rest.
	\item Falls $R = K$ ein Körper ist, so können wir $d_1 = d_2 = \ldots = d_{n} = 1$ annehmen und $n = $ Rang von  $A$.
\end{itemize}



\section{Endlich erzeugte abelsche Gruppen}
\begin{theorem}
	Sei $G$ eine endlich erzeugte (additiv geschriebene) abelsche Gruppe.
	Dann gilt
	\[
		G \cong \sfrac{\Z}{(d_1)} \times \ldots \times \sfrac{\Z}{(d_{n})} \times \Z^{k}
	\] 
	wobei $1 \leq s_1 \mid d_2 \mid \ldots \mid d_{n} \neq 0$ und $k \geq 0$.
	
	Alternativ gilt
	\[
		G \cong \prod_{p > 0 \text{ prim}} G_{p} \times \Z^{k} \qq{und} G_{p} \cong \sfrac{\Z}{(p^{k_{p,1}})} \times \ldots \times \sfrac{\Z}{(p^{k_{p,n}})}
	.\] 
	wobei $G_{p}$ die Sylow $p$-Untergruppe ist.
\end{theorem}


\section{Jordan-Normalform}
\begin{theorem}
	Sei $V$ ein endlich dimensionaler Vektorraum über $\C$ und $\varphi: V \to  V$ linear.
	Dann existiert eine Basis von $V$, so dass $\varphi$ eine Matrixdarstellung der folgenden Form besitzt:
	\[
	\begin{pmatrix} 
		J_1\\
		&J_2\\
		& & \ddots
	\end{pmatrix} \qq{und jeder Block $J_{k}$ hat die Form} \begin{pmatrix} 
		\lambda &1\\
		&\lambda &1\\
		& & \ddots &\ddots\\
		& & & \ddots &1\\
		& & & &\lambda
	\end{pmatrix} 
	.\]
	Dies ist die Jordan-Normalform von $\varphi$.
\end{theorem}





















