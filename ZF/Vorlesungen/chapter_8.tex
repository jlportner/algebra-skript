%! TEX root=../algebra.tex
\graphicspath{{Images/}}

\chapter{Lösung durch Radikale und auflösbare Gruppen}

Sei $K = k(u)$ eine Körpererweiterung von $k$, $u \neq 0$.
Dann ist $\{n \in \Z \mid u^{n} \in k\}$ ist eine Untergruppe von $\Z$ und deshalb von der Form $m \Z$ wobei $m \in \N$
eindeutig bestimmt.

\begin{definition}
	$k(u) / k$ ist eine reine Erweiterung vom Typ $m$ falls $m \Z = \{n \in \Z \mid u^{n} \in k\} \neq 0$
\end{definition}

\begin{definition}
	Eine Körpererweiterung $K / k$ heißt radikal falls es einen Turm von Zwischenkörpern gibt
	\[
	k = K_0 \subseteq K_1 \subseteq \ldots \subseteq K_{t} = K
	\] 
	so dass $K_{i+1} / K_{i} \forall 0 \leq i \leq t-1$ reine Erweiterungen sind.
\end{definition}

\begin{definition}
	Ein Polynom $f \in k[x]$ ist mittels \emph{redikalen Lösbar} falls ein Zerfällungskörper von $f$ in einer radikalen Erweiterung
	von $k$ enthalten ist.
\end{definition}

$k(u) / k$ : $u^{m} \in k$ $u$ ist $m$-te Wurzel von einem Element in $k$.
Sei $E$ der Zerfällungskörper von $f$.
%TODO missing 8:28 26.03.2021


Sei $k(u) / k$ eine reine Erweiterung von Typ $m \geq 1$.
Sei $m = p_1 \cdot \ldots \cdot p_{r}$ eine Zerlegung in Primzahlen.
\[
	k(u) \supseteq k(u^{p_1}) \supseteq k(u^{p_1 p_2} \supseteq \ldots \supseteq k(u^{m}) = k
\] 
wobei die erste Erweiterung von Typ $p_1$, die zweite von Typ $p_2$ etc. ist.
Dies Führt zum Studium von $x^{p} - c \in k[x]$.

\begin{lemma}
	Sei $p$ ein Primzahl. Sei $f(x) = x^{p}-c \in k[x]$.
	\begin{enumerate}[(1)]
		\item Folgende Dichotomie:
			\begin{enumerate}[({1}.1)]
				\item $(f)$ ist irreduzibel
				\item $c$ ist eine $p$-te Potenz eines Elements in $k$
			\end{enumerate}
		\item Sei $E / k$ der Zerfällungskörper von $f$. Wir nehmen an, $k$ enthält alle $p$-ten Wurzeln von $1$.
			Sei $u \in E, u \in R(f)$. Dann ist $E = k(u)$.
			\begin{enumerate}[(2.1)]
				\item $f$ irreduzibel:
					\begin{itemize}
						\item Falls $\charak(k) \neq p$ ist $\gal(E / k) \cong \sfrac{\Z}{p \Z}$ 
						\item Falls $\charak(k) = p$ ist $\gal(E / k) \cong e$.
					\end{itemize}
				\item $f$ reduzibel so ist $E = k$ und $\gal(E / k) \cong (e)$.
			\end{enumerate}
	\end{enumerate}
\end{lemma}



Sei $f \in k[x]$. $k \subseteq E \subseteq K$ mit $E$ Zerfällungskörper, $K$ Radikale Erweiterung. 
In Verbindung bringen mit Galois Gruppe.
Wir wollen zeigen, dass jede radikale Erweiterung $K / k$ in einer normalen radikalen Erweiterung $F$ enthalten ist.
\[
k \subseteq E \subseteq K \subseteq F
\] 
normal und Radikal. Aus Satz 2.26 folgt $\nstack{\gal(F / k) \to \gal( E / k)}{\sigma \mapsto \sigma \mid_{E}}$ surjektiv.
Falls wir zeigen, dass $\gal(F / k)$ von $\frac{f}{k}$ normal radikal auflösbar ist.
Dann folgt, dass $\gal(E / k)$ auflösbar ist.
In Algebra I hatten wir den Satz
\begin{theorem}
	Jede Untergruppe und jeder Quotient einer auflösbaren Gruppe ist auflösbar.
\end{theorem}

Kontext folgender zwei Lemmata: Sei $B = k(u_1,\ldots,u_{t})$ eine endliche Erweiterung von $k$.
Insbesondere sind $u_1,\ldots,u_{t}$ algebraisch über $k$.
Sei $p_{i} = \irr(u_{i},k) \in k[x]$ das Minimalpolynom von $u_{i}$ über $k$.
Sei $f = p_1 \ldots p_{t} \in k[x]$. Sei $E$ Zerfällungskörper von $f$ und $G = \gal(E / k) = \{\sigma_1,\ldots,\sigma_{l}\}	$.

\begin{lemma}
	$E = k(\sigma(u_1),\ldots,\sigma(u_{t}), \sigma \in G) = k \begin{pmatrix} 
		\sigma_1(u_1), &\ldots, &\sigma_{l}(u_1)\\
		\vdots & &\vdots\\
		\sigma_1(u_{t}), &\ldots, &\sigma_{l}(u_{t})
	\end{pmatrix} $
\end{lemma}


\begin{lemma}
	Im Kontext von Lemma 3.6 nehmen wir an, dass: $u_1^{m_1} \in k, u_2^{m_2} \in k(u_1),\ldots, u_{t}^{m_{t}} \in k(u_1,\ldots,u_{t-1})$.
	Dann ist $E / k$ eine radikale Erweiterung.
\end{lemma}


\begin{corollary}
	Sei $K / k$ eine radikale Erweiterung. Dann gibt es $k \subseteq K \subseteq F, F / k$ radikal und normal.
\end{corollary}


\begin{definition}[Algebra I]
	Eine Gruppe $G$ ist auflösbar falls es eine subnormale Folge
	\[
	\{e\} = G_0 \lTri G_1 \lTri g_2 \lTri \ldots \lTri G_{t} = G
	\]
	gibt mit $\sfrac{G_{i+1}}{G_{i}}$ abelsch $0 \leq i \leq t-1$.
\end{definition}


Es gibt ein Kriterium für Auflösbarkeit, dass iterierte Kommutatorunterguppen benützt.
Für eine Gruppe $G$ bezeichnet $[G,G]$ die von $\{[a,b] \mid a,b \in G\} $ erzeugte Untergruppe.
Hier ist $[a,b] = a b a^{-1} b^{-1}$.
Die Untergruppe $[G,G]$ ist \emph{charakteristisch} d.h. $\forall \alpha \in \aut(G)$ ist $\alpha([G,G]) = [G,G]$.

Wir führen folgende Notation ein $G^{(1)} = [G,G] = $ Kommutatorgruppe, $G^{(j)} = [G^{(j-1)},G^{(j-1)}]$.
\begin{proposition}
	$G$ ist genau dann auflösbar falls es $n$ gibt mit $G^{(n)} = (e)$.
\end{proposition}


\begin{proposition}
	\begin{enumerate}[(1)]
		\item $H < G$ : $G$ auflösbar $\implies H$  auflösbar.
		\item $N \lTri G$ : $G$ ist gdw. auflösbar falls $N$ und $\sfrac{G}{N}$ auflösbar ist.
	\end{enumerate}
\end{proposition}


\begin{theorem}
	Sei $f \in k[X]$, $E$ ein Zerfällungskörper von $f$. Falls $f$ mittels Radikalen lösbar ist, folgt, dass $\gal(E / k)$ auflösbar ist.
\end{theorem}

\begin{lemma}
	Sei $k = K_0 \subseteq K_1 \subseteq \ldots \subseteq K_{t}$ ein Turm von Erweiterungen wobei
	\begin{enumerate}[(1)]
		\item $K_{t} / k$ normale Erweiterung
		\item $K_{i}$ ist eine reine Erweiterung von Primzahlen $p_{i}$ mit $1 \leq i \leq t$.
		\item $k$ enthält alle $p_{i}$-ten Wurzeln von $1$, $1 \leq i \leq t$.
	\end{enumerate}
	Dann ist $\gal(K_{t} / k)$ auflösbar.
\end{lemma}



\begin{corollary}[Abels-Ruffini]
	Für $n \geq 5$ ist das \enquote{allgemeine Polynom}
	\[
		f(x) = \prod_{i=1}^{n} (X-y_{i})
	\] 
	mittels Radikalen nicht lösbar.
\end{corollary}


\begin{corollary}
	$f(x) = x^{5} - 4x + 2 \in \Q[x]$ ist nicht mittels Radikalen lösbar, da $\gal(f) \cong S_{5}$.
\end{corollary}

Sei $R$ ein angeordneter Körper mit
\begin{enumerate}
	\item jedes $x \geq 0$ ist ein Quadrat
	\item jedes $P \in R[X]$ mit  $\deg(P)$ ungerade hat eine Nullstelle in $R$
\end{enumerate}
dann ist $R(\sqrt{-1})$ algebraisch abgeschlossen.





























