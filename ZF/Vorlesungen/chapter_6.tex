%! TEX root=../algebra.tex
\graphicspath{{Images/}}

\chapter{Körpertheorie}

\section{Körpererweiterungen}
\begin{remark}
	Ein Ringhomomorphismus $\varphi: K \to L$ von einem Körper zu einem anderen Körper ist immer injektiv 
	%TODO missing
\end{remark}

\begin{definition}
	Sei $L$ ein Körper und $K \subseteq L$ ein Unterring und auch ein Körper.
	Dann heißt $K \subseteq L$ auch ein \emph{Unterkörper} und $L$ wird eine \emph{Körpererweiterung} von  $K$ genannt.
	Wir schreiben auch $L \mid K$ (\enquote{$L$ über $K$}) falls $L$ eine Körpererweiterung von $K$ ist.
	Da $L$ in diesem Fall ein Vektorraum über $K$ ist, können wir die Dimension von $L$ über $K$ betrachten -
	diese wir als der \emph{Grad $[L:K]$ der Körpererweiterung  $L \mid K$} bezeichnet.
	Falls $[L:K] < \infty$, so heißt $L$ eine \emph{endliche Körpererweiterung von $K$}.
\end{definition}


\begin{theorem}[Multiplikativität dere Grade]
	Angenommen $F \mid L $ und $L \mid K$ sind (endliche) Körpererweiterungen.
	Dann gilt $[F:K] = [F:L] [L:K]$.
\end{theorem}


\begin{definition}
	Sei $L \mid K$ eine Körpererweiterung, $x \in L$, und $\varphi_{x}: K[T] \to L, f \mapsto f(x)$ der Auswertungshomomorphismus.

	Falls $\varphi_{x}$ injektiv ist, so heißt $x$ \emph{transzendent} über $K$ 

	Falls $\varphi_{x}$ nicht injektiv ist, so heißt $x$ \emph{algebraisch} über $K$.
	In diesem Fall ist $\ker(\varphi_{x}) = (m_{x}(T))$ \& $m_{x}(T)$ heißt das
	\emph{Minimalpolynom von $X$}, der Grad von $m_{x}(T)$ ist auch der \emph{Grad von $X$}.
\end{definition}


\begin{proposition}
	Sei $L \mid K$ und $x \in L$. Falls $x$ transzendent ist, so ist 
	\[
		K[X] = \Im(\varphi_{x}) \cong K[T]
	.\] 
	und der kleinste Unterkörper $K(X)$ von $L$, der sowohl $K$ als auch $x$ enthält ist, erfüllt
	\[
		K(X) \cong K(T)
	\] 
	mit  $K(T)$ der Körper der rationalen Funktionen.

	Falls $x$ algebraisch ist, so ist
	\[
		K[X] = \Im(\varphi_{x}) \cong \sfrac{K[T]}{(m_{x}(T))}
	\] 
	bereits der kleinste Unterkörper $K(X)$, der sowohl $K$ als auch $e$ enthält.
	Es gilt
	\[
		[K(x):K] = \deg(m_x(T))
	.\] 
\end{proposition}


\begin{definition}
	Sei $L \mid K$ und $x_1,\ldots,x_{n} \in L$. Dann bezeichnen wir den kleinsten Unterkörper der sowohl $K$ als auch $x_1,\ldots,x_{n}$ enthält mit
	\[
		K(x_1,\ldots,x_{n}) = \{\frac{f(x_1,\ldots,x_{n})}{g(x_1,\ldots,x_{n})} \mid f,g \in K[T_1,\ldots,T_{n}], g(x_1,\ldots,x_{n}) \neq 0\} 
	.\] 
\end{definition}

\begin{corollary}[Wantzel, 1837]
	Mit Zirkel und Linear lassen sich weder $\sqrt[3]{2}$ noch ein Winkel von $29^{\circ}$ konstruieren.
	Des Weiteren gilt: Falls $p > 2$ eine Primzahl ist und das regelmäßige $p$-Ecke mit
	Zirkel und Lineal konstruierbar ist, so ist $p$ eine Fermat-Primzahl ($p-1 = 2^{2^{n}}$ ).
\end{corollary}


\begin{definition}
	Eine Körpererweiterung $L \mid K$ heißt \emph{algebraisch} falls jedes $x \in L$ algebraisch über $K$ ist.
\end{definition}

\begin{lemma}
	Eine endliche Körpererweiterung ist algebraisch.
\end{lemma}


\begin{corollary}
	Sei $L \mid K$ und $x,y \in L$ algebraisch über $K$. Dann sind auch $x+y, x\cdot y, x-y, \frac{1}{x}$ für $x\neq 0$ algebraisch über $K$.
\end{corollary}


\begin{corollary}
	Angenommen $F \mid L$ und $L \mid K$. Dann ist \emph{$F \mid K $ ist algebraisch} genau dann wenn  \emph{$F \mid L$ algebraish ist und $L \mid K$ algebraisch ist}.
\end{corollary}



\section{Zerfällungskörper}

\begin{theorem}[Kronecker]
	Sei $K$ ein Körper, $f \in K[T]$ mit $n = \deg(f) > 0$. Dann existiert eine Körpererweiterung $L$ von $K$, so dass
	\[
		f(T) = a \prod_{i=1}^{n} (T-\alpha_{i}),
	\] 
	$a \in k$, $\alpha_1,\ldots,\alpha_{n} \in L$.
\end{theorem}



\begin{definition}
	Sei $K$ ein Körper, $f \in K[T]$ mit $\deg(f) > 0$.
	Ein \emph{Zerfällungskörper von $f$ über $K$ } ist eine Körpererweiterung $L \mid K$ so dass
	\begin{enumerate}[1)]
		\item $f$ zerfällt (in Linearfaktoren) in $L[i]$.
		\item Falls $K \subseteq E \subsetneq L$, dann zerfällt $f$ über $E$ nicht.
	\end{enumerate}
\end{definition}

\begin{remark}
	\begin{itemize}
		\item Ein Zerfällungskörper existiert immer (und ist bis auf Isomorphie eindeutig).
			Falls $f \in K[T]$ und $F \mid K$ eine Körpererweiterung, so dass $f$ in $F[T]$ zerfällt (Kronecker)
			mit Nullstellen $\alpha_1,\ldots,\alpha_{n} \in F$ so ist $L := K(\alpha_1,\ldots,\alpha_{n})$ ein Zerfällungskörper.
		\item Ein Zerfällungskörper ist eine algebraische Körpererweiterung von $K$.
	\end{itemize}
\end{remark}


\begin{remark}
	Sei $K$ ein Körper, $f \in K[T]$ und $L$ ein Zerfällungskörper von $f$ über $K$, dann gilt
	\[
		[L:K] \leq (\deg(f))!
	.\]
	Ist $f$ über $K$ irreduzibel, so gilt $[L:K] \geq \deg(f)$.
	\begin{itemize}
		\item $T^3-2$ irreduzibel über $\Q$ mit Grad $6$.
		\item $T^2+1$ irreduzibel über $\Q$ mit Grad $2$.
		\item $T^3-2$ nicht irreduzibel über $\R$ und hat Zerfällungskörper mit Grad $2$.
	\end{itemize}
\end{remark}

\section{Algebraischer Abschluss}
\begin{definition}
	Sei $K$ ein Körper. $K$ ist \emph{algebraisch abgeschlossen}, falls jedes Polynom $f \in K[T]$ 
	mindestens eine Nullstelle in $K$ hat.
\end{definition}

Es folgt (Induktion), dass $f$ über $K$ zerfällt.

\begin{remark}
	Ein algebraisch abgeschlossener Körper hat unendlich viele Elemente.
\end{remark}

\begin{proposition}
	Sei $L \mid K$ eine Körpererweiterung und $L$ algebraisch abgeschlossen.
	Dann ist 
	\[
		E = \{x \in L \mid x \text{ ist algebraisch über } K\}
	\]
	eine algebraisch abgeschlossene algebraische Körpererweiterung von $K$.
\end{proposition}

\begin{definition}
	Wir nennen $E$ wie in der Proposition den \emph{algebraischen Abgschluss $\overline{K}$ } von $K$
\end{definition}


\begin{remark}
	\begin{itemize}
		\item $K$ endlich $\implies \overline{K}$ ist abzählbar
		\item $K$ abzählbar $\implies \overline{K}$ ist abzählbar [Bsp: $\Q, \overline{\Q} = \Q_{\on{alg}} = \{z \in \C \mid z \text{ alg. über } \Q\}$ genannt algebraische Zahlen]
	\end{itemize}
\end{remark}

\begin{theorem}
	Sei $K$ ein Körper, dann existiert eine Körpererweiterung $L \mid K$ mit $L$ algebraisch abgeschlossen ($L$ ist bis auf Isomorphie eindeutig).
\end{theorem}


\section{Eindeutigkeit}
(Seite $343$, Teile auch Seite $88$ )

Wir haben gesehen: 
\begin{itemize}
	\item Für jedes $f \in K[T]$ gibt es einen Zerfällungskörper.
	\item Es gibt einen algebraischen Abschluss.
\end{itemize}
Sind diese (bis auf Isomorphie) eindeutig?

\begin{theorem}
	Sei $K$ ein Körper, $L \mid K$ eine Körpererweiterung und $L$ algebraisch abgeschlossen.
	\begin{enumerate}
		\item Falls $E = K[\alpha]$ eine endliche Körpererweiterung von $K$ ist, so gibt es mindestens eine
			und höchstens $[E:K]$ Körpereinbettungen $\sigma: E \to L$ mit $\underbrace{\sigma \mid_{K} = \id_{K}}_{\sigma \; K\text{-linear}}$.
			Falls $\charak(K)=0$, so gibt es genau $[E:K]$ derartige Einbettungen.
		\item Falls $E \mid K$ eine algebraische Körpererweiterung ist, so gibt es eine $K$-lineare Körpereinbettung
			$\sigma: E \to L$.
	\end{enumerate}
\end{theorem}

\begin{lemma}
	Sei $K$ eine Körper, $m(T) \in K[T]$ coprim zu $m'(T)$. Dann hat $m$ in einer algebraisch abgeschlossenen Körpererweiterung
	genau $\deg(m(T))$ viele einfache Nullstellen.

	Dies gilt z.B. wenn $\charak(K) = 0$ und $m(T)$ irreduzibel in $K[T]$ ist.
\end{lemma}


\begin{remark}
	 Für $K = \F_{p}$ und $m(T) = T^{p}$ gilt $m'(T) = 0$ und daher nicht $\deg(m'(T)) = \deg(m(T)) -1$.
\end{remark}




\begin{corollary}
	Sei $K$ ein Körper
	\begin{enumerate}[1)]
		\item Für jedes $f \in K[T]$ ist die Zerfällungskörper bis auf einen $K$-linearen Körperisomorphismus eindeutig bestimmt.
		\item Je zwei algebraische Abschlüsse von $K$ sind $K$-linear isomorph.
	\end{enumerate}
\end{corollary}


\section{Endliche Körper}
$\F_{p} = \sfrac{\Z}{(p)}$ für $p \in \N$ prim ist ein endlicher Körper.\\
Gibt es weitere? Können wir diese klassifizieren?

\begin{theorem}[Gauss, Galois]
	\begin{enumerate}
		\item Falls $K$ ein endlicher Körper ist, so ist $\abs{K} = p^{n}$ für eine Primzahl $p \in \N$ und ein $n \geq 1$.
		\item Für jede Primzahlpotenz $p^{n}$ gibt es einen bis auf Isomorphie eindeutig bestimmten Körper mit $p^{n}$ Elementen.
		\item Sei $p \in \N$ prim und $K$ ein algebraischer Abschluss von $\F_{p}$. Dann enthält $K$ einen eindeutig bestimmten
			Unterkörper $\F_{p^{n}}$ mit $p^{n}$ Elementen.
			\[
				\F_{p^{n}} = \{x \in K \mid x^{\left( p^{n} \right) = x}\} 
			.\] 
		\item Für $m,n \geq 1$ und die Körper wie in $3)$ gilt
			\[
			F^{p^{m}} \subseteq F^{p^{n}} \Leftrightarrow m \mid n
			.\] 
	\end{enumerate}
\end{theorem}


\begin{theorem}
	Sei $K$ ein Körper und $G \subseteq K^{\times}$ eine endliche Untergruppe.
	Dann ist $G$ zyklisch. Insbesondere ist $\F_{p^{n}}^{\times}$ zyklisch für jede Primzahlpotenz $p^{n}$.
\end{theorem}


\begin{corollary}
	Sei $p > 2$ eine Primzahl. Für $a \in \F_{p}$ gilt 
	\[
		a^{\frac{p-1}{2}} = \begin{cases}
			0 &\text{ falls } a = 0\\
			1 &\text{ falls } a = b^{2} \text{ für ein } b \in \F_{p}^{\times}\\
			-1 &\text{ sonst }
		\end{cases}
	\]
\end{corollary}



































 
